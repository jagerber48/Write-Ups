\documentclass[12pt]{article}
\usepackage{amssymb, amsmath, amsfonts}

\usepackage{tcolorbox}

\usepackage{bbm}
\usepackage[utf8]{inputenc}
\usepackage{subfigure}%ngerman
%\usepackage[pdftex]{graphicx}
\usepackage{textcomp} 
\usepackage{color}
\usepackage[hidelinks]{hyperref}
\usepackage{anysize}
\usepackage{siunitx}
\usepackage{verbatim}
\usepackage{float}
\usepackage{braket}
\usepackage{xfrac}
\usepackage{array, booktabs} 
\usepackage{tabularx}


\newcommand{\ddt}[1]{\frac{d #1}{dt}}
\newcommand{\ppt}[1]{\frac{\partial #1}{\partial t}}
\newcommand{\ep}{\epsilon}
\newcommand{\sinc}{\text{sinc}}
\newcommand{\bv}[1]{\boldsymbol{#1}}
\newcommand{\ahat}{\hat{a}}
\newcommand{\adag}{\ahat^{\dag}}
\newcommand{\braketacomm}[1]{\left\langle\left\{#1\right\} \right\rangle}
\newcommand{\braketcomm}[1]{\left\langle\left[#1\right] \right\rangle}
\newcommand{\ketbra}[2]{\Ket{#1}\!\Bra{#2}}


\begin{document}
\title{Time Evolution Example Problems}
\author{Justin Gerber}
\date{\today}
\maketitle

\section{Introduction}

In the `Time Evolution in Quantum Mechanics' write up I laid out a formalism to approach questions about the time evolution of quantum systems.
The formalism presents a `picture/frame' independent representation of the Born rule as the fundamental postulate for the calculation of time evolution in quantum mechanics.
It then goes on to present multiple mathematical approaches to calculating the Born rule expression including a Schrodinger, Heisenberg, and multiple interaction pictures.

As a reminder the Born rule expression under consideration is something like

\begin{align}
\braket{A}_t = \bra{\psi_0}T^{\dag} A_0 T\ket{\psi_0}
\end{align}

Where $\ket{\psi_0}$ is the initial state, $A$ is the operator of interest, and $T$ is the time evolution operator.

In this document I will present the application of each of these different pictures to different canonical quantum mechanical problems.

In the first part of this document I will calculate the time evolution of a magnetic moment precessing in a magnetic field in the Schrodinger and then Heisenberg picture.
In the second part I will calculate the dynamics of a coherently driven two level system in both the ket and operator interaction pictures.

\section{Precision of a Magnetic Moment}

The Hamiltonian for a magnetic moment in a magnetic field is given by

\begin{align}
H = -\bv{\mu}\cdot\bv{B}
\end{align}

The magnetic moment under consideration is often the magnetic moment arising from the intrinsic spin of a quantum mechanical particle.
If the particle has total angular momentum $\bv{F}$ then the associated magnetic moment is given by

\begin{align}
\bv{\mu} = \gamma_{gy}\bv{F}
\end{align}

Where $\gamma_{gy}$ is the gyromagnetic ratio for the particle.
The gyromagnetic ratio is given by

\begin{align}
\gamma_{gy} =& \gamma_c g_F\\
\gamma_c =& -\frac{|e|}{2m_e} = -\SI{1.4}{\MHz \per G}
\end{align}

Here $\gamma_c$ is the classical gyromagnetic ratio which has a value of $-\SI{1.4}{\MHz \per G}$.
From this we see

\begin{align}
H = -\gamma_{gy} \bv{B}\cdot\bv{F} = -g_F \gamma_C \bv{B}\cdot\bv{F}
\end{align}

We know from quantum mechanics that particle angular momentum comes in quantized units of $\hbar$. This motivates the definition of a dimensionless angular momentum

\begin{align}
\bv{f} = \frac{\bv{F}}{\hbar}
\end{align}

Here $\bv{F}$ has dimensions of angular momentum while $\bv{f}$ is dimensionless. 
$\bv{F}$ can be thought of as physical angular momentum while $\bv{f}$ is a measure of number of quanta of angular momentum (in units of $\hbar$).
From this we see

\begin{align}
\bv{\mu} =& \gamma_{gy}\hbar \bv{f}\\
=& g_F \gamma_c \hbar\bv{f}\\
=& -g_F \mu_B \bv{f}
\end{align}

Where I've defined the Bohr magneton $\mu_B$.

\begin{align}
\mu_B = -\gamma_c \hbar = +\frac{|e|\hbar}{2m_e}
\end{align}

\begin{align}
H = g_F \mu_B\bv{B}\cdot\bv{f}
\end{align}

We then see (re-expressing the Hamiltonian in terms of frequency rather than energy by dividing by $\hbar$)

\begin{align}
\frac{H}{\hbar} =& -g_F \gamma_C \bv{B}\cdot \bv{f}\\
=& -g_F \gamma_C \bv{B}\cdot \frac{\bv{F}}{\hbar}\\
=& g_F \frac{\mu_B}{\hbar}\bv{B}\cdot\bv{f}
\end{align}

This motivates the definition of the Larmor frequency

\begin{align}
\omega_L = \gamma_{gy}|\bv{B}| = g_F \gamma_C |\bv{B}| = g_F \frac{\mu_B}{\hbar}|\bv{B}|
\end{align}

and magnetic field direction $\bv{\hat{n}}$ with $\bv{B} = |\bv{B}|\bv{\hat{n}}$ so that

\begin{align}
\frac{H}{\hbar} = -\omega_L \bv{\hat{n}}\cdot \bv{f}
\end{align}

All of this aside, going forward we will take our Hamiltonian

\begin{align}
H = -\gamma_{gy} \bv{B} \cdot \bv{F}
\end{align}

\section{Schrodinger Picture}

As a reminder, in the Schrodinger picture approach the Born rule expression

\begin{align}
\Braket{A}_t = \bra{\psi_0}T^{\dag}AT\ket{\psi_0}
\end{align}

is calculated by determining 

\begin{align}
\ket{\psi^{(S)}(t)} = T\ket{\psi_0}
\end{align}

And this can be done by solving the differential Schrodinger equation

\begin{align}
\frac{d\ket{\psi^{(S)}(t)}}{dt} = -\frac{i}{\hbar}H^{(S)}\ket{\psi^{(S)}(t)}
\end{align}

Where $H^{(S)} = i\frac{dT}{dt} T^{\dag}$ is the Hamiltonian which is the Hermitian Schrodinger generator of Unitary time evolution $T$.

In the unitary time-evolution write up I introduce the vector of kets

\begin{align}
\bv{q} = \begin{bmatrix}\ket{q_1}, \ldots,  \ket{q_N}\end{bmatrix}
\end{align}

Where the ${\ket{q_i}}$ from an orthonormal `computation' basis.
Orthonormality of the $\{\ket{q_i}\}$ is expressed by

\begin{align}
\bv{q}^{\dag}\bv{q} = 
\begin{bmatrix}
\braket{q_1|q_1} && \cdots && \braket{q_1|q_N}\\
\vdots && \ddots && \vdots \\
\braket{q_N|q_1} && \cdots && \braket{q_N|q_N}
\end{bmatrix} = \bv{1}
\end{align}

Where $\bv{1}$ is an $N\times N$ diagonal matrix.
Completeness of the $\{\ket{q_i}\}$ is expressed by

\begin{align}
\bv{q}\bv{q}^{\dag} = \sum_i \ket{q_i}\bra{q_i} = \bv{1}
\end{align}

Where here $\bv{1}$ is the identity operator on the Hilbert space.

Given this new notation we can introduce a pure-scalar vector

\begin{align}
\bv{c} =& \bv{q}^{\dag}\ket{\psi}\\
\begin{bmatrix}
c_1\\ \vdots\\ c_n
\end{bmatrix} =& 
\begin{bmatrix}
\braket{q_1|\psi}\\ \vdots \\ \braket{q_N|\psi} 
\end{bmatrix}
\end{align}

so that

\begin{align}
\ket{\psi} = \bv{q}\bv{c}
\end{align}

From this we see that knowledge of $\bv{c}$ and $\bv{q}$ constitutes full knowledge of the state $\ket{\psi}$.
Similarly, for any operator $O$ we can introduce a pure-scalar matrix by

\begin{align}
\bv{o} =& \bv{q}^{\dag}O\bv{q}\\
=& \begin{bmatrix}
\bra{q_1}O\ket{q_1} && \cdots && \bra{q_1}O\ket{q_N}\\
\vdots && \ddots && \vdots \\
\bra{q_N}O\ket{q_1} && \cdots && \bra{q_N}O\ket{q_N}
\end{bmatrix}
\end{align}

so that

\begin{align}
O = \bv{q}\bv{o}\bv{q}^{\dag}
\end{align}

We can then re-write the Schrodinger differential equation.

\begin{align}
\frac{d\ket{\psi^{(S)}}(t)}{dt} =& -\frac{i}{\hbar}H^{(S)}\ket{\psi^{(S)}(t)}\\
\bv{q}\frac{d\bv{c}(t)}{dt} =& -\frac{i}{\hbar}\bv{q}\bv{h}\bv{q}^{\dag}\bv{q}\bv{c}(t)
\end{align}

From which we can see (by acting $q^{\dag}$ on the left) that

\begin{align}
\dot{\bv{c}}(t) = -\frac{i}{\hbar}\bv{h}\bv{c}(t)
\end{align}

This differential equation is solved by

\begin{align}
c(t) = e^{-i\frac{\bv{h}}{\hbar}t}\bv{c}(0)
\end{align}

Note that the above manipulations rely on $\bv{h}$ being independent of time.
In general the matrix exponential $e^{-i\frac{\bv{h}}{\hbar}t}$ can be calculated directly from the definition of the matrix exponentially, however, this problem is typically approached by diagonalizing the Hamiltonian as follows

\begin{align}
\bv{h} = \bv{P}\bv{D}\bv{P}^{\dag}
\end{align}

Here $\bv{D}$ is a diagonal matrix consisting of the eigenvalues of $\bv{h}$ and $\bv{P}$ is a unitary matrix with columns of the corresponding eigenvectors of $\bv{h}$.
$\bv{P}$ is unitary because $\bv{h}$ is Hermitian.

We can then write

\begin{align}
\bv{c}(t) = \bv{P}e^{-i \frac{\bv{D}}{\bv{h}}t}\bv{P}^{\dag}\bv{c}(0)
\end{align}

This can be re-written as

\begin{align}
\bv{P}^{\dag}\dot{\bv{c}}(t) =& e^{-i\frac{\bv{D}}{\hbar}t}\bv{P}^{\dag}\bv{c}(0)\\
\tilde{\bv{c}}(t) =& e^{-i\bv{\omega} t}\tilde{\bv{c}}(0)
\end{align}

Here $\tilde{\bv{c}}(t) = \bv{P}^{\dag}\bv{c}(t)$ is transformed representation of the state, now represented in the basis corresponding to the eigen-states of the Hamiltonian as opposed to the original computation basis.
$\bv{\omega} = \frac{\bv{D}}{\hbar}$ is a diagonal matrix of the Bohr frequencies for the system.

This is a quite compact solution to the problem which demonstrates that any problem can be quickly solved in the Schrodinger picture if a computation basis is known (that is if we can pick out a particular basis which we can use to decompose both the initial state the Hamiltonian) and the Hamiltonian can be diagonalized.

To that end I will demonstrate the solution to the above magnetic field problem by diagonalizing the Hamiltonian.
I will consider a spin-$\frac{1}{2}$ particle.
The Hamiltonian is

\begin{align}
H^{(S)} = -\gamma_{gy}\bv{B}\cdot\bv{F}^{(S)}
\end{align}

This can be written as

\begin{align}
H^{(S)} = -\gamma_{gy}\left(B_x F_x^{(S)} + B_y F_y^{(S)} + B_z F_z^{(S)}\right)
\end{align}

We can further note that

\begin{align}
\bv{F} = \hbar \bv{f} = \frac{\hbar}{2}\bv{\sigma}
\end{align}

Where $\bv{\sigma}$ are the (possibly high spin) pauli matrices.

For a spin half particle there are two basis states, $\ket{-\frac{1}{2}} = \ket{g}$ and $\ket{+\frac{1}{2}} = \ket{e}$.
We have

\begin{align}
\sigma_x =& \ket{g}\bra{e} + \ket{e}\bra{g} \rightarrow \begin{bmatrix}
0 & 1\\ 1 & 0
\end{bmatrix}\nonumber\\
\sigma_y =& i(\ket{g}\bra{e} - \ket{e}\bra{g} \rightarrow \begin{bmatrix}
0 & -i\\ i & 0
\end{bmatrix}\nonumber\\
\sigma_z =& \ket{g}\bra{g} - \ket{e}\bra{e} \rightarrow \begin{bmatrix}
1 & 0\\ 0 & -1
\end{bmatrix}
\end{align}

From this we can see that in matrix form we have

\begin{align}
\bv{h} = \bv{q}^{\dag}H\bv{q} = -\frac{\gamma_{gy}\hbar}{2}\begin{bmatrix}
B_z && B_x-iB_y\\
B_x+iB_y && -B_z
\end{bmatrix}
\end{align}

In my $2\times2$ matrix diagonalization write up I demonstrate how such a matrix can be diagonalized.
Consider an arbitrary Hermitian $2\times 2$ matrix

\begin{align}
\bv{h} = \begin{bmatrix}
a && c\\
c^* && b
\end{bmatrix}
\end{align}

with $a$ and $b$ real numbers and complex $c = |c|e^{i\phi_c}$.
The result is that the eigenvalues and eigenvectors are given by

\begin{align}
D_+ &= \frac{a+b}{2} + \frac{1}{2}\sqrt{(b-a)^2 + 4|c|^2} \nonumber\\
D_- &= \frac{a+b}{2} - \frac{1}{2}\sqrt{(b-a)^2 + 4|c|^2}
\end{align}

\begin{align}
\bv{v}_+ &= 
\frac{1}{\sqrt{2}}\begin{bmatrix}
\sqrt{1-\frac{b-a}{\sqrt{(b-a)^2+4|c|^2}}}\\e^{-i\phi_c}\sqrt{1+\frac{b-a}{\sqrt{(b-a)^2+4|c|^2}}}
\end{bmatrix} \nonumber\\
\bv{v}_- &= 
\frac{1}{\sqrt{2}}\begin{bmatrix}
-\sqrt{1+\frac{b-a}{\sqrt{(b-a)^2+4|c|^2}}}\\e^{-i\phi_c}\sqrt{1-\frac{b-a}{\sqrt{(b-a)^2+4|c|^2}}}
\end{bmatrix}
\end{align}

We consider diagonalizing the matrix 

\begin{align}
\bv{b} = \begin{bmatrix}
B_z && B_x-iB_y\\
B_x+iB_y && -B_z
\end{bmatrix}
\end{align}

noting that $\bv{h} = -\frac{\gamma_{gy}\hbar}{2} \bv{b}$.

\begin{align}
a =& B_z \nonumber\\
b =& -B_z \nonumber\\
c =& B_x-iB_y \nonumber\\
|c| =& \sqrt{B_x^2 + B_y^2} \nonumber\\
\phi_c =& \arctan\left(-\frac{B_y}{B_x}\right)
\end{align}

and we get

\begin{align}
d_{\pm} = \pm \sqrt{B_x^2 + B_y^2+B_z^2} = \pm|B|
\end{align}

\begin{align}
\bv{v}_+ =& \frac{1}{\sqrt{2}}\begin{bmatrix}
\sqrt{1+\frac{B_z}{|B|}}\\
e^{-i\phi_c}\sqrt{1-\frac{B_z}{|B|}}
\end{bmatrix}\nonumber\\
\bv{v}_- =& \frac{1}{\sqrt{2}}\begin{bmatrix}
-\sqrt{1-\frac{B_z}{|B|}}\\
e^{-i\phi_c}\sqrt{1+\frac{B_z}{|B|}}
\end{bmatrix}\nonumber\\
\end{align}

We then have

\begin{align}
\bv{D} =& -\frac{\gamma_{gy}\hbar}{2}\begin{bmatrix}
d_+ && 0\\
0 && d_-
\end{bmatrix} \nonumber\\
\bv{P} =& \begin{bmatrix}
\bv{v}_+ && \bv{v_-}
\end{bmatrix}
\end{align}

We can calculate the dynamical matrix using Mathematica or by hand and we find

\begin{align}
\bv{P}e^{-i\frac{\bv{D}}{\hbar}t}\bv{P}^{\dag} = \begin{bmatrix}
\cos\left(\frac{\omega_L}{2} t\right) + i\frac{B_z}{|B|}\sin\left(\frac{\omega_L}{2}t\right) && i\frac{B_{\perp}}{|B|}\sin\left(\frac{\omega_L}{2} t\right)\\
i\frac{B_{\perp}^*}{|B|}\sin\left(\frac{\omega_L}{2} t\right) && \cos\left(\frac{\omega_L}{2}t\right) -i\frac{B_z}{|B|}\sin\left(\frac{\omega_L}{2}t\right)
\end{bmatrix}
\end{align}

Here I've defined $B_{\perp} = B_x - i B_y$.
We then see that the problem is solved because we can calculate

\begin{align}
\bv{c}(t) = \begin{bmatrix}c_+(t)\\c_-(t)\end{bmatrix} = 
\bv{P}e^{-i\frac{\bv{D}}{\hbar} t}\bv{P}^{\dag}\begin{bmatrix}c_+(0)\\c_-(0)\end{bmatrix}
\end{align}

So we have

\begin{align}
c_+(t) = & \left(\cos\left(\frac{\omega_L}{2} t\right) + i\frac{B_z}{|B|}\sin\left(\frac{\omega_L}{2}t\right)\right)c_+(0) + \left(i\frac{B_{\perp}}{|B|}\sin\left(\frac{\omega_L}{2} t\right)\right)c_-(0) \nonumber\\
c_-(t) = & \left(\cos\left(\frac{\omega_L}{2} t\right) - i\frac{B_z}{|B|}\sin\left(\frac{\omega_L}{2}t\right)\right)c_-(0) + \left(i\frac{B_{\perp}^*}{|B|}\sin\left(\frac{\omega_L}{2} t\right)\right)c_+(0)
\end{align}

This solves the problem of a spin-$\frac{1}{2}$ particle precessing in a magnetic field in the Schrodinger picture.

\section{Heisenberg Picture}

We now solve the same problem in the Heisenberg picture.

As a reminder, in the Heisenberg picture approach the Born rule expression

\begin{align}
\braket{A}_t = \bra{\psi_0}T^{\dag}AT\ket{\psi_0}
\end{align}

is calculated by determining 

\begin{align}
A^{(H)}(t) = T^{\dag}AT
\end{align}

By solving the Heisenberg equations of motion arising from

\begin{align}
\frac{dO^{(H)}}{dt} = -\frac{i}{\hbar}\left[O^{(H)}, H^{(H)}\right]
\end{align}

Here $H^{(H)} = iT^{\dag}\frac{dT}{dt}$ is the Heisenberg generator of the unitary time evolution $T$.

We write the Hamiltonian as

\begin{align}
H^{(H)} = -\gamma_{gy} \bv{B}\cdot \bv{F}^{(H)} = -\gamma_{gy} B_j F_j^{(H)}
\end{align}

Where I've used Einstein summation convention.
All operators here are in the Heisenberg picture, I've dropped the superscript for brevity.


We recall that

\begin{align}
[F_i, F_j] = i \hbar \ep_{ijk} F_k
\end{align}

We can then work out the Heisenberg equation of motion by

\begin{align}
\dot{F}_i^{(H)} = -\frac{i}{\hbar} \left(-\gamma_{gy}\right) B_j [F_i^{(H)}, F_j^{(H)}] = -\gamma_{gy} B_j \ep_{ijk} F_k^{(H)}
\end{align}

This can be written in vector form as

\begin{align}
\dot{\bv{F}}^{(H)} = -\gamma_{gy} \bv{B}\times \bv{F}^{(H)}
\end{align}

Alternatively this can be written as a linear differential equation:

\begin{align}
\dot{\bv{F}}^{(H)} = 
\begin{bmatrix}
\dot{F}_x^{(H)} \\ \dot{F}_y^{(H)} \\ \dot{F}_z^{(H)}
\end{bmatrix}
= 
-\gamma_{gy}
\begin{bmatrix}
0 && -B_z && B_y\\
B_z && 0 && -B_x\\
-B_y && B_x && 0
\end{bmatrix}
\begin{bmatrix}
F_x^{(H)} \\ F_y^{(H)} \\ F_z^{(H)}
\end{bmatrix}
= -\gamma_{gy} \bv{N} \bv{F}^{(H)}
\end{align}

The solution to this differential equation is 

\begin{align}
\bv{F}^{(H)}(t) = e^{-i\bv{N} t} \bv{F}^{(H)}(0)
\end{align}

Born rule expressions can be calculated by

\begin{align}
\braket{\bv{F}}_t = \braket{\bv{F}^{(H)}} = e^{-i\bv{N} t} \braket{\bv{F}^{(H)}}_0
\end{align}

We can also write

\begin{align}
\frac{d}{dt}\braket{\bv{F}}_t =& -\gamma_{gy}\bv{B} \times \braket{\bv{F}}_t\\
\frac{d}{dt}\braket{\bv{F}^{(H)}(t)} =& -\gamma_{gy}\bv{B} \times \braket{\bv{F}^{(H)}(t)}
\end{align}

\section{Ket Interaction Picture}

We now move on to demonstrating the solution to quantum problems involving the ket and operator interaction pictures.

We consider the problem described by Hamiltonian

\begin{align}
H =& \hbar \omega_0 \ket{e}\bra{e} + \hbar\frac{\Omega}{2} e^{-i\omega t}\ket{e}\bra{g} + \hbar \frac{\Omega^*}{2}e^{+i\omega t}\ket{g}\bra{e}\\
=& \hbar \omega_0 \Pi_e + \hbar \frac{\Omega}{2}e^{-i\omega t}\sigma^+ + \hbar \frac{\Omega^*}{2}e^{+i\omega t}\sigma^-
\end{align}

This Hamiltonian arises, for example, in the case of a two level atom driven by a sinusoidally varying dipole field after making a rotating wave approximation. 
The goal of working in the various interaction pictures will be to transform the Hamiltonian into a new form which is time-independent. 
That is, we would like to get rid of the $e^{\pm i\omega t}$ factors.

Our intuition will be guided by the fact that we know such a two level system would evolve as $e^{\pm i \omega t}$ if there were no coupling $\Omega$ and the free Hamiltonian was $\hbar \omega \Pi_e$.

The Ket interaction picture arises from factoring the time evolution operator $T$ as 

\begin{align}
T = X\tilde{Y}
\end{align}

Where $X$ is a simple solved time-evolution and $\tilde{Y}$ is more complex, but hopefully simpler than $T$.
In this case $X$ will be related to the free Hamiltonian $H_0=

\end{document}