\documentclass[12pt]{article}

\usepackage{amssymb, amsmath, amsfonts}
\usepackage{mathtools}
\usepackage{braket}
\usepackage{bussproofs}
\usepackage{amsthm}

\makeatletter
\newtheoremstyle{break}% name
    {12pt}%         Space above, empty = `usual value'
    {12pt}%         Space below
    {\addtolength{\@totalleftmargin}{1.5em}
     \addtolength{\linewidth}{-3em}
     \parshape 1 1.5em \linewidth
     \itshape}% Body font
    {}%         Indent amount (empty = no indent, \parindent = para indent)
    {\bfseries}% Thm head font
    {}%        Punctuation after thm head
    {\newline}% Space after thm head: \newline = linebreak
    {}%         Thm head spec
\theoremstyle{break}
\newtheorem{definition}{Definition}[section]
\theoremstyle{break}
\newtheorem{theorem}{Theorem}[section]
\theoremstyle{break}
\newtheorem{corollary}[theorem]{Corollary}
\theoremstyle{break}
\newtheorem{lemma}[theorem]{Lemma}
\theoremstyle{break}
\newtheorem{informal definition}[definition]{Informal Definition}


\newcommand{\bv}[1]{\boldsymbol{#1}}
\newcommand{\mc}[1]{\mathcal{#1}}
\newcommand{\bc}[1]{\bv{\mc{#1}}}
\newcommand{\qq}[1]{``#1''}

\begin{document}
\title{Math from Zermelo-Fraenkel Axioms}
\author{Justin Gerber}
\date{\today}
\maketitle

\section{Introduction}
The goal of this document is to do some basic manipulations starting from the Zermelo-Fraenkel set theory.
We do not take equality to be a logical symbol but will instead define it as a defined 2-ary predicate.

\section{Preliminaries}

\begin{definition}[``at most one`` existence]
Let $\phi$ be a formula with $x\in FV(\phi)$ and with $y\not\equiv y' \not \equiv x$ with $y$ and $y'$ substitutable for $x$ in $\phi$.

We let $\exists^{\leqslant 1} x\phi$ abbreviate

$$
\forall x\forall y \forall y' ((\phi[y/x]\land \phi[y'/x]) \implies y'=y)
$$
Informally this expression says that if $\phi$ holds for $y'$ and $y$ then $y'$ must equal $y$.
But note that this expression doesn't require $\phi$ to hold for any $x$.
That is, this expression is not inconsistent with
$$
\lnot\exists x \phi
$$
\end{definition}

If we are working in a language with equality then we may define unique existence.
\begin{definition}[Unique Existence]
Let $\phi$ be a formula with $x\in FV(\phi)$ and with $y\not \equiv x$ and $y$ substitutable for $x$ in $\phi$.
We let $\exists ! x \phi$ abbreviate
$$
\exists x \phi \land \exists^{\leqslant 1} x \phi
$$
\end{definition}

\begin{definition}[Extension by Predicate Definition]
Suppose $\phi$ is a formula with $FV(\phi) \equiv \{x_1,\ldots, x_n\}$.
We may extend a theory by adding a new $n$-ary predicate $P$ and the definitional axiom
$$
\forall x_1 \ldots \forall x_n (Px_1\ldots x_n \iff \phi)
$$
Such an extension is provably conservative
\end{definition}

\begin{definition}[Extension by Function Definition]
Suppose $\phi$ is a formula with $FV(\phi)\equiv \{x_1,\ldots, x_n, y\}$.
Suppose in a given theory with equality we have proven
$$
\forall x_1 \ldots \forall x_n \exists! y \phi
$$
Then we may extend the theory by adding the $n$-ary function $f$ and the definitional axiom
$$
\forall x_1 \ldots \forall x_n \phi[fx_1\ldots x_n/y]
$$
Such an extension is provably conservative
\end{definition}

ZF set theory takes place in the context of a language of first order logic and deduction system.
The only non-logical symbol included in our theory is the set membership 2-ary predicate $\in$.

\begin{definition}[Equality]
Consider
$$
\phi_= \equiv \forall z(z\in x \iff z \in y) \land \forall z (x\in z \iff y\in z)
$$
We see that $FV(\phi_=) \equiv \{x, y\}$.
In plain words, $\phi_=$ says sets $x$ and $y$ contain the same sets and sets $x$ and $y$ are contained in the same sets.
We perform an extension by definition using $\phi_=$ to introduce $=$ as a new 2-ary predicate into $\in$-theory.
We introduce the following equality axiom.
$$
\Gamma_= \equiv \forall x \forall y (x=y \iff \phi_=)
$$
In other words, for all sets $x$ and $y$, we say that $x=y$ if $x$ and $y$ contain the same sets and are contained in the same sets.
\end{definition}

\begin{definition}[Empty Set Predicate]
Let
$$
\phi_E \equiv \forall x (x\not\in y)
$$
We see $FV(\phi_E) = \{y\}$.
In plain words $\phi_E$ says that no set is contained in $y$.
We define the `$y$ is an empty set' predicate by
$$
\Gamma_E \equiv \forall y (Ey \iff \phi_E)
$$
\end{definition}

\begin{definition}[Successor Predicate]
Let
$$
\phi_S \equiv \forall v (v\in y \iff (v = x \lor v\in x))
$$
We see $FV(\phi_S) = \{x, y\}$.
In plain words, $\phi_S$ says that the elements of $y$ include $x$ and all elements of $x$.
We define the `$y$ is the successor of $x$' predicate by
$$
\Gamma_S \equiv \forall x \forall y (Syx \iff \phi_S)
$$
\end{definition}

\begin{definition}[Is Inductive Predicate]
Let
$$
\phi_I \equiv \exists y (y\in x \land Ey) \land \forall v(v\in x \implies \exists w (w\in x \land Swv))
$$
We see $FV(\phi_I) = \{x\}$.
In plain words $\phi_I$ says that there is an empty set in $x$ and for every set contained in $x$ the successor of that set is also in $x$.
We define the `$x$ is inductive' predicate by
$$
\forall x (Ix \iff \phi_I)
$$
\end{definition}

\section{Zermelo-Fraenkel Axioms}

We take the ZF axioms from a Stack Exchange post: https://math.stackexchange.com/questions/916072/what-axioms-does-zf-have-exactly

Having extended $\in$-theory by including the $=$ predicate, we will use the $=$ predicate in our ZF axioms.

\begin{definition}[Axiom of Extensionality]
$$
\forall x \forall y(\forall z (z\in x \iff z \in y) \implies \forall z(x\in z \iff y \in z))
$$
This axiom says that if sets $x$ and $y$ contain the same sets then $x$ and $y$ are contained in the same sets.
\end{definition}

\begin{definition}[Axiom of Union]
$$
\forall x \exists y \forall z (z\in y \iff \exists w (w\in x \land z\in w))
$$
This axiom says that for any set $x$ there exists a set $y$ whose elements are the elements of the elements of $x$.
\end{definition}

\begin{definition}[Axiom of Foundation]
$$
\forall x (\lnot Ex \implies \exists y(y\in x \land \forall z(z\in x \implies z \not \in y)))
$$
This axiom says that if $x$ is non-empty then it contains at least one element $y$ such that none of the elements of $x$ are contained in $y$.
\end{definition}

\begin{definition}[Axiom of Power Set]
$$
\forall x \exists y \forall z (z\in y \iff \forall w (w \in z \implies w \in x))
$$
This axiom says that for any set $x$ there exists a set whose elements are the subsets of $x$.
\end{definition}

\begin{definition}[Axiom Schema of Replacement]
Suppose $\phi$ is a formula with $FV(\phi) \subset \{x, y, A, w_1, \ldots, w_n\}$ with $y\in FV(\phi)$ and $B\not \in FV(\phi)$.
Then we include the axiom
$$
\forall w_1\ldots\forall w_n \forall A \Big( \forall x(x\in A \implies \exists!y \phi)  \implies \exists B \forall y(y\in B \iff \exists x (x\in A \land \phi))\Big)
$$
This axiom schemata says that if $\phi$ is a functional logical relation between $x$ and $y$ (when $A, w_1, \ldots, w_n$ are held constant) then there is a set which consists exactly of the sets that are related to elements $x\in A$.
\end{definition}

\begin{definition}[Axiom Schema of Replacement]
Suppose $\phi$ is a formula with $FV(\phi) \subset \{x, y, A, w_1, \ldots, w_n\}$ with $y\in FV(\phi)$ and $B\not \in FV(\phi)$.
Then we include the axiom
\begin{align}
\forall w_1\ldots\forall w_n \forall A \Big(& \forall x(x\in A \implies \forall y \forall y' (\phi \land \phi[y/y']\implies y'=y)) \notag\\
&  \implies \exists B \forall y(y\in B \iff \exists x (x\in A \land \phi))\Big) \notag
\end{align}

This axiom schemata says that if $\phi$ is a functional logical relation between $x$ and $y$ (when $A, w_1, \ldots, w_n$ are held constant) then there is a set which consists exactly of the sets that are related to elements $x\in A$.
\end{definition}

\begin{definition}[Axiom of Infinity]
$$
\exists x Ix
$$

This axiom says that there exists an inductive set.

%\begin{equation}
%\exists x \Big(\exists y(y \in x \land Ey) \land \forall u (u\in x \implies \exists v(v\in x \land Svu)) \Big)
%\end{equation}
%
%
%\begin{equation}
%\exists x \Big(\exists y (y\in x \land \forall z (z\not \in y)) \land \forall u (u\in x  \implies \exists v (v\in x \land \forall w (w\in v \iff (w\in u \lor w = u))))\Big)
%\end{equation}
\end{definition}

\section{Basic Theorems and Definitions}

\begin{theorem}[Extensionality and Equality]
Here we prove that the Axiom of Extensionality and the definition of equality imply
$$
\forall x \forall y \left(\forall z(z\in x \iff z \in y) \implies x=y\right)
$$

Let
\begin{align}
A &\equiv \forall z (z\in x \iff z \in y) \notag\\
B &\equiv \forall z (x\in z \iff y \in z) \notag
\end{align}

Then
\begin{align}
\Gamma_= &\equiv \forall x \forall y (x=y \iff (A\land B)) \notag\\
\Gamma_{\text{ext}} &\equiv \forall x \forall y (A \implies B)
\end{align}

With some applications of $\forall E$ and $\forall I$ and some basic logic we can see
$$
\forall x \forall y (x=y \iff A)
$$
\end{theorem}

\begin{definition}[Subset]
Let
$$
\phi_{\subset} \equiv \forall z \left(z\in x \implies z \in y\right)
$$

We see that $FV(\phi_{\subset}) \equiv \{x, y\}$.
In plain words $\phi_\subset$ says all elements of $x$ are also elements of $y$.
We introduce the following subset axiom
$$
\Gamma_{\subset} \equiv \forall x \forall y (x\subset y \iff \phi_{\subset})
$$
\end{definition}

\begin{theorem}[Equality and Subsets]
From the above characterization of the Axiom of Extensionality in terms of equality and the definition of subset we can see

$$
\forall x \forall y \left((x\subset y \land y \subset x) \iff x = y \right)
$$
\end{theorem}

\begin{theorem}[Power Set and Subsets]
From the above characterization of the Axiom of Power Set and the definition of subset we can see

$$
\forall x \exists y \forall z(z\in y \iff z\subset x)
$$

Define

$$
\phi_{\mathcal{P}} \equiv \forall z (z\in y \iff z \subset x)
$$
We can see $FV(\phi_{\mathcal{P}}) = \{x, y\}$.
We can prove, using the Axiom of Power Set and the Axiom of Extensionality that

$$
\forall x \exists! y \phi_{\mathcal{P}}
$$

We introduce a new 1-ary function $\mathcal{P}$ and the definitional axiom

$$
\Gamma_{\mathcal{P}} \equiv \forall x \forall z(z\in \mathcal{P}x \iff z \subset x)
$$

We often write $\mathcal{P}$ as $\mathcal{P}(x)$ so that

$$
\Gamma_{\mathcal{P}} \equiv \forall x \forall z (z\in \mathcal{P}(x) \iff z \subset x)
$$

\end{theorem}

\begin{theorem}[Existence of an Empty Set]
We prove the existence of an empty set from the Axiom of Infinity, which directly asserts the existence of an empty set.

Recall the defining formula for an inductive set:
$$
\phi_I \equiv \exists y(y\in x \land Ey) \land \forall v (v \in x \implies \exists w(w\in x \land Swv))
$$

Let
\begin{align}
\pi &\equiv \exists y (y\in x \land Ey)\\
\psi &\equiv \forall v (v \in x \implies \exists w(w\in x \land Swv))
\end{align}
so that
$$
\phi_I \equiv \pi \land \psi
$$

\begin{center}
\begin{tabular}{ p{2cm} p{8cm} p{3cm} }
(1) & $\Gamma_{def} \vdash \forall x (Ix \iff \phi_I)$ & Prem \\
(2) & $\Gamma_{ZF} \vdash \exists x Ix$ & Prem\\
(3) & $\Gamma_{def} \vdash I\alpha_1 \iff \phi[\alpha_1/x]$ & $\forall$E, 1 \\
(4) & $I\alpha_1 \vdash I\alpha_1$ & A \\
(5) & $\Gamma_{def}, I\alpha_1 \vdash \phi[\alpha_1/x]$ & $\iff$E, 3, 4\\
(6) & $\Gamma_{def}, I\alpha_1 \vdash \pi[\alpha_1/x]\land \psi[\alpha_1/x]$ & W, 5\\
(7) & $\Gamma_{def}, I\alpha_1 \vdash \pi[\alpha_1/x]$ & $\land$E, 6\\
(8) & $\Gamma_{def}, I\alpha_1 \vdash \exists y(y\in \alpha_1 \land Ey)$ & W, 7\\
(9) & $\alpha_2\in\alpha_1 \land E\alpha_2\vdash \alpha_2\in\alpha_1 \land E\alpha_2$ & A\\
(10) & $\alpha_2\in\alpha_1\land E\alpha_2\vdash E\alpha_2$ & $\land E$ 9\\
(11) & $\alpha_2\in\alpha_1 \land E\alpha_2 \vdash \exists y Ey$ & $\exists I$, 10\\
(12) & $\vdash (\alpha_2\in \alpha_1 \land E\alpha_2)\implies \exists y Ey$ & $\implies$I, 11\\
(13) & $\Gamma_{def}, I\alpha_1 \vdash \exists y Ey$ & $\exists$E, 8, 12\\
(14) & $\Gamma_{def} \vdash I\alpha_1 \implies \exists y Ey$ & $\implies$I, 13\\
(15) & $\Gamma_{ZF}, \Gamma_{def} \vdash \exists y Ey$ & $\exists$E, 2, 14
\end{tabular}
\end{center}
\end{theorem}

\begin{theorem}[Uniqueness of the Empty Set]
We prove that $Ey \land Ex \implies x = y$.
Because $x$ and $y$ are empty they contain no elements so we can prove $x\subset y \land y \subset x$ so, by the theorem of equality and subsets, we have $x=y$.
We let $\emptyset$ denote the empty set.
\end{theorem}

\begin{theorem}[$\emptyset \subset x$]
\label{thm:emptysetsubsetx}
We have

$$
\forall x \forall z (z\in \emptyset \implies  z \in x)
$$

This is always vacuously true because the antecedent $z\in \emptyset$ is always false by the definitional axiom of the empty set.
Therefore

$$
\forall x (\phi \subset x)
$$

In particular this means
$$
\forall x (\emptyset \in \mathcal{P}(x))
$$
\end{theorem}

\begin{theorem}[$x\subset x$]
\label{thm:xsubsetx}
We have
$$
\forall x \forall z(z \in x \implies z\in x)
$$

which follows from simple inferences.
In particular this means
$$
\forall x (x \subset x)
$$
and
$$
\forall x(x\in \mathcal{P}(x))
$$
\end{theorem}

\begin{definition}[Explicit Set Builder Notation]
Suppose $t_1, \ldots, t_m$ are terms and suppose that $x_1, \ldots, x_n$ appear free among these terms.
We define

$$
\phi_{\{t_1,\ldots, t_m\}} \equiv \forall z(z\in y \iff (z=t_1 \lor \ldots \lor z=t_m))
$$
If we can prove that

$$
\forall x_1\ldots \forall x_n \exists! y \phi_{\{t_1, \ldots, t_m\}}
$$

Then we may introduce a new $n-ary$ symbol $S_{\{t_1, \ldots, t_m\}}$ with the definitional axiom

$$
\Gamma_{\{t_1,\ldots, t_m\}} \equiv \forall x_1 \ldots \forall x_n \forall z (z\in S_{\{t_1,\ldots, t_m\}} \iff (z=t_1 \lor \ldots \lor z=t_m))
$$

We typically write $S_{\{t_1,\ldots,t_m\}}$ as $\{t_1,\ldots, t_m\}$ so that

$$
\Gamma_{\{t_1,\ldots, t_m\}} \equiv \forall x_1 \ldots \forall x_n \forall z(z\in \{t_1,\ldots, t_m\} \iff(z=t_1\lor \ldots \lor z=t_m))
$$
\end{definition}

\begin{theorem}[$\mathcal{P}(\emptyset) = \{\emptyset\}$]
By either Theorem \ref{thm:emptysetsubsetx} or \ref{thm:xsubsetx} we can see that $\emptyset \in \mathcal{P}(\emptyset)$.

Suppose $x \not = \emptyset$.
We know that $\emptyset \subset x$ so we must have $x\not\subset \emptyset$ which means $x\not \in \mathcal{P}(\emptyset)$.
Therefore we have
$$
\forall z (z\in \mathcal{P}(\emptyset) \iff z = \emptyset)
$$

From which we can conclude $\mathcal{P}(\emptyset) = \{\emptyset\}$.
\end{theorem}

\begin{theorem}[$\{z\}$ exists]
Here will prove that

$$
\forall z \exists y \forall x (x\in y \iff x = z)
$$
From which we can define the set $\{z\}$ so long as $z$ is a set.

Let

$$
\phi \equiv (x = \emptyset \land y = z)
$$
and consider the Axiom of Replacement with $A = P(\emptyset) = \{\emptyset\}$.

$$
\forall z(\forall x (x\in \{\emptyset\} \implies \exists!y \phi) \implies \exists B \forall y (y\in B \iff \exists x(x\in \{\emptyset\} \land\phi)))
$$
The antecedent is true for all $z$ so we get
$$
\forall z \exists B \forall y (y\in B \iff \exists x(x\in \{\emptyset\} \land \phi))
$$
The formula $\exists x (x\in \{\emptyset\} \land \phi)$ can be reduced to just $y=z$ so that
$$
\forall z \exists B \forall y(y\in B \iff y=z)
$$
We apply a change of variables
$$
\forall z \exists y \forall x (x\in y \iff x=z)
$$
which is what we were trying to prove.
The axiom of extensionality tells us that
$$
\forall z \exists!y \forall x(x\in y \iff x=z)
$$
We define the 1-ary function $\{z\}$ with the definitional formula
$$
\phi_{\{z\}}\ \equiv \forall z \forall x(x\in \{z\} \iff x=z)
$$

\end{theorem}

\begin{definition}[Arbitrary Union]
Let

$$
\phi_{\cup} \equiv \forall z(z\in y \iff \exists w(w\in x \land z \in w))
$$

We see $FV(\phi_{\cup}) \equiv \{x, y\}$.
In plain words $\phi_{\cup}$ says that $z$ is in $y$ if there is a $w$ in $x$ with $z$ in $w$.
The Axiom of Union tells us that such a set $y$ exists for each set $x$.
Furthermore, the axiom of extensionality tells us that this set $y$ is unique.
Therefore
$$
\forall x \exists! y \phi_{\cup}
$$

We introduce the 1-ary function symbol $\cup$ and the defining axiom

$$
\Gamma_{\cup} \equiv \forall x \forall z (z\in \cup x \iff \exists w(w\in x \land z \in w))
$$

\end{definition}

\begin{theorem}[Schema of Substitution]
The Schema substitution says that for any formula $\psi$ with $FV(\psi) \subset \{x, A, w_1, \ldots, w_n\}$ with $B\not \in FV(\psi)$ that

$$
\forall w_1\ldots \forall w_n \forall A \exists B \forall x (x \in B \iff (x\in A \land \phi))
$$

We will prove this using some previous results.
First let
$$
\psi \equiv
$$



\end{theorem}



\section{old}


\begin{theorem}[Axiom Schema of Specification]
From the existence of the empty set and the Axiom Schema of Replacement we can prove the Axiom Schema of Specification.

The Axiom Schema of Replacement states the following.
Suppose $\phi$ is a formula with $FV(\phi) \subset \{x, y, A, w_1, \ldots, w_n\}$ with $y\in FV(\phi)$ and $B\not \in FV(\phi)$.
Then
$$
\forall w_1\ldots \forall w_n \forall A\left(\forall x (x\in  A \implies \exists! y\phi) \implies \exists B \forall y(y\in B \iff \exists x(x\in A \land \phi))\right)
$$

The Axiom Schema of Specification states the following.
Suppose $\psi$ is a formula with $FV(\psi) \subset (x, A, w_1, \ldots, w_n)$ and $B\not \in FV(\psi)$.
$$
\forall w_1\ldots w_n \forall A \exists B \forall x ( x\in B \iff (x\in A \land \psi))
$$

For each $\psi$ we prove the corresponding axioms as follows.
Let $\phi \equiv \psi \land x=y$.
We can see that $FV(\phi) \subset \{x, y, A, w_1, \ldots, w_n\}$ because $FV(\psi) \subset \{x, A, w_1, \ldots, w_n\}$.
We can also see that $B\not \in FV(\phi)$.
Because of the $x=y$ subformula, we can see that (for fixed $w_1,\ldots, w_n$) that if $\psi(x)$ holds then $\forall x\exists! y \phi(x, y)$ holds.
This implies, by the Axiom Schema of Replacement, that
$$
\exists B \forall y(y\in B \iff \exists x (x \in A \land \psi \land x=y))
$$

which is equivalent to
$$
\exists B \forall x (x\in B \iff (x\in A \land \psi))
$$
where $B$ is exactly the set required by the axiom Schema of Specification.
\end{theorem}


\begin{definition}[Arbitrary Intersection]
Let
$$
\phi_{\cap} \equiv \forall z(z\in y \iff (z\in \cup x \land (\forall w (w\in x \implies z\in w))))
$$

We see $FV(\phi_{\cap}) \equiv \{x, y\}$.
In plain words, $\phi_{\cap}$ says that a set $z$ is an element of $y$ if that set is in the union of $x$ (meaning that each element of $z$ is an element of an element of $x$) and that $z$ is an element of element of $x$.

With the identification $A \to \cup x$, $B\to y$, $x\to z$, we see that for each set $x$ there exists a set $y$ satisfying $\phi_{\cap}$ from the Axiom Schema of Specification.
The axiom of extensionality tells us that this set is unique.
We therefore have
$$
\forall x \exists!y \phi_{\cap}
$$

We introduce the function symbol $\cap$ and introduce the defining axiom

$$
\Gamma_{\cap} \equiv \forall x \forall z(z\in \cap x \iff (z\in \cup x \land(\forall w(w\in x \implies z \in w))))
$$

\end{definition}

\begin{definition}[Power Set]
The axiom of the power set says
$$
\forall x \exists y \forall z(z\in y \iff \forall w (w\in z \implies w\in x)
$$
We can rewrite this as
$$
\forall x \exists y \forall z(z\in y \iff z \subset x)
$$

Let
$$
\phi_{\mathcal{P}} \equiv \forall z (z\in y \iff z \subset x)
$$
We see that $FV(\phi_{\mathcal{P}}) \equiv \{x, y\}$.
We can also see, by the axiom of extensionality, that
$$
\forall x \exists! y \phi_{\mathcal{P}}
$$
We introduce the function symbol $\mathcal{P}$ and the axiom
$$
\forall x \forall z (z \in \mathcal{P}x \iff z \subset x)
$$
$\mathcal{P}x$ is the powerset of $x$
\end{definition}

\begin{theorem}[Pairing]
Suppose $x$ is set $y$ is a set.
We will prove that
$$
\exists z \forall w(w\in z \iff (z = x \lor z = y))
$$
We will prove this using the Axiom Schema of Replacement.
We will construct a functional logical relation whose range is the set we're interested in, but first we need a domain set.
Since the range has two sets in it we'll need a domain with two sets.

\end{theorem}

\begin{definition}[Set Builder Notation]
Suppose $x_1,\ldots, x_n$ are terms with free variables $\{w_1, \ldots, w_m\}$
Let
$$
\phi_{\{x_1, \ldots, x_n\}} \equiv x \in y \iff (x = x_1 \lor \ldots \lor x = x_n)
$$


\end{definition}


\section{The Natural Numbers}

We now define the natural numbers and prove many properties about them.




\section{old}
\begin{definition}[Explicit Set Notation]
Suppose $t_1,\ldots, t_n$ are terms with $FV(t_1)\cup\ldots \cup FV(t_n) = \{v_1, \ldots, v_n\}$.
Let the formula $\phi$ be
$$
\phi \equiv \forall x (x\in y \iff (x=t_1 \lor \ldots \lor x=t_n))
$$
We see that $FV(\phi) = \{y, v_1,\ldots, v_n\}$.

\begin{center}
\begin{tabular}{ p{2cm} p{8cm} p{3cm} }
($N_{t_1}$) & $\Gamma_1 \vdash \forall v_{1,1} \ldots \forall v_{1,n_1}\exists z (z=t_1)$ & \\
\vdots & \vdots & \vdots \\
($N_{t_n}$) & $\Gamma_n \vdash \forall v_{n,1} \ldots \forall v_{n,n_n}\exists z (z=t_n)$ & \\
($k$) & $\Gamma_I \vdash \gamma_I$ & Prem \\
($k+1$) & $\Gamma_I \vdash \exists x \phi$ & W \\
($k+2$) & $\phi[\alpha_1/x] \vdash \phi[\alpha_1/x]$ & A \\
($k+3$) & $\phi[\alpha_1/x] \vdash \pi[\alpha_1/x]\land \psi[\alpha_1/x]$ & W \\
($k+4$) & $\phi[\alpha_1/x] \vdash \pi[\alpha_1/x]$ & $\land$E, $k+3$\\
($k+5$) & $\phi[\alpha_1/x] \vdash \exists y \theta[\alpha_1/x]$ & W \\
&$\theta[\alpha_1/x][\alpha_2/y] \vdash \theta[\alpha_1/x][\alpha_2/y]$ & \\
&$\theta[\alpha_1/x][\alpha_2/y] \vdash \alpha_2\in \alpha_1 \land \forall z(z\not\in \alpha_2)$ & \\
&$\theta[\alpha_1/x][\alpha_2/y] \vdash \forall z(z\not\in \alpha_2)$ & $\land$E \\
&$\theta[\alpha_1/x][\alpha_2/y] \vdash \exists y \forall z(z\not\in y)$ & $\land$E \\
& $\vdash \theta[\alpha_1/x][\alpha_2/y] \implies \exists y \forall z(z\not\in y)$ & $\land$E \\
& $\phi[\alpha_1/x] \vdash \exists y \forall z (z\not \in y)$ & \\
\end{tabular}
\end{center}
\end{definition}

Arbitrary set:



Successor: 1-ary function

$$
\phi \equiv v \in y \iff (v=x \lor (x\in v \land \forall z (z\in v \implies z=x)))
$$

\end{document}