\documentclass[12pt]{article}

\usepackage{amssymb, amsmath, amsfonts}
\usepackage{mathtools}
\usepackage{braket}
\usepackage{bussproofs}
\usepackage{amsthm}

\makeatletter
\newtheoremstyle{break}% name
    {12pt}%         Space above, empty = `usual value'
    {12pt}%         Space below
    {\addtolength{\@totalleftmargin}{1.5em}
     \addtolength{\linewidth}{-3em}
     \parshape 1 1.5em \linewidth
     \itshape}% Body font
    {}%         Indent amount (empty = no indent, \parindent = para indent)
    {\bfseries}% Thm head font
    {}%        Punctuation after thm head
    {\newline}% Space after thm head: \newline = linebreak
    {}%         Thm head spec
\theoremstyle{break}
\newtheorem{definition}{Definition}[section]
\theoremstyle{break}
\newtheorem{theorem}{Theorem}[section]
\theoremstyle{break}
\newtheorem{corollary}[theorem]{Corollary}
\theoremstyle{break}
\newtheorem{lemma}[theorem]{Lemma}
\theoremstyle{break}
\newtheorem{informal definition}[definition]{Informal Definition}


\newcommand{\bv}[1]{\boldsymbol{#1}}
\newcommand{\mc}[1]{\mathcal{#1}}
\newcommand{\bc}[1]{\bv{\mc{#1}}}
\newcommand{\qq}[1]{``#1''}

\begin{document}
\title{Structural Recursion and Trees}
\author{Justin Gerber}
\date{\today}
\maketitle

\section{Introduction}

\begin{definition}[Structurally Inductive]
Suppose $U$ is a set, $B\subset U$ and $\mathcal{F}$ is a family of functions where $f\in \mathcal{F}$ means $f:U^n \to U$ for some $n\in \mathbb{N}$.
If $C$ is a set that satisfies

\begin{itemize}
\item{$B\subset C$}
\item{For each $f\in \mathcal{F}$ with $f:U^n \to U$ then for each $\bv{c}\in C^n$ we have $f(\bv{c}) \in C$}
\end{itemize}

Then we say $C$ is $(U, B, \mathcal{F})$-inductive
\end{definition}

\begin{definition}[Structural Inductive Closure]
Let
$$
\mathcal{C}_{(U, B, \mathcal{F})} = \{C \in \mathcal{P}(U): C\text{ is }(U, B, \mathcal{F})\text{-inductive}\}
$$
The we define the $(U, B, \mathcal{F})$-inductive closure as
$$
\mathbb{N}_{(U, B, \mathcal{F})} = \bigcap \mathcal{C}
$$
\end{definition}

\begin{theorem}[Structural Induction]
Suppose $\phi$ is a formula that has $u$ as a free variable.
Let $C = \mathbb{N}_{(U, B, \mathcal{F})}$
If we have that
\begin{align*}
&\forall b \in B(\phi[b/u]) \\
&\forall f \in \mathcal{F}(\forall n \in \mathbb{N}(f:U^n \to U \implies \forall \bv{c} \in C^n(\forall i \in [n](\phi[\bv{c}_i/u])\implies \phi[f(\bv{c}/u)])))
\end{align*}

Then we may conclude
$$
\forall c \in C(\phi[c/u])
$$
\end{theorem}

We note that $\mathbb{N}=\mathbb{N}_{(\mathbb{N}, \{0\}, \{+_1\})}$ so that regular induction on the natural numbers is a special case of structural induction.

\begin{theorem}[Structural Recursion Theorem]
Let $C = \mathbb{N}_{(U, B, \mathcal{F})}$.
Suppose there is a set $S$.
Suppose also

\begin{itemize}
\item{There is a function $h_B:B \to S$.}
\item{There is a family of function $\tilde{\mathcal{F}}$ and a bijection $\mathcal{B}: \mathcal{F} \to \tilde{\mathcal{F}}$ and that if $f\in \mathcal{F}$ has $f:U^n \to U$ for $n\in \mathbb{N}$ then $\mathcal{B}(f): S^n \to S$. We let $\tilde{f} = \mathcal{B}(f)$.}
\item{For each $f\in \mathcal{F}$ with $f:U^n \to U$ the restriction $f|_{C^n}$ is injective.}
\item{For each $f\in \mathcal{F}$ with $f:U^n \to U$ we have $\text{Img}(f|_{C^n}) \cap B = \emptyset$.}
\item{For each $f, f' \in \mathcal{F}$ with $f:U^n \to U$ and $f':U^{n'} \to U$ we have $\text{Img}(f|_{C^n}) \cap \text{Img}(f|_{C^{n'}}) = \emptyset$.}
\end{itemize}

Under these conditions it can be proven that there exists a function $h:C\to S$ which satisfies

\begin{align*}
h|_B &= h_B\\
h(f(\bv{c})) &= \tilde{f}(h^{(n)}(\bv{c}))
\end{align*}

for all $f\in \mathcal{F}$ with $f:U^n \to U$ and for all $\bv{C} \in C^n$.
here $h^{(n)}(\bv{c}) = (h(\bv{c}_0), \ldots, h(\bv{c}_{n-1}))$.

\end{theorem}

\begin{definition}[Directed Graph]
A graph of size $n$ consists of a set $V$ of vertices which has cardinality $|V|=n$ and a set of directed edges $E \subset V \times V$.
Each element $e\in E$ is an ordered pair $\braket{v_1, v_2}$ for $v_1, v_2 \in V$.
We say $G=(V, E)$ is a directed graph with vertices $V$ and directed edges $E$.
\end{definition}

\begin{definition}[Parent/Child]
Suppose $G=(V, E)$ is a directed graph.
Suppose $v \in V$ and $(v', v)\in E$.
Then we say $v'$ is a parent of $v$ and $v$ is a child of $v'$.
\end{definition}

\begin{definition}[Rooted Tree]
Suppose $G=(V, E)$ is a directed graph.
Suppose that there is exactly one node $v_0 \in V$ which has no parent.
Suppose that every other node $v\in V$ with $v\not=v_0$ has exactly one parent.
Then we say $G=(V, E)$ is a directed rooted tree.
\end{definition}

\begin{definition}[Parse Tree]
Suppose $U$ is a set, $B\subset U$ and $\mathcal{F}$ is family of functions where $f\in \mathcal{F}$ means $f:U^n \to U$ for some $n\in \mathbb{N}$.

Suppose $G=(V, E)$ is a rooted tree.
Define a set
$$
\tilde{\mathcal{F}} = \mathcal{F} \cup B
$$
Suppose there are function $F_V:V \to \tilde{\mathcal{F}}$ and $F_E:E \to \mathbb{N}$.
Suppose these functions satisfy
\begin{itemize}
\item{If $v\in V$ has no children then $F_V(v) \in B$, otherwise $F_V(v) \in \mathcal{F}$.}
\item{Consider $v\in V$ which has a non-zero number of children so that $F_V(v) \in \mathcal{F}$. Suppose $F_V(v) = f$ with $f:U^n \to U$. Then consider the set $E^{(C)}_v$ which is the set of edges connecting $v$ to its children. This set must satisfy $|E^{(C)}_v| = n$ and $F_E(E^{(C)}_v) = [n] = \{0, \ldots, n-1\}$. That is $F_E$ restricted to $E^{(C)}_v$ is a bijection onto $[n]$.}
\end{itemize}

If these conditions are satisfied then we say the collection $(V, E, F_V, F_E)$ constitutes a $(U, B, \mathcal{F})$-parse tree.

We let $\mathbb{T}_{(U, B, \mathcal{F})}$ denote the set of all $(U, B, \mathcal{F})$ parse trees.
\end{definition}

\begin{definition}[Parse Tree Equivalence Relation]
We define an equivalence relation $\sim$ on $T = \mathbb{T}_{(U, B, \mathcal{F})}$.
Suppose $t, t' \in T$.
Suppose $t = (V, E, F_V, F_E)$ and $t' = (V', E', F'_{V'}, F'_{E'})$.
We say $t \sim t'$ if the following conditions hold

\begin{itemize}
\item{$V = V' = [n] = \{0, \ldots, n-1\}$ for some $n\in \mathbb{N}$.}
\item{There exists a bijection $\rho: V \to V'$.}
\item{For all $v_1, v_2 \in V$ we have that $(v_1, v_2) \in E$ if and only if $(\rho(v_1), \rho(v_2)) \in E'$}
\item{For all $v\in V$ we have that $F_V(v) = F'_{V'}(\rho(v))$.}
\item{For all $e\in E$ with $e = (v, v')$ we let $e' = (\rho(v), \rho(v'))$ and we have $F_E(e) = F'_{E'}(e')$.}
\end{itemize}`
\end{definition}

\begin{theorem}[Parse Tree Equivalence Relation]
Here we prove that $\sim$ is indeed an equivalence relations.

If we let $\rho = I_V$, the identity on $V$, we can see that $t\sim t$ for all $t\in T$.

Suppose $t \sim t'$.
We must show $t' \sim t$.
We have $t = (V, E, F_V, F_E)$ and $t' = (V', E', F'_{V'}, F'_{E'})$.
Because $t \sim t'$ we know there is a bijection $\rho$ satisfying the conditions for $\sim$.
Consider $\rho^{-1}$.
We see that $\rho^{-1}$ is a bijection $\rho^{-1}:V'\to V$
Consider $(v_1', v_2') \in E'$.
Let $v_1 = \rho^{-1}(v_1')$ and $v_2 = \rho^{-1}(v_2')$ so that $v_1, v_2 \in V$.
Because $t \sim t'$ we have that $(v_1, v_2) = (\rho^{-1}(v_1'), \rho^{-1}(v_2')) \in E$ if and only if $(\rho(\rho^{-1}(v_1')), \rho(\rho^{-1}(v_1')), \rho(\rho^{-1}(v_2'))) = (v_1', v_2') \in E'$.
Suppose $v' \in V'$.
We then know $\rho^{-1}(v) \in V$ which means $F'_{V'}(\rho(\rho^{-1}(v'))) = F'_{V'}(v') = F_V(\rho^{-1}(v))$.
Likewise if $e' \in E'$ with $e' = (v_1', v_2')$ then we know $e = (\rho^{-1}(v_1'), \rho^{-1}(v_2')) \in E$ and that $F'_{E'}((\rho(\rho^{-1}(v_1')), \rho(\rho^{-1}(v_2')))) = F'_{E'}((v_1' ,v_2')) = F_E((\rho^{-1}(v_1'), \rho^{-1}(v_2')))$.
All of this means $t' \sim t$.

Suppose $t \sim t'$ and $t' \sim t''$.
If $\rho$ is the bijection relating $t$ and $t'$ and $\tau$ is the relation relating $t'$ and $t''$, then we can show $t \sim t''$ by consider the bijection $\sigma = \tau \circ \rho$.

\end{theorem}

\begin{definition}[Parse Tree Equivalence Class]
We define $\bar{\mathbb{T}}_{(U, B, \mathbb{F})} = \mathbb{T}_{(U, B, \mathcal{F})} / \sim$ to be the equivalence class of parse trees.
Elements of $\bar{\mathbb{T}}_{(U, B, \mathcal{F})}$ are parse where the actual identities of the vertices $V$ do not matter.
All that is significant are the connectivity of the vertices and the labels of the vertices and edges.
That is, two parse trees are equivalent if vertices with the same labels are connected by edges with the same labels.

Note that for $\bar{\mathbb{T}}_{(U, B, \mathcal{F})}$ we require the constraint that $V = [n] = \{0, \ldots, n-1\}$.
If we allowed $V$ to be arbitrary then the equivalence class would indeed be a proper class and not a set.
\end{definition}

We would like to show there is a bijection between between $\mathbb{N}_{(U, B, \mathcal{F})}$ and $\bar{\mathbb{T}}_{(U, B, \mathcal{F})}$.

\begin{theorem}[Bijection between structural inductive closure and parse trees]
Let $C = \mathbb{N}_{(U, B, \mathcal{F})}$ and $T = \bar{\mathbb{T}}_{(U, B, \mathcal{F})}$.
First we define $h_B:B \to T$.
We define $h_B$ by
$$
h_B(b) = [(\{0\}, \{\emptyset\}, \{\braket{0, b}\}, \{\emptyset\})]
$$
That is $h_B$ maps $b\in B$ to the parse tree in $T$ which has a single vertex which is mapped to $b$ and has no edges.

We construct a family of functions $\tilde{\mathcal{F}}$ corresponding to $\mathcal{F}$.
For each $f\in \mathcal{F}$ with $f:U^n \to U$ we construct $\tilde{f}:T^n \to T$.
Consider $\bv{t} = (t_0, \ldots, t_{n-1})$ for $t_i\in T$ for $0\le i < n$.
We define $\tilde{f}(\bv{t})$ to be the parse tree defined as follows.
The new parse tree $\tilde{f}(\bv{t})$ will have $1 + \sum_{i=0^{n-1}} |t_i|$ vertices.
\end{theorem}



\end{document}

