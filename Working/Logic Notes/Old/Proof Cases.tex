\documentclass[12pt]{article}

\usepackage{amssymb, amsmath, amsfonts}
\usepackage{mathtools}
\usepackage{ND}
\usepackage{tcolorbox}
\usepackage{bbm}
\usepackage[utf8]{inputenc}
\usepackage{subfigure}%ngerman
%\usepackage[pdftex]{graphicx}
\usepackage{textcomp} 
\usepackage{color}
\usepackage[hidelinks]{hyperref}
\usepackage{anysize}
\usepackage{siunitx}
\usepackage{verbatim}
\usepackage{float}
\usepackage{braket}
\usepackage{xfrac}
\usepackage{array, booktabs} 
\usepackage{tabularx}

\newtheorem{definition}{Definition}

\newcommand{\ddt}[1]{\frac{d #1}{dt}}
\newcommand{\ppt}[1]{\frac{\partial #1}{\partial t}}
\newcommand{\ep}{\epsilon}
\newcommand{\sinc}{\text{sinc}}
\newcommand{\bv}[1]{\boldsymbol{#1}}
\newcommand{\ahat}{\hat{a}}
\newcommand{\adag}{\ahat^{\dag}}
\newcommand{\braketacomm}[1]{\left\langle\left\{#1\right\} \right\rangle}
\newcommand{\braketcomm}[1]{\left\langle\left[#1\right] \right\rangle}
\newcommand{\ketbra}[2]{\Ket{#1}\!\Bra{#2}}
\newcommand{\mc}[1]{\mathcal{#1}}
\newcommand{\bc}[1]{\bv{\mc{#1}}}

\begin{document}
\title{Proof Cases}
\author{Justin Gerber}
\date{\today}
\maketitle

\section*{Introduction}

In this documents I'll prove the many subcases for showing extension by definition is a conservative extension.


\section{Inference Rules}

\hrulefill
\begin{ND}[Rule of $A$][][][][.6\linewidth]
\ndljg{j}{(j)}{$\mc{A}$}{$A$}
\end{ND}
Any Wff can be assumed at any time

\hrulefill
\begin{ND}[Rule of $\land I$][][][][.6\linewidth]
\ndljg{X}{(j)}{$\mc{A}$}{}
\ndljg{Y}{(k)}{$\mc{B}$}{}
\ndljg{X\cup Y}{(l)}{$(\mc{A}\land\mc{B})$}{$j,k,\land I$}
\end{ND}
\hrulefill
\begin{ND}[Rule of $\land E$][][][][.6\linewidth]
\ndljg{X}{(j)}{$(\mc{A}\land\mc{B})$}{}
\ndljg{X}{(k)}{$\mc{A}$}{$j,\land E$}
\ndljg{}{\text{or}}{}{}
\ndljg{X}{(k)}{$\mc{B}$}{$j,\land E$}
\end{ND}
\hrulefill
\begin{ND}[Rule of $\lor I$][][][][.6\linewidth]
\ndljg{j}{(j)}{$\mc{A}$}{ }
\ndljg{j}{(k)}{$(\mc{A} \lor \mc{B})$}{$j,\lor I$}
\ndljg{}{\text{or}}{}{}
\ndljg{j}{(k)}{$(\mc{B} \lor \mc{A})$}{$j,\lor I$}
\end{ND}
\hrulefill
\begin{ND}[Rule of $\lor E$][][][][.6\linewidth]
\ndljg{X}{(g)}{$(\mc{A}\lor\mc{B})$}{}
\ndljg{h}{(h)}{$\mc{\mathcal{A}}$}{$A$}
\ndljg{Y}{(i)}{$\mc{C}$}{}
\ndljg{j}{(j)}{$\mc{B}$}{$A$}
\ndljg{Z}{(k)}{$\mc{C}$}{}
\ndljg{X\cup Y/h \cup Z/j}{(l)}{$\mc{C}$}{$g,h,i,j,k,\lor E$}
\end{ND}
\hrulefill
\begin{ND}[Rule of $\implies I$][][][][.6\linewidth]
\ndljg{j}{(j)}{$\mc{A}$}{$A$}
\ndljg{X}{(k)}{$\mc{B}$}{}
\ndljg{X/j}{(l)}{$(\mc{A}\implies \mc{B})$}{$j,k,\implies I$}
\end{ND}
\hrulefill
\begin{ND}[Rule of $\implies E$][][][][.6\linewidth]
\ndljg{X}{(j)}{$(\mc{A}\implies\mc{B})$}{}
\ndljg{Y}{(k)}{$\mc{A}$}{}
\ndljg{X\cup Y}{(l)}{$\mc{B}$}{$j,k,\implies E$}
\end{ND}
\hrulefill
\begin{ND}[Rule of $\iff I$][][][][.6\linewidth]
\ndljg{X}{(j)}{$(\mc{A}\implies\mc{B})$}{}
\ndljg{Y}{(k)}{$(\mc{B}\implies\mc{A})$}{}
\ndljg{X\cup Y}{(l)}{$(\mc{A}\iff \mc{B})$}{$j,k,\iff I$}
\ndljg{}{\text{or}}{}{}
\ndljg{X\cup Y}{(l)}{$(\mc{B}\iff \mc{A})$}{$j,k,\iff I$}
\end{ND}
\hrulefill
\begin{ND}[Rule of $\iff E$][][][][.6\linewidth]
\ndljg{X}{(j)}{$(\mc{A}\iff\mc{B})$}{}
\ndljg{X}{(k)}{$(\mc{A}\implies\mc{B})$}{$j,\iff E$}
\ndljg{}{\text{or}}{}{}
\ndljg{X}{(k)}{$(\mc{B}\implies\mc{A})$}{$j,\iff E$}
\end{ND}
\hrulefill
\begin{ND}[Rule of $\lnot I$][][][][.6\linewidth]
\ndljg{j}{(j)}{$\mc{A}$}{$A$}
\ndljg{X}{(k)}{$\curlywedge$}{ }
\ndljg{X/j}{(l)}{$(\lnot \mc{A})$}{$j,k,\lnot I$}
\end{ND}
\hrulefill
\begin{ND}[Rule of $\lnot E$][][][][.6\linewidth]
\ndljg{X}{(j)}{$(\lnot\mc{A})$}{}
\ndljg{Y}{(k)}{$\mc{A}$}{ }
\ndljg{X\cup Y}{(l)}{$\curlywedge$}{$j,k,\lnot E$}
\end{ND}
\hrulefill
\begin{ND}[Rule of $DN$][][][][.6\linewidth]
\ndljg{X}{(j)}{$(\lnot(\lnot\mathcal{A}))$}{}
\ndljg{X}{(k)}{$\mathcal{A}$}{$j,DN$}
\end{ND}
\hrulefill
\begin{ND}[Rule of $\exists I $][][][][.6\linewidth]
\ndljg{X}{(j)}{$\mc{A}[v\leftarrow t]$}{}
\ndljg{X}{(k)}{$((\exists v)\mc{A})$}{$j,\exists I$}
\end{ND}

$t$ is a closed term and $v$ must appear free and not bound in $\mc{A}$

\hrulefill
\begin{ND}[Rule of $\exists E$][][][][.6\linewidth]
\ndljg{X}{(i)}{$((\exists x)\mc{A})$}{}
\ndljg{j}{(j)}{$\mc{A}[x\leftarrow t]$}{$A$}
\ndljg{Y}{(k)}{$\mc{B}$}{}
\ndljg{X\cup Y/j}{(l)}{$\mc{B}$}{$i,j,k,\exists E$}
\end{ND}
$t$ is a closed term which does not appear in $\mc{A}, \mc{B}$ or any of the lines $Y$ other than $j$.

\hrulefill
\begin{ND}[Rule of $\forall I$][][][][.6\linewidth]
\ndljg{X}{(j)}{$\mc{A}[v\leftarrow t]$}{}
\ndljg{X}{(k)}{$((\forall v)\mc{A})$}{$j,\forall I$}
\end{ND}

$t$ is a closed term which does not appear\footnote{The requirement that $t$ does not appear in $\mc{A}$ ensures that all occurrences of $t$ are replaced by $v$ after the quantifier is introduced.} in $\mc{A}$ or any of the lines $X$. $v$ must appear free and not bound in $\mc{A}$.



\hrulefill
\begin{ND}[Rule of $\forall E$][][][][.6\linewidth]
\ndljg{X}{(j)}{$((\forall x)\mathcal{A})$}{}
\ndljg{X}{(k)}{$\mathcal{A}[x\leftarrow t]$}{$j,\forall E$}
\end{ND}
$t$ is a closed term.

\hrulefill
\begin{ND}[Rule of $=I$][][][][.6\linewidth]
\ndljg{X}{(j)}{$t=t$}{$=I$}
\end{ND}
$t$ is a closed term

\hrulefill
\begin{ND}[Rule of $=E$][][][][.6\linewidth]
\ndljg{X}{(j)}{$t_1=t_2$}{}
\ndljg{Y}{(k)}{$\mc{A}[v\leftarrow t_1]$}{}
\ndljg{X\cup Y}{(l)}{$\mc{A}[v \leftarrow t_2]$}{$j,k,=E$}
\end{ND}
$t_1$ and $t_2$ are closed terms and $v$ appears free in $\mc{A}$.

\hrulefill

We can call the set of all inference rules $\bc{I}$.

\subsubsection*{New Predicate}

To add a new $n$-ary predicate to the theory we append the predicate symbol $P$ to the set of symbols $\mathbf{Symb}$ to construct $\mathbf{Symb}' = \mathbf{Symb}\cup \{P\}$ for $\bv{\mc{L}}'$.

We introduce the axiom

\begin{equation}
\textbf{def} \equiv (P\vec{x} \iff \mc{Q}(\vec{x}))
\end{equation}

and define $\Gamma' = \Gamma \cup \{\textbf{def}\} \}$

The notation $\mc{Q}(\vec{x})$ indicates that $\mc{Q}$ is a Wff whose only free variables are $\vec{x}$.

\subsubsection*{New Function or Constant}

To add a new $n$-ary function to the theory we append the function symbol $f$ to the set of symbols $\textbf{Symb}$ to construct $\textbf{Symb}' = \textbf{Symb}\cup\{f\}$ for $\bv{\mc{L}}$.

We introduce the axiom

\begin{equation}
\textbf{def} \equiv \forall \vec{x} \forall y(y=f(\vec{x})\iff \mc{Q}(\vec{x},y))
\end{equation}

and define $\Gamma' = \Gamma \cup \{\textbf{def}\}$.

In addition we require that the formula

\begin{equation}
\Gamma \vdash (\forall \vec{x})(\exists! y) \mc{Q}(\vec{x},y) \equiv K
\end{equation}

is a theorem of $\bc{T}$.

Unique quantification here is defined as follows. Suppose $y$ appears free in $\mc{A}$.

\begin{equation}
((\exists!y)\mc{A}) \equiv ((\exists y)(\mc{A} \land ((\forall z)\mc{A}[y\leftarrow z] \implies y=z)))
\end{equation}

The intuitive reason we require $\Gamma \vdash K$ is because if we did not then it would be possible for $f(\vec{x})$ to refer to multiple different 'objects' so it would not be entirely unambiguous how to interpret a Wff involving $f$. More rigorously, we will see that this formula $K$ will be necessary in proving that the new theory $\bc{T}'$ is in fact a conservative extension of $\bc{T}$.

\section{Definition Translation}

To prove extensions by definition are conservative we will need the notion of the translation of a formula from $\bv{\mc{L}}'$ into $\bv{\mc{L}}$. Since $\bv{\mc{L}}'$ contains more symbols than $\bv{\mc{L}}$ it is of course not possible to express all formulas in $\bv{\mc{L}}'$ in language $\bv{\mc{L}}$, but the idea is that for every $\phi$ of $\bv{\mc{L}}'$ there is a sentence $\phi^\sharp$ in $\bv{\mc{L}}$ with the property that $\vdash_{\bc{T}'}\phi$ if and only if $\vdash_{\bc{T}}\phi^{\sharp}$. This is the sense in which $\phi$ and $\phi^{\sharp}$ are equivalent, even though they may be different strings.

That task is then to define $\phi^{\sharp}$ for a given $\phi$ as well as to prove that if $\vdash_{\bc{T}'}\phi$ that $\vdash_{\bc{T}}\phi^{\sharp}$. Note that the converse is easy to prove. Also note that once we have proven this then it will follow immediately that $\bc{T}'$ is a conservative extension of $\bc{T}$ because we can apply the above rule to a sentence $\phi \in \textbf{Wff}$ which would mean $\phi^{\sharp} \equiv \phi$.

First we define the translation of a sentence in the case that $\bc{T}'$ extends $\bc{T}$ by adding the preposition $P$ and axiom $\textbf{def}$ above for prepositions.

We define $\phi^{\sharp}$ recursively on the structure of the Wff for $\phi$. 

\begin{itemize}
\item{If $\mc{A}$ is atomic and $\mc{A} \equiv P(\vec{x})$ then $(\mc{P}(\vec{x}))^{\sharp} \equiv \mc{Q}(\vec{x})$. Otherwise $\mc{A}^{\sharp} \equiv \mc{A}$.}
\item{$(\mc{A}\land \mc{B})^{\sharp} \equiv (\mc{A}^{\sharp} \land \mc{B}^{\sharp})$}
\item{$(\mc{A} \lor \mc{B})^{\sharp} \equiv (\mc{A}^{\sharp} \lor \mc{B}^{\sharp})$}
\item{$(\mc{A} \implies \mc{B})^{\sharp} \equiv (\mc{A}^{\sharp} \implies \mc{B}^{\sharp})$}
\item{$(\mc{A}\iff \mc{B})^{\sharp} \equiv (\mc{A}^{\sharp} \iff \mc{B}^{\sharp})$}
\item{$(\lnot \mc{A})^{\sharp} \equiv (\lnot \mc{A}^{\sharp})$}
\item{$((\exists x)\mc{A})^{\sharp} \equiv ((\exists v)(\mc{A}[x\leftarrow v])^{\sharp})$ where $v$ is the first unused variable.}
\item{$((\forall x)\mc{A})^{\sharp} \equiv ((\forall v)(\mc{A}[x\leftarrow v])^{\sharp})$ where $v$ is the first unused variable.}
\item{$(t_1=t_2)^{\sharp} \equiv t_1=t_2$}
\end{itemize}

\section*{Translation proof for New Predicate}

I would like to prove that if $\vdash_{\bc{T}'}\phi$ that $\vdash_{\bc{T}}\phi^{\sharp}$. To do this I will induct on proof steps in a Lemmon style proof. I will do this by performing metatheorems on lines of Lemmon style proofs. What I will prove is the following.

Suppose that the following line appears in a Lemmon style derivation:

\begin{ND}[][][][][.6\linewidth]
\ndljg{X}{(k)}{$\phi$}{$R$}
\end{ND}

for any rule $R$.
Suppose also that $\textbf{Def}$, the definition axiom, appears on line $d$. It then follows that it is possible to derive

\begin{ND}[][][][][.6\linewidth]
\ndljg{X^{\sharp}/d}{(l)}{$\phi^{\sharp}$}{$R^{\sharp}$}
\end{ND}

Where for each $i\in X$ with $\psi_i$ appearing on line $i$ we have a corresponding $j \in X^{\sharp}$ with $\psi^{\sharp}$ appearing on line $j$ and no additional elements in $X^{\sharp}$. There is no constraint on $R^{\sharp}$.

I will prove this by inducting on the line number on which $\phi$ appears.

\subsection*{Base Case}

Suppose that $\phi$ appears on line 1 of a derivation. This could have happened if 1) $\phi \in \Gamma'$ and was introduced as a premise, 2) $\phi$ was introduced as an assumption using rule $A$ (and depends only on line 1, or 3) $\phi \equiv t=t$ for some term $t$ and was introduced by $=I$. We work out each case. Note that in each case the derived statement $\phi$ depends on line 1 and no other lines

\subsubsection*{$\phi \in \Gamma'$ arising as premise}

If $\phi \in \Gamma'$ then either $\phi \in \Gamma$ or $\phi \equiv \textbf{def} \equiv ((\forall \vec{x})(P\vec{x} \Leftrightarrow \mc{Q})$. If $\phi \in \Gamma \subset \textbf{Wff}$ then $\phi^{\sharp} \equiv \phi$ and it is possible to derive $\phi^{\sharp} \in \Gamma$ as a premise using $\bc{T}$. We see that $\phi^{\sharp}$ would depend on line 1.

If $\phi\equiv \textbf{def}\equiv ((\forall \vec{x})(P\vec{x}\Leftrightarrow \mc{Q}))$ then $\phi^{\sharp} \equiv (\tilde{\mc{Q}}\iff\mc{Q})$. This statement is a tautology which can be derived depending on no lines in $\mc{T}$! This is exactly what is needed since in this case we have that $X=\{1\}$ and $d=1$ so $X/d = X/1 = \{\}$.

\subsubsection*{$\phi$ Assumed}

if $\phi$ was assumed at line 1 then we could equivalently assume $\phi^{\sharp}$ at line 1 using $\bc{T}$.

\subsubsection*{Equality Introduction}

If $\phi\equiv t=t$ for some term $t$ then it is impossible for any predicates to appear so $\phi \equiv \phi^{\sharp}$ so $\phi^{\sharp}$ also could have been derived in one line using $=I$ in $\bc{T}$.

\subsection*{Induction Step}

For the induction step we assume that the theorem holds for lines $1\ldots n-1$. That is, if the following line appears with $k < n$

\begin{ND}[][][][][.6\linewidth]
\ndljg{X}{(k)}{$\phi$}{$R$}
\end{ND}

then we assume it is possible to derive

\begin{ND}[][][][][.6\linewidth]
\ndljg{X^{\sharp}/d}{(l)}{$\phi^{\sharp}$}{$R^{\sharp}$}
\end{ND}

With $X^{\sharp}$ defined as above. The goal then, is to prove that the same holds for $k=n$. I will do this by ``adding the last step of the proof in by hand''. That is, if the derivation of $\phi$ in $\bc{T}'$ is $n$ steps long then the inductions hypothesis guarantees that all of the lines through $n-1$ can be derived in $\bc{T}$. The goal is to then use a final inference rule to derive $\phi^{\sharp}$ in $\bc{T}$

Unfortunately we need to break out into cases for each possible final rule used in the proof of $\phi$.

\hrulefill

\subsubsection{Rule of $A$}

If the last line of the proof for $\phi$ is an assumption introduction then it only depends on line $n$ and $\phi^{\sharp}$ can simply be assumed in $\bc{T}$.

If we have

\begin{ND}[][][][][.6\linewidth]
\ndljg{n}{(n)}{$\phi$}{$A$}
\end{ND}

Then we can get

\begin{ND}[][][][][.6\linewidth]
\ndljg{n^{\sharp}}{(n^{\sharp})}{$\phi^{\sharp}$}{$A$}
\end{ND}

We did not need the induction hypothesis. Also note that $n\neq d$ so $n^{\sharp}/d = n^{\sharp}$.

\hrulefill

\subsubsection*{Rule of $\land I$}

If the last rule is $\land I$ then we have

\begin{ND}[][][][][.6\linewidth]
\ndljg{X}{(j)}{$\mc{A}$}{}
\ndljg{Y}{(k)}{$\mc{B}$}{}
\ndljg{X\cup Y}{(n)}{$(\mc{A}\land\mc{B})$}{$j,k,\land I$}
\end{ND}

$\phi \equiv (\mc{A}\land\mc{B})$. The proof would be the same if it was instead $(\mc{B}\land \mc{A})$. The induction hypothesis gives us

\begin{ND}[][][][][.6\linewidth]
\ndljg{X^{\sharp}/d}{(j^{\sharp})}{$\mc{A}^{\sharp}$}{}
\ndljg{Y^{\sharp}/d}{(k^{\sharp})}{$\mc{B}^{\sharp}$}{}
\end{ND}

From which we see can easily derive

\begin{ND}[][][][][.6\linewidth]
\ndljg{(X^{\sharp}/d\cup Y^{\sharp}/d)}{(n^{\sharp})}{$(\mc{A}^{\sharp}\land\mc{B}^{\sharp})$}{$j^{\sharp},k^{\sharp},\land I$}
\end{ND}

and we note that $(\mc{A}^{\sharp}\land\mc{B}^{\sharp})\equiv (\mc{A}\land\mc{B})^{\sharp}\equiv \phi^{\sharp}$ and $(X^{\sharp}/d\cup Y^{\sharp}/d) = (X\cup Y)^{\sharp}/d$.


\hrulefill

\subsubsection*{Rule of $\land E$}

If the last rule is $\land E$ then we have

\begin{ND}[][][][][.6\linewidth]
\ndljg{X}{(j)}{$\mc{A}\land\mc{B}$}{}
\ndljg{X}{(n)}{$\mc{A}$}{$j,\land E$}
\end{ND}

With $\phi \equiv \mc{A}$. We could have equivalently inferred $\mc{B}$ and the proof would have gone the same.

The induction hypothesis give us

\begin{ND}[][][][][.6\linewidth]
\ndljg{X^{\sharp}/d}{(j^{\sharp})}{$(\mc{A}\land\mc{B})^{\sharp} \equiv (\mc{A}^{\sharp} \land \mc{B}^{\sharp})$}{}
\end{ND}

From which we can easily derive

\begin{ND}[][][][][.6\linewidth]
\ndljg{X^{\sharp}/d}{(n^{\sharp})}{$\mc{A}^{\sharp} \equiv \phi^{\sharp}$}{$j^{\sharp},\land E$}
\end{ND}

\hrulefill

\subsubsection*{Rule of $\lor I$}

If the last rule is $\lor I$ then we have

\begin{ND}[][][][][.6\linewidth]
\ndljg{X}{(j)}{$\mc{A}$}{}
\ndljg{X}{(n)}{$(\mc{A} \lor \mc{B})$}{$j,\lor I$}
\end{ND}

With $\phi \equiv (\mc{A} \lor \mc{B})$. We could have equivalently inferred $(\mc{B} \lor \mc{A})$ and the proof would have gone the same.

The induction hypothesis gives us

\begin{ND}[][][][][.6\linewidth]
\ndljg{X^{\sharp}/d}{(j^{\sharp})}{$\mc{A}^{\sharp}$}{}
\end{ND}

From which we can infer

\begin{ND}[][][][][.6\linewidth]
\ndljg{X^{\sharp}/d}{(n^{\sharp})}{$(\mc{A}^{\sharp}\lor \mc{B}^{\sharp})$}{$j^{\sharp}, \lor I$}
\end{ND}

and we note that $(\mc{A}^{\sharp}\lor \mc{B}^{\sharp}) \equiv (\mc{A} \lor \mc{B})^{\sharp} \equiv \phi^{\sharp}$.

\hrulefill

\subsubsection*{Rule of $\lor E$}

If the last rule is $\lor E$ then we have

\begin{ND}[][][][][.6\linewidth]
\ndljg{X}{(g)}{$(\mc{A} \lor \mc{B})$}{}
\ndljg{h}{(h)}{$\mc{A}$}{$A$}
\ndljg{Y}{(i)}{$\mc{C}$}{}
\ndljg{j}{(j)}{$\mc{B}$}{$A$}
\ndljg{Z}{(k)}{$\mc{C}$}{}
\ndljg{X\cup Y/h\cup Z/j}{(n)}{$\mc{C}$}{$g,h,i,j,k,\lor E$}
\end{ND}

With $\phi \equiv \mc{C}$.

The induction hypothesis gives us

\begin{ND}[][][][][.6\linewidth]
\ndljg{X^{\sharp}/d}{(g^{\sharp})}{$(\mc{A} \lor \mc{B})^{\sharp} \equiv (\mc{A}^{\sharp}\lor \mc{B}^{\sharp})$}{}
\ndljg{h^{\sharp}}{(h^{\sharp})}{$\mc{A}^{\sharp}$}{$A$}
\ndljg{Y^{\sharp}/d}{(i^{\sharp})}{$\mc{C}^{\sharp}$}{}
\ndljg{j^{\sharp}}{(j^{\sharp})}{$\mc{B}^{\sharp}$}{$A$}
\ndljg{Z^{\sharp}/d}{(k^{\sharp})}{$\mc{C}^{\sharp}$}{}
\end{ND}

From which we can infer

\begin{ND}[][][][][.8\linewidth]
\ndljg{(X^{\sharp}\cup Y^{\sharp}/h \cup Z^{\sharp}/j)/d}{(n^{\sharp})}{$\mc{C}^{\sharp}\equiv \phi^{\sharp}$}{$g^{\sharp},h^{\sharp},i^{\sharp},j^{\sharp},k^{\sharp},\lor E$}
\end{ND}

Note that the assumptions on lines $h^{\sharp}$ and $j^{\sharp}$ do not follow from the induction hypothesis but rather from the fact that we can always infer any formula as an assumption.

\hrulefill

\subsubsection*{Rule of $\implies I$}

If the last rule is $\implies I$ then we have

\begin{ND}[][][][][.6\linewidth]
\ndljg{j}{(j)}{$\mc{A}$}{$A$}
\ndljg{X}{(k)}{$\mc{B}$}{}
\ndljg{X/j}{(n)}{$(\mc{A} \implies \mc{B})$}{$j,k,\implies I$}
\end{ND}

With $\phi \equiv (\mc{A} \implies \mc{B})$. The induction hypothesis (and being able to infer any formula as an assumption) gives us

\begin{ND}[][][][][.6\linewidth]
\ndljg{j}{(j^{\sharp})}{$\mc{A}^{\sharp}$}{$A$}
\ndljg{X^{\sharp}/d}{(k^{\sharp})}{$\mc{B}^{\sharp}$}{}
\end{ND}

From which we can infer

\begin{ND}[][][][][.6\linewidth]
\ndljg{X^{\sharp}/j/d}{(n^{\sharp})}{$(\mc{A}^{\sharp}\implies \mc{B}^{\sharp})$}{$j^{\sharp},k^{\sharp},\implies I$}
\end{ND}

Noting that $(\mc{A}^{\sharp}\implies \mc{B}^{\sharp}) \equiv (\mc{A}\implies\mc{B})^{\sharp} \equiv \phi^{\sharp}$.

\hrulefill

\subsubsection*{Rule of $\implies E$}

If the last rule is $\implies E$ then we have

\begin{ND}[][][][][.6\linewidth]
\ndljg{X}{(j)}{$(\mc{A}\implies\mc{B})$}{}
\ndljg{Y}{(k)}{$\mc{A}$}{}
\ndljg{X\cup Y}{(n)}{$\mc{B}$}{$j,k,\land E$}
\end{ND}


With $\phi \equiv \mc{B}$.
The induction hypothesis gives us

\begin{ND}[][][][][.6\linewidth]
\ndljg{X^{\sharp}/d}{(j^{\sharp})}{$(\mc{A}\implies\mc{B})^{\sharp} \equiv (\mc{A}^{\sharp}\implies \mc{B}^{\sharp})$}{}
\ndljg{Y^{\sharp}/d}{(k^{\sharp})}{$\mc{A^{\sharp}}$}{}
\end{ND}

From which we can infer

\begin{ND}[][][][][.6\linewidth]
\ndljg{(X^{\sharp}\cup Y^{\sharp})/d}{(n^{\sharp})}{$\mc{B}^{\sharp}\equiv\phi^{\sharp}$}{$j^{\sharp},k^{\sharp},\implies E$}
\end{ND}

\hrulefill

\subsubsection*{Rule of $\iff I$}

If the last rule is $\iff I$ then we have

\begin{ND}[][][][][.6\linewidth]
\ndljg{X}{(j)}{$(\mc{A}\implies\mc{B})$}{}
\ndljg{Y}{(k)}{$(\mc{B}\implies\mc{A})$}{}
\ndljg{X\cup Y}{(n)}{$(\mc{A}\Leftrightarrow \mc{B})$}{$j,k,\Leftrightarrow I$}
\end{ND}

with $\phi \equiv (\mc{A}\iff\mc{B})$. The induction hypothesis gives us

\begin{ND}[][][][][.6\linewidth]
\ndljg{X^{\sharp}/d}{(j^{\sharp})}{$(\mc{A}\implies\mc{B})^{\sharp} \equiv (\mc{A}^{\sharp}\implies \mc{B}^{\sharp})$}{}
\ndljg{Y^{\sharp}/d}{(k^{\sharp})}{$(\mc{B}\implies\mc{A})^{\sharp}\equiv(\mc{B}^{\sharp}\implies \mc{A}^{\sharp})$}{}
\end{ND}

From which we can infer

\begin{ND}[][][][][.6\linewidth]
\ndljg{(X^{\sharp}\cup Y^{\sharp})/d}{(n^{\sharp})}{$(\mc{A}^{\sharp}\iff\mc{B}^{\sharp})$}{$j^{\sharp},k^{\sharp},\iff I$}
\end{ND}

We note that $(\mc{A}^{\sharp}\iff\mc{B}^{\sharp}) \equiv (\mc{A}\iff \mc{B})^{\sharp} \equiv \phi^{\sharp}$.

\hrulefill

\subsubsection*{Rule of $\iff E$}

If the last rule is $\iff E$ then we have

\begin{ND}[][][][][.6\linewidth]
\ndljg{X}{(j)}{$(\mc{A}\iff\mc{B})$}{}
\ndljg{X}{(n)}{$(\mc{A}\implies\mc{B})$}{}
\end{ND}

with $\phi \equiv (\mc{A}\implies \mc{B})$. The induction hypothesis gives us

\begin{ND}[][][][][.6\linewidth]
\ndljg{X^{\sharp}/d}{(j^{\sharp})}{$(\mc{A}\iff\mc{B})^{\sharp} \equiv (\mc{A}^{\sharp}\iff \mc{B}^{\sharp})$}{}
\end{ND}

From which we can infer

\begin{ND}[][][][][.6\linewidth]
\ndljg{X^{\sharp}/d}{(n^{\sharp})}{$(\mc{A}^{\sharp}\implies\mc{B}^{\sharp})$}{$j^{\sharp},\iff E$}
\end{ND}

Noting that $(\mc{A}^{\sharp}\implies\mc{B}^{\sharp})\equiv (\mc{A} \implies \mc{B})^{\sharp} \equiv \phi^{\sharp}$. The proof the same for $(\mc{B}\implies \mc{A})$.

\hrulefill

\subsubsection*{Rule of $\lnot I$}

if the last rule is $\lnot I$ then we have

\begin{ND}[][][][][.6\linewidth]
\ndljg{j}{(j)}{$\mc{A}$}{$A$}
\ndljg{X}{(k)}{$\curlywedge$}{}
\ndljg{X/j}{(n)}{$(\lnot \mc{A})$}{$j,k,\lnot I$}
\end{ND}

with $\phi \equiv (\lnot \mc{A})$. The induction hypothesis gives us

\begin{ND}[][][][][.6\linewidth]
\ndljg{j^{\sharp}}{(j^{\sharp})}{$\mc{A}^{\sharp}$}{$A$}
\ndljg{X^{\sharp}/d}{(k^{\sharp})}{$\curlywedge^{\sharp} \equiv \curlywedge$}{}
\end{ND}

From which we can infer

\begin{ND}[][][][][.6\linewidth]
\ndljg{X^{\sharp}/j/d}{(n^{\sharp})}{$(\lnot \mc{A}^{\sharp})$}{$j^{\sharp},k^{\sharp},\lnot I$}
\end{ND}

Noting that $(\lnot \mc{A}^{\sharp})\equiv (\lnot \mc{A})^{\sharp} \equiv \phi^{\sharp}$.

\hrulefill

\subsubsection*{Rule of $\lnot E$}
If the last rule is $\lnot E$ then we have

\begin{ND}[][][][][.6\linewidth]
\ndljg{X}{(j)}{$(\lnot\mc{A})$}{}
\ndljg{Y}{(k)}{$\mc{A}$}{ }
\ndljg{X\cup Y}{(n)}{$\curlywedge$}{$j,k,\lnot E$}
\end{ND}

With $\phi \equiv \curlywedge$. The induction hypothesis gives us

\begin{ND}[][][][][.6\linewidth]
\ndljg{X^{\sharp}/d}{(j^{\sharp})}{$(\lnot\mc{A})^{\sharp}\equiv(\lnot \mc{A}^{\sharp})$}{}
\ndljg{Y^{\sharp}/d}{(k^{\sharp})}{$\mc{A}^{\sharp}$}{}
\end{ND}

From which we can infer

\begin{ND}[][][][][.6\linewidth]
\ndljg{(X^{\sharp}\cup Y^{\sharp})/d}{(n^{\sharp})}{$\curlywedge^{\sharp} \equiv \curlywedge$}{$j^{\sharp},k^{\sharp},\lnot E$}
\end{ND}

noting that $\phi^{\sharp} \equiv \curlywedge$.

\hrulefill

\subsubsection*{Rule of $DN$}
If the last rule is $DN$ then we have

\begin{ND}[][][][][.6\linewidth]
\ndljg{X}{(j)}{$(\lnot(\lnot\mc{A}))$}{}
\ndljg{X}{(n)}{$\mc{A}$}{$j,DN$}
\end{ND}

with $\phi \equiv \mc{A}$. The induction hypothesis gives us

\begin{ND}[][][][][.6\linewidth]
\ndljg{X^{\sharp}/d}{(j^{\sharp})}{$(\lnot(\lnot\mc{A}))^{\sharp} \equiv (\lnot(\lnot\mc{A}^{\sharp}))$}{}
\end{ND}

From which we can infer

\begin{ND}[][][][][.6\linewidth]
\ndljg{X^{\sharp}/d}{(n^{\sharp})}{$\mc{A}^{\sharp}$}{$j^{\sharp},DN$}
\end{ND}

With $\phi^{\sharp}\equiv \mc{A}^{\sharp}$.

\hrulefill

\subsubsection*{Rule of $\exists I1$}

If the last rule is $\exists I$ then we have

\begin{ND}[][][][][.6\linewidth]
\ndljg{X}{(j)}{$\mc{A}$}{}
\ndljg{X}{(n)}{$((\exists x)\mc{A}[t\xleftarrow[1]{} x])$}{$j,\exists I$}
\end{ND}

$t$ is a closed term appearing in $\mc{A}$. It is necessary that variable $x$ does not appear (free or bound) in $\mc{A}$. Here $\phi \equiv ((\exists x)\mc{A}[t\xleftarrow[1]{} x])$. The induction hypothesis gives us

\begin{ND}[][][][][.6\linewidth]
\ndljg{X^{\sharp}/d}{(j^{\sharp})}{$\mc{A}^{\sharp}$}{}
\end{ND}

We know that $t$ is a closed term appearing in $\mc{A}$. Choose $y$ as the next variable which does not appear in $\mc{A}$ or $\mc{Q}$ and we can infer

\begin{ND}[][][][][.6\linewidth]
\ndljg{X^{\sharp}/d}{(n^{\sharp})}{$((\exists y)\mc{A}^{\sharp}[t\xleftarrow[1]{}y])$}{$j^{\sharp},\exists I$}
\end{ND}

Note that since $t$ appears in $\mc{A}$ it is possible to choose the replacement $\mc{A}^{\sharp}[t\xleftarrow[1]{}y]$ such that $\mc{A}^{\sharp}[t\xleftarrow[1]{}y] \equiv (\mc{A}[t\xleftarrow[1]{}y])^{\sharp} \equiv (\mc{A}[t \xleftarrow[1]{}x][x\leftarrow y])^{\sharp}$. so that $((\exists y)\mc{A}^{\sharp}[t\xleftarrow[1]{}y]) \equiv ((\exists x)\mc{A}[t\xleftarrow[1]{}x])^{\sharp} \equiv \phi^{\sharp}$.

\hrulefill


\subsubsection*{Rule of $\exists I2$}

If the last rule is $\exists I$ then we have

\begin{ND}[][][][][.6\linewidth]
\ndljg{X}{(j)}{$\mc{A}[v\leftarrow t]$}{}
\ndljg{X}{(n)}{$((\exists v)\mc{A})$}{$j,\exists I$}
\end{ND}

$t$ is a closed term appearing in $\mc{A}$ and $v$ the first unused variable appearing free and not bound in $\mc{A}$.

Here $\phi \equiv ((\exists v)\mc{A})$. The induction hypothesis gives us

\begin{ND}[][][][][.6\linewidth]
\ndljg{X^{\sharp}/d}{(j^{\sharp})}{$\mc{A}[v\leftarrow t]^{\sharp}$}{}
\end{ND}

From this we can infer



We know that $t$ is a closed term appearing in $\mc{A}$. Choose $y$ as the next variable which does not appear in $\mc{A}$ or $\mc{Q}$ and we can infer

\begin{ND}[][][][][.6\linewidth]
\ndljg{X^{\sharp}/d}{(n^{\sharp})}{$((\exists y)\mc{A}^{\sharp}[t\xleftarrow[1]{}y])$}{$j^{\sharp},\exists I$}
\end{ND}

Note that since $t$ appears in $\mc{A}$ it is possible to choose the replacement $\mc{A}^{\sharp}[t\xleftarrow[1]{}y]$ such that $\mc{A}^{\sharp}[t\xleftarrow[1]{}y] \equiv (\mc{A}[t\xleftarrow[1]{}y])^{\sharp} \equiv (\mc{A}[t \xleftarrow[1]{}x][x\leftarrow y])^{\sharp}$. so that $((\exists y)\mc{A}^{\sharp}[t\xleftarrow[1]{}y]) \equiv ((\exists x)\mc{A}[t\xleftarrow[1]{}x])^{\sharp} \equiv \phi^{\sharp}$.

\hrulefill

\subsubsection*{Rule of $\exists E$}

If the last rule is $\exists E$ then we have

\begin{ND}[][][][][.6\linewidth]
\ndljg{X}{(i)}{$((\exists x)\mc{A})$}{}
\ndljg{j}{(j)}{$\mc{A}[x\leftarrow t]$}{$A$}
\ndljg{Y}{(k)}{$\mc{B}$}{}
\ndljg{X\cup Y/j}{(n)}{$\mc{B}$}{$i,j,k,\exists E$}
\end{ND}
$t$ is a closed term which does not appear in $\mc{A}, \mc{B}$ or any of the lines $Y$ other than $j$. Here $\phi \equiv \mc{B}$. The induction hypothesis gives us ($y$ is first variable not used in $\mc{A}$ or $\mc{Q}$).

\begin{ND}[][][][][.8\linewidth]
\ndljg{X^{\sharp}/d}{(i^{\sharp})}{$((\exists x)\mc{A})^{\sharp}\equiv((\exists y)(\mc{A}[x\leftarrow y])^{\sharp})$}{}
\ndljg{j^{\sharp}}{(j^{\sharp})}{$(\mc{A}[x\leftarrow y])^{\sharp}[y\leftarrow t]$}{$A$}
\ndljg{Y^{\sharp}/d}{(k^{\sharp})}{$\mc{B}^{\sharp}$}{}
\end{ND}

Note that line $(j^{\sharp})$ can either be assumed directly or given $(\mc{A}[x\leftarrow t])^{\sharp} \equiv (\mc{A}[x \leftarrow y][y\leftarrow t])^{\sharp} \equiv (\mc{A}[x\leftarrow y])^{\sharp}[y\leftarrow t]$ it can be arrived at via the induction hypothesis.
From the above we can infer

\begin{ND}[][][][][.6\linewidth]
\ndljg{(X^{\sharp}\cup Y^{\sharp})/j^{\sharp}/d}{(n^{\sharp})}{$\mc{B}^{\sharp}$}{$i^{\sharp},j^{\sharp},k^{\sharp},\exists E$}
\end{ND}

With $\phi^{\sharp} \equiv \mc{B}^{\sharp}$.

\hrulefill

\subsubsection*{Rule of $\forall I$}

If the last rule is $\forall I$ then we have

\begin{ND}[][][][][.6\linewidth]
\ndljg{X}{(j)}{$\mc{A}$}{}
\ndljg{X}{(k)}{$((\forall x)\mc{A}[t\leftarrow x])$}{$j,\forall I$}
\end{ND}

$t$ is a closed term appearing in $\mc{A}$ but not in any of $X$. $x$ does not appear (free or bound) in $\mc{A}$. Here $\phi \equiv ((\forall x)\mc{A}[t\leftarrow x])$. The induction hypothesis gives us

\begin{ND}[][][][][.6\linewidth]
\ndljg{X^{\sharp}/d}{(j^{\sharp})}{$\mc{A}^{\sharp}$}{}
\end{ND}

Let $y$ be the first variable which does not appear in $\mc{A}$ or $\mc{Q}$. We can then infer

\begin{ND}[][][][][.8\linewidth]
\ndljg{X^{\sharp}/d}{(n^{\sharp})}{$((\forall y)\mc{A}^{\sharp}[t\leftarrow y]) \equiv ((\forall y)(\mc{A}[t\leftarrow y])^{\sharp})$}{$j^{\sharp}, \forall I$}
\end{ND}

\end{document}