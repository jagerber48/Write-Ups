\documentclass[12pt]{article}
\usepackage{amssymb, amsmath, amsfonts}

\usepackage[pdftex]{graphicx}
\usepackage{siunitx}
\usepackage{braket}
\usepackage{multirow}
\usepackage{empheq}

\usepackage[
sorting=none,
style=numeric
]{biblatex}
\addbibresource{refs.bib}

\newcommand{\ep}{\epsilon}
\newcommand{\bv}[1]{\boldsymbol{#1}}
\newcommand{\mc}[1]{\mathcal{#1}}
\newcommand{\ul}[1]{\underline{#1}}
	

\begin{document}
\title{What type of account should I send savings to for retirement? An algebra based approach.}
\author{Justin Gerber}
\date{\today}
\maketitle

\section{Introduction}

In this write-up I'll go through some algebra showing the returns for investing a fixed amount of money into either a pre or post tax tax-advantaged retirement account or a non-tax-advantaged post tax account.

I will make a number of assumptions throughout which will Sarah will help point out.

\section{Setup}

I will assume a simple model for someone's financial life. 
The initial time of investment will be called $t_0$ and I will assume that in that year the person pays an income tax rate of $T_0$. 
In this year the person invests a certain amount into a particular account or combination of accounts.
Due to investment returns the amount in the retirement account increases by a factor of $G$.
Finally, at time $t_f$ the person withdraws the entire amount from the investment account.
I assume that at this time the person pays an income tax rate of $T_f$ and that the capital gains tax rate it $T_{CG}$.

\section{Pre-tax tax-advantaged}

Examples of a pre-tax tax-advantaged accounts are traditional 401k or traditional IRA.
A pre-tax tax-advantaged account is assumed to work as follows. A person earns a certain amount of income $I$ from their employer. 

They then, before paying any tax on this, contribute the whole amount $I$ to a pre-tax tax-advantages retirement account.
This grows by a factor of $G$.
Then at time $t_f$ they withdraw the entire amount and pay income tax at rate $T_f$ on it.

\begin{align}
&I\\
&GI\\
&G(1-T_f)I
\end{align}

\section{Post-tax tax-advantaged}

Examples of post-tax tax-advantaged accounts are Roth 401k or Roth IRA.

In a post-tax tax-advantaged account a person earns a certain amount of income $I$ from their employer.
They then promptly pay income tax at rate $T_0$ and invest the remaining amount.
This amount grows by a factor of $G$.
Then at time $t_f$ they withdraw the entire amount without paying any taxes.

\begin{align}
&I\\
&(1-T_0)I\\
&G(1-T_0)I
\end{align}

\section{Post-tax non-tax advantaged}

Many types of accounts are non-tax advantaged.

In a non tax-advantaged post-tax account a person earns income $I$.
They then pay income taxes at rate $T_0$ and invest the remaining amount.
The amount grows by a factor of $G$.
Then at time $t_f$ they withdraw the entire amount. 
They do not pay taxes on the initial premium investment. 
They pay capital gains tax on all of the earnings.

\begin{align}
&I\\
&(1-T_0)I\\
&G(1-T_0)I\\ 
= &(G-1)(1-T_0)I + (1-T_0)I\\
&(G-1)(1-T_0)(1-T_{CG})I + (1-T_0)I\\
= &((G-1)(1-T_0)(1-T_{CG}) + (1-T_0))I
\end{align}

We'll make two simplifications to this formula. We first assume that $(G-1)(1-T_{CG}) \gg 1$.
This is equivalent to the assumptions that $G\\g1$ since $1-T_{CG} \approx 1$.
This allows us to both ignore the second term (basically the premium is negligible compared to the earnings) and it allows us to replace $G-1$ by $G$.

\begin{align}
G(1-T_0)(1-T_{CG})I = G(1 - T_0 - T_{CG} - T_0T_{CG})I
\end{align}

Next we assume that $T_0T_{CG} \ll T_0, T_{CG}$ which is equivalent to $T_0, T_{CG} \ll 1$ so we can ignore the cross term above.

We then have

\begin{align}
G(1-T_0- T_{CG})I
\end{align}

\section{Summary}

\begin{align}
\text{Pre-tax tax-advantaged: }& G(1-T_F)\\
\text{Post-tax tax-advantaged: }& G(1-T_0)\\
\text{Post-tax non-advantaged: }& G(1-T_0-T_{CG})
\end{align}

We see that post-tax tax-advantaged always provides higher gains than post-tax non tax advantaged.

We see that if $T_0<T_F$ that post-tax tax-advantaged gives higher returns then pre-tax tax advantaged and vice-versa if $T_F<T_0$.

Pre-tax tax-advantaged is better that post-tax non-advantaged if $T_F<T_0+T_{CG}$ and vice-versa if $T0 + T_{CG} < T_F$.

\section{Assumptions/ignored things}
\end{document}