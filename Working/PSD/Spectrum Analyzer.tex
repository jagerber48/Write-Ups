\documentclass[12pt]{article}
\usepackage{amssymb, amsmath, amsfonts}

\usepackage{tcolorbox}

\usepackage{bbm}
\usepackage[utf8]{inputenc}
\usepackage{subfigure}%ngerman
%\usepackage[pdftex]{graphicx}
\usepackage{textcomp} 
\usepackage{color}
\usepackage[hidelinks]{hyperref}
\usepackage{anysize}
\usepackage{siunitx}
\usepackage{verbatim}
\usepackage{float}
\usepackage{braket}
\usepackage{xfrac}
\usepackage{array, booktabs} 
\usepackage{tabularx}


\newcommand{\ddt}[1]{\frac{d #1}{dt}}
\newcommand{\ppt}[1]{\frac{\partial #1}{\partial t}}
\newcommand{\ep}{\epsilon}
\newcommand{\sinc}{\text{sinc}}
\newcommand{\bv}[1]{\boldsymbol{#1}}
\newcommand{\ahat}{\hat{a}}
\newcommand{\adag}{\ahat^{\dag}}
\newcommand{\braketacomm}[1]{\left\langle\left\{#1\right\} \right\rangle}
\newcommand{\braketcomm}[1]{\left\langle\left[#1\right] \right\rangle}
\newcommand{\ketbra}[2]{\Ket{#1}\!\Bra{#2}}


\begin{document}
\title{Spectrum Analyzer}
\author{Justin Gerber}
\date{\today}
\maketitle

\section{Introduction}

In this write up I will prove that what is measured by a spectrum analyzer is the power spectral density.

\section{main}

We consider a complex signal $X(t)$.
In general we will suppose that $X(t)$ is infinite in temporal extant, meaning that it contains infinite energy.
Often $X(t)$ will have finite power at all times, however, I will point out that white noise, an important case for consideration, in fact has infinite power at all times.
However, if the frequency bandwidth of white noise is limited by a bandpass or low pass filter then it is again a finite-power signal.

We define a window function $W_0(t)$ which has the property that

\begin{align}
\int |W_1(t)|^2 dt = 1
\end{align}

From this we extend to a family of normalized window functions:

\begin{align}
W_{\Delta T}(t) = \frac{1}{\sqrt{\Delta T}} W_0\left(\frac{t}{\Delta t}\right)
\end{align}

With

\begin{align}
\int |W_{\Delta T}(t)|^2 dt = \int \frac{1}{\Delta T} \left|W_1\left(\frac{t}{\Delta T}\right)\right|^2 dt = \int |W_1(t')|^2 dt' = 1
\end{align}

by a change of variables.

The prototypical window function will be the box window function:

\begin{align}
W^{\theta}_1(t) =& \theta\left(t-\frac{1}{2}\right)\theta\left(\frac{1}{2}-t\right)\\
W^{\theta}_{\Delta T}(t) =& \frac{1}{\sqrt{\Delta T}}\theta\left(t-\frac{\Delta T}{2}\right)\theta\left(\frac{\Delta T}{2}-t\right)
\end{align}

We define the time windowed version of the signal $X(t)$ as

\begin{align}
X_{W_{\Delta T}}(t) = X(t)W_{\Delta T}(t)
\end{align}

The instantaneous power in a signal is simply defined as

\begin{align}
P_X(t) = |X(t)|^2
\end{align}

We define the time averaged power in a signal as

\begin{align}
\bar{P}_X(\Delta T) =& \frac{1}{\Delta T}\int_{t=-\frac{\Delta T}{2}}^{\frac{\Delta T}{2}} P_X(t) dt\\
=& \frac{1}{\Delta T}\int_{t=-\frac{\Delta T}{2}}^{\frac{\Delta T}{2}} |X(t)|^2 dt\\
=& \int |X_{W^{\theta}_{\Delta T}}(t)|^2 dt\\
=& \int P_{X_{W^{\theta}_{\Delta T}}}(t) dt
\end{align}

We can generalize this by just letting

\begin{align}
\bar{P}_X(\Delta T) =& \int P_{X_{W_{\Delta T}}}(t) dt
\end{align}

The normalization condition on $W_{\Delta T}(t)$ ensures that this behaves like a weighted average of the power for any window function $W_{\Delta t}(t)$.

We can think of $W_{\Delta T}(t)$ as a signal in which case we have

\begin{align}
P_{W_{\Delta T}}(t) = |W_{\Delta T}(t)|^2
\end{align}

We define the energy of a signal as

\begin{align}
E_X = \int P_X(t) dt = \int |X(t)|^2 dt
\end{align}

With this we can see

\begin{align}
E_{W_{\Delta T}} = \int |W_{\Delta T}(t)|^2 dt = 1
\end{align}

We can also see that

\begin{align}
\bar{P}_X(\Delta T) = E_{X_{W_{\Delta T}}}
\end{align}

We define the Fourier transform of a signal as

\begin{align}
\tilde{X}(f) = \int e^{-i2\pi f t} X(t) dt
\end{align}

The power spectral density is defined as 

\begin{align}

\end{align}



\end{document}