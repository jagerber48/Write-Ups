\documentclass[12pt]{article}

\usepackage{amssymb, amsmath, amsfonts}

\usepackage{amsthm}

\newtheoremstyle{break}% name
  {}%         Space above, empty = `usual value'
  {}%         Space below
  {\itshape}% Body font
  {}%         Indent amount (empty = no indent, \parindent = para indent)
  {\bfseries}% Thm head font
  {}%        Punctuation after thm head
  {\newline}% Space after thm head: \newline = linebreak
  {}%         Thm head spec
\theoremstyle{break}
\newtheorem{definition}{Definition}[section]
\theoremstyle{break}
\newtheorem{theorem}{Theorem}[section]
\theoremstyle{break}
\newtheorem{corollary}[theorem]{Corollary}
\theoremstyle{break}
\newtheorem{lemma}[theorem]{Lemma}

\begin{document}
\title{Parallelepiped Volume}
\author{Justin Gerber}
\date{\today}
\maketitle

We consider functions of the following sort:

$$
F: \underbrace{\mathbb{R}^n \times \ldots \times \mathbb{R}^n}_{n \text{ times}} \to \mathbb{R}
$$


\begin{definition}[Shear Property]
test
\end{definition}
Scale property:
\begin{align}
F(v_1, \ldots, \lambda v_i, \ldots, v_n) = \lambda F(v_1, \ldots, v_i, \ldots, v_n)
\end{align}

Shear property: for $i\neq j$:
\begin{align}
F(v_1, \ldots, v_i + v_j, \ldots, v_n) = F(v_1,\ldots, v_i, \ldots, v_n)
\end{align}

Show that these two imply multilinear:
\begin{align}
F(v_1, \ldots, \lambda v_i + w, \ldots, v_n) = \lambda F(v_1, \ldots, v_i, \ldots, v_n) + F(v_1, \ldots, w, \ldots, v_n)
\end{align}



$w$ can be decomposed as 



Alternating:
\begin{align}
F(v_1, \ldots, v_i, \ldots, v_j, \ldots, v_n) = - F(v_1, \ldots, v_j, \ldots, v_i, \ldots, v_n)
\end{align}

\end{document}


