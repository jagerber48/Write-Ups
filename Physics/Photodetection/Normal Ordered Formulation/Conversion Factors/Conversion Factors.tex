\documentclass[12pt]{article}
\usepackage{amssymb, amsmath, amsfonts}

\usepackage[utf8]{inputenc}
\bibliographystyle{plain}
\usepackage{subfigure}%ngerman
\usepackage[pdftex]{graphicx}
\usepackage{textcomp} 
\usepackage{xcolor}
\usepackage[hidelinks]{hyperref}
\usepackage{anysize}
\usepackage{siunitx}
\usepackage{verbatim}
\usepackage{float}
\usepackage{braket}
\usepackage{xfrac}
\usepackage{booktabs}

\newcommand{\ep}{\epsilon}
\newcommand{\sinc}{\text{sinc}}
\newcommand{\bv}[1]{\boldsymbol{#1}}
\newcommand{\highlightr}[1]{%
  \colorbox{red!50}{$\displaystyle#1$}}
\newcommand{\highlightb}[1]{%
  \colorbox{blue!50}{$\displaystyle#1$}}

\begin{document}
\title{Signal Conversion Factors}
\author{Justin Gerber}
\date{\today}
\maketitle

\section{Introduction}
In this document I will work through some simple conversion factors involved in the E3 detection chain. This is mostly a conversion from some of my notes into Jonathan's notation in the matched filters paper. 

\section{Quadrature Operators}

In much of my work I take the convention that

\begin{align}
X_{JG} &= a^{\dag} + a = 2\text{Re}(a)\\
P_{JG} &= i(a^{\dag} - a) = 2\text{Im}(a)\\
a &= \frac{1}{2}\left(X_{JG} + i P_{JG}\right)\\
a^{\dag} &= \frac{1}{2}\left(X_{JG} - i P_{JG}\right)
\end{align}

In the matched filters paper Jonathan takes the unitary convention

\begin{align}
X_{JK} &= \frac{1}{\sqrt{2}}\left(a^{\dag} + a\right) = \sqrt{2}\text{Re}(a)\\
P_{JK} &= \frac{i}{\sqrt{2}}\left(a^{\dag} - a\right) = \sqrt{2}\text{Im}(a)\\
a &= \frac{1}{\sqrt{2}}\left(X_{JK} + i P_{JK}\right)\\
a^{\dag} &= \frac{1}{\sqrt{2}}\left(X_{JK} - i P_{JK}\right)
\end{align}

So we see that

\begin{align}
X_{JG} &= \sqrt{2} X_{JK}\\
P_{JG} &= \sqrt{2} P_{JK}
\end{align}

In this document I will drop the $JK$ subscript and adopt the unitary convention. I will use the $JG$ subscript when necessary.

\section{Detected Signal}
In my Kelley-Kleiner write up I derive the following formulas about the measured photocurrent when a probe field ($S$) is measured in a balanced heterodyne setup by beating with a local oscillator.

\begin{align}
\Braket{i(t)} = &e |\alpha| \epsilon_Q \sqrt{\epsilon_{MM}} \Braket{:X_{S, JG}^{\Delta_{LO}t + \phi_{LO}}(t):}\\
\Braket{i(t_1)i(t_2)} = &e^2|\alpha|^2 \epsilon_Q^2 \epsilon_{MM} \Braket{:X_{S, JG}^{\Delta_{LO}t_1+\phi_{LO}}(t_1)X_{S, JG}^{\Delta_{LO}t_2 + \phi_{LO}}(t_2):}\\
+ &e^2|\alpha|^2\epsilon_Q \delta(t_2-t_1)
\end{align}

Here $i(t)$ is the classical measured photocurrent expressed as a random variable. $e$ is the electron charge. $\alpha$ is the complex amplitude of the local oscillator (LO) field, that is $|\alpha|^2=n_{LO}$, the photon number flux of the LO field, related to the optical power in the LO beam. $\epsilon_Q$ is the quantum efficiency of the detector. $\epsilon_{MM}$ is the mode-matching efficiency at the detector between the LO and probe. $X_{S, JG}^{\Delta_{LO} t + \phi_{LO}}(t)$ is a rotating quadrature operator defined shortly. $::$ denotes normal and time ordering.

I define rotated quadrature operators as

\begin{align}
X^{\theta}_{JG} &= a^{\dag}e^{-i\theta} + a e^{i\theta}\\
P^{\theta}_{JG} &= i(a^{\dag}e^{-i\theta} - a e^{i \theta}) = X^{\theta-\frac{\pi}{2}}_{JG}
\end{align}

So one can see that

\begin{align}
X^0_{JG} &= X_{JG}\\
P^0_{JG} &= P_{JG}
\end{align}

Let's consider a homodyne measurement of the $P$ quadrature so that $\Delta_{LO} = 0$ and $\phi_{LO} = -\frac{\pi}{2}$. We then get

\begin{align}
\braket{i(t)} = &e |\alpha|\epsilon_Q \sqrt{\epsilon_{MM}}\Braket{:P_{S,JG}(t):}\\
\braket{i(t_1)i(t_2)} = &e^2|\alpha|^2\epsilon_Q^2\epsilon_{MM}\Braket{:P_{S,JG}(t_1)P_{S,JG}(t_2):} + e^2|\alpha|^2\epsilon_Q\delta(t_2-t_1)
\end{align}

We rescale this into the $JK$ convention:

\begin{align}
\braket{i(t)} = &e |\alpha|\epsilon_Q  \sqrt{\epsilon_{MM}}\sqrt{2}\Braket{:P_{S}(t):}\\
\braket{i(t_1)i(t_2)} = &e^2|\alpha|^2\epsilon_Q^2\epsilon_{MM}2\Braket{:P_{S}(t_1)P_{S}(t_2):} + e^2|\alpha|^2\epsilon_Q\delta(t_2-t_1)
\end{align}

\section{Two-time quadrature operator}

I show below that we can expand

\begin{align}
:P_{S,JG}(t_1)P_{S,JG}(t_2): &= \\
&= \frac{1}{2}\left\{P_{S,JG}(t_1), P_{S,JG}(t_2)\right\} + \frac{i}{2}\left[X_{S,JG}(t_{min}), P_{S,JG}(t_{max})\right]\\
&= \frac{1}{2}\left\{P_{S,JG}(t_1), P_{S,JG}(t_2)\right\} - \delta(t_2-t_1)\\
&= P_{JG}(t_1)P_{S,JG}(t_2) - \delta(t_2-t_1)
\end{align}

Rescaling this into the $JK$ convention we find

\begin{align}
:P(t_1)P(t_2): &= \\
&= \frac{1}{2}\left\{P_S(t_1), P_S(t_2)\right\} + \frac{i}{2}\left[X_S(t_{min}), P_S(t_{max})\right]\\
&= \frac{1}{2}\left\{P_S(t_1), P_S(t_2)\right\} - \frac{1}{2}\delta(t_2-t_1)\\
&= P_S(t_1)P_S(t_2) - \frac{1}{2}\delta(t_2-t_1)
\end{align}

The second expression will turn out to be the one that best matches the matched filter paper results. Plugging this into the expression for the two-time correlation function we get

\begin{align}
\Braket{i(t_1)i(t_2)} =& e^2 |\alpha|^2 \epsilon_Q^2 \epsilon_{MM} \Braket{\left\{P_S(t_1), P_S(t_2)\right\}}\\
&- e^2|\alpha|^2\epsilon_Q^2\epsilon_{MM}\delta(t_2-t_1) + e^2|\alpha|^2\epsilon_Q\delta(t_2-t_1)\\
=& e^2 |\alpha|^2 \epsilon_Q^2 \epsilon_{MM} \Braket{\left\{P_S(t_1), P_S(t_2)\right\}} + e^2|\alpha|^2 \epsilon_Q(1-\epsilon_Q\epsilon_{MM})\delta(t_2-t_1)
\end{align}



\section{Adiabatic approximation}

\begin{align}
H = -\hbar \Delta c^{\dag}c + \sum_i \hbar \omega_i a^{\dag}_i a_i + \sum_i \hbar \sqrt{\bar{n}}g_i(c^{\dag}+c)(a^{\dag}_i + a_i)
\end{align}

Setting $\Delta=0$

\begin{align}
H = \sum_i \hbar \omega_i a^{\dag}_i a_i + \sum_i \hbar \sqrt{\bar{n}}g_i(c^{\dag}+c)(a^{\dag}_i + a_i)
\end{align}

Equation of motion:

\begin{align}
\dot{c} &= -\frac{i}{\hbar}\sum_i \hbar \sqrt{\bar{n}}g_i(a^{\dag}_i + a_i)\\
&= -i \sqrt{2\bar{n}}\sum_i g_i X_i
\end{align}

Adding in noise drives

\begin{align}
\dot{c} &= -i \sqrt{2\bar{n}}\sum_i g_i X_i - \kappa c - \sqrt{2\kappa\epsilon_C} c_{G,in} - \sqrt{2\kappa\epsilon_B} c_{B, in}
\end{align}

Adiabatic elimination:

\begin{align}
c = -\frac{i \sqrt{2 \bar{n}}}{\kappa}\sum_i g_iX_i - \sqrt{\frac{2\epsilon_C}{\kappa}}c_{G,in} - \sqrt{\frac{2\epsilon_B}{\kappa}}c_{B,in}
\end{align}

Convert to $P_{cav}$ using $P_{cav} = \frac{i}{\sqrt{2}}(c^{\dag}-c)$

\begin{align}
P_{cav} &= \frac{i}{\sqrt{2}}\left(\frac{2i\sqrt{2\bar{n}}}{\kappa}\sum_i g_i X_i - \sqrt{\frac{2\epsilon_C}{\kappa}}\left(c_{G,in}^{\dag} - c_{G,in}\right) - \sqrt{\frac{2\epsilon_B}{\kappa}}\left(c_{B,in}^{\dag} - c_{B,in}\right)\right)\\
&= -\frac{2\sqrt{\bar{n}}}{\kappa}\sum_i g_i X_i - \sqrt{\frac{2\epsilon_C}{\kappa}}P_{G, in} - \sqrt{\frac{2\epsilon_B}{\kappa}}P_{B, in}
\end{align}

\section{Detected field in terms of cavity field}

The field leaking out of the cavity can be related to the intracavity field through the input output relations.

\begin{align}
P_{out} = \sqrt{2\kappa\epsilon_C}P_{cav} + P_{G, in}
\end{align}

$\kappa$ is the cavity half-linewidth and $\epsilon_C$ is the cavity detection efficiency, that is, the fraction of photons which leave the cavity that actually go towards the detector. $P_{in}$ is the white noise correlated input bath field.

After exiting the cavity the light passes through an optical path which has possible losses. The final detected "signal" field is given by

\begin{align}
P_S = \sqrt{\epsilon_P}P_{out} + \sqrt{1-\epsilon_P}P_{path}
\end{align}

Where $P_{path}$ is another white noise bath and $\epsilon_P$ is the path detection efficiency. We thus get

\begin{align}
P_S = \sqrt{\epsilon_C\epsilon_P}\sqrt{2\kappa}P_{cav} + \sqrt{\epsilon_P}P_{G, in} + \sqrt{1-\epsilon_P}P_{path}
\end{align}

We then must calculate

\begin{align}
\Braket{\left\{P_S(t_1), P_S(t_2)\right\}}
\end{align}

$P_{path}$ is uncorrelated with the previous two terms so there are no cross terms. Thus we are left with

\begin{align}
\Braket{\left\{P_S(t_1), P_S(t_2)\right\}} =& \epsilon_C \epsilon_P 2\kappa \Braket{\left\{P_{cav}(t_1), P_{cav}(t_2)\right\}} \\
&+ \epsilon_P\sqrt{\epsilon_C}\sqrt{2\kappa}\left(\Braket{\left\{P_{cav}(t_1), P_{G,in}(t_2)\right\}} + \Braket{\left\{P_{G,in}(t_1), P_{cav}(t_2)\right\}}\right)\\
&+ \epsilon_P \Braket{\left\{P_{G,in}(t_1), P_{G,in}(t_2)\right\}} + (1-\epsilon_P)\Braket{\left\{P_{path}(t_1), P_{path}(t_2)\right\}}
\end{align}

Note that for the noise drives we have $\Braket{P_N(t_1)P_N(t_2)} = \frac{1}{2}\delta(t_2-t_1)$ and $\Braket{\left\{P_N(t_1)P_N(t_2)\right\}} = \delta(t_2-t_1)$. This gives us

\begin{align}
\Braket{\left\{P_S(t_1), P_S(t_2)\right\}} =& \epsilon_C \epsilon_P 2\kappa \Braket{\left\{P_{cav}(t_1), P_{cav}(t_2)\right\}} \\
&+ \epsilon_P\sqrt{\epsilon_C}\sqrt{2\kappa}\left(\Braket{\left\{P_{cav}(t_1), P_{G,in}(t_2)\right\}} + \Braket{\left\{P_{G,in}(t_1), P_{cav}(t_2)\right\}}\right)\\
&+\delta(t_2-t_1)
\end{align}

Let's calculate

\begin{align}
\Braket{P_{cav}(t_1)P_{G,in}(t_2)} = \Braket{-\left(\frac{2\sqrt{\bar{n}}}{\kappa} \sum_i g_i X_i(t_1) + \sqrt{\frac{2\epsilon_C}{\kappa}} P_{G,in}(t_1) + \sqrt{\frac{2\epsilon_B}{\kappa}}P_{B,in}(t_1) \right)P_{G,in}(t_2)}
\end{align}

Only the central term survives leaving

\begin{align}
\Braket{P_{cav}(t_1)P_{G,in}(t_2)} = -\sqrt{\frac{2\ep_C}{\kappa}} \frac{1}{2}\delta(t_2-t_1)
\end{align}

Using this in the above expression we get

\begin{align}
\Braket{\left\{P_S(t_1), P_S(t_2)\right\}} =& \ep_C \ep_P 2\kappa \Braket{\left\{P_{cav}(t_1), P_{cav}(t_2)\right\}}\\
&- 4\ep_P \ep_C\delta(t_2-t_1) +  \delta(t_2-t_1)
\end{align}

Finally looking at

\begin{align}
\Braket{\left\{P_{cav}(t_1),P_{cav}(t_2)\right\}} &= \frac{4\bar{n}}{\kappa^2}\sum_{ij}g_ig_j \Braket{\left\{X_i(t_1),X_j(t_2)\right\}} + \frac{2\ep_C}{\kappa} \delta(t_2-t_1) + \frac{2\ep_B}{\kappa} \delta(t_2-t_1)\\
&= \frac{4\bar{n}}{\kappa} \sum_{ij}g_ig_j\Braket{\left\{X_i(t_1), X_j(t_2)\right\}} + \frac{2}{\kappa}\delta(t_2-t_1)
\end{align}

Plugging this in we find

\begin{align}
\Braket{\left\{P_S(t_1),P_S(t_2)\right\}} &= \ep_C\ep_P \frac{8\bar{n}}{\kappa}\sum_{ij} g_ig_j\Braket{\left\{X_i(t_1),X_j(t_2)\right\}} + \delta(t_2-t_1)
\end{align}

\section{Putting it together}

\begin{align}
\Braket{i(t_1)i(t_2)} =& e^2 |\alpha|^2 \epsilon_Q^2 \epsilon_{MM} \Braket{\left\{P_S(t_1), P_S(t_2)\right\}} + e^2|\alpha|^2 \epsilon_Q(1-\epsilon_Q\epsilon_{MM})\delta(t_2-t_1)
\end{align}

\begin{align}
\Braket{\left\{P_S(t_1),P_S(t_2)\right\}} &= \ep_C\ep_P \frac{8\bar{n}}{\kappa}\sum_{ij} g_ig_j\Braket{\left\{X_i(t_1),X_j(t_2)\right\}} + \delta(t_2-t_1)
\end{align}

So we get that

\begin{align}
\Braket{i(t_1)i(t_2)} =& e^2|\alpha|^2\ep_Q^2 \ep_{MM} \ep_C\ep_P \frac{8\bar{n}}{\kappa}\sum_{ij}g_ig_j\Braket{\left\{X_i(t_1),X_j(t_2)\right\}}\\
&+ e^2|\alpha|^2\ep_Q^2 \ep_{MM}\delta(t_2-t_1) + e^2|\alpha|^2\ep_Q(1-\ep_Q\ep_{MM})\delta(t_2-t_1)\\
&= e^2|\alpha|^2\ep_Q\left(\ep_Q\ep_{MM}\ep_C\ep_P \frac{8\bar{n}}{\kappa}\sum_{ij}g_ig_j\Braket{\left\{X_i(t_1),X_j(t_2)\right\}} + \delta(t_2-t_1)\right)
\end{align}

We define $\epsilon = \ep_Q\ep_{MM}\ep_C\ep_P$.

\begin{align}
\Braket{i(t_1)i(t_2)} = e^2|\alpha|^2\ep_Q\left(\frac{8\bar{n}\ep}{\kappa}\sum_{ij}g_ig_j \Braket{\left\{X_i(t_1),X_j(t_2)\right\}} + \delta(t_2-t_1) \right)
\end{align}

We also have that

\begin{align}
\Braket{P_S(t)} &= \sqrt{\ep_C\ep_P}\sqrt{2\kappa}\Braket{P_{cav}(t)} = -\sqrt{\ep_C\ep_P}\sqrt{2\kappa}\frac{2\sqrt{\bar{n}}}{\kappa}\sum_i g_i \Braket{X_i(t)}\\
&=-\sqrt{\frac{\ep_C\ep_P 8 \bar{n}}{\kappa}}\sum_i g_i\Braket{X_i(t)}
\end{align}

Recalling

\begin{align}
\braket{i(t)} = &e |\alpha|\epsilon_Q  \sqrt{\epsilon_{MM}}\sqrt{2}\Braket{P_{S}(t)}\\
\end{align}

we find

\begin{align}
\Braket{i(t)} = -e|\alpha|\sqrt{\ep_Q} \sqrt{\frac{8\bar{n}\ep}{\kappa}} \sum_i \sqrt{2}g_i\Braket{X_i(t)}
\end{align}

Define

\begin{align}
S(t) = \frac{1}{e|\alpha|\sqrt{\ep_Q}} \sqrt{\frac{\kappa}{8\bar{n}\ep}} i(t) = \frac{1}{e|\alpha|\sqrt{\ep_Q}} \sqrt{P_{SN}}
\end{align}

We then have

\begin{align}
\Braket{S(t)} &= -\sum_i \sqrt{2}g_i\Braket{X_i(t)}\\
\Braket{S(t_1)S(t_2)} &= \sum_{ij}g_ig_j \Braket{\left\{X_i(t_1), X_j(t_2)\right\}} + P_{SN}\delta(t_2-t_1)
\end{align}

\section{Calculation}

In this section I work in the JG quadrature operator convention.

\begin{align}
X^{\phi} &= a^{\dag}e^{-i\phi} + a e^{i\phi}\\
P^{\phi} &= i(a^{\dag}e^{-i\phi} - a e^{i\phi})\\
a^{\phi} &= ae^{i\phi} = \frac{1}{2}(X^{\phi} + iP^{\phi})\\
a^{\dag \phi} &= a^{\dag}e^{-i\phi} = \frac{1}{2}(X^{\phi}-iP^{\phi})
\end{align}

\begin{align}
[X, Y] &= XY - YX\\
XY &= YX + [X, Y]\\
YX &= XY - [X, Y]\\
\{X, Y\} &= XY + YX\\
XY &= \frac{1}{2}\{X, Y\} + \frac{1}{2}[X, Y]
\end{align}

\subsection{Commutator Anti-Commutator Expansion}
\subsubsection{$t_1>t_2$}
\begin{align}
:X^{\phi_1}(t_1)X^{\phi_2}(t_2): =&\\
=& :(a^{\dag}(t_1)e^{-i\phi_1} + a(t_1)e^{i\phi_1})(a^{\dag}(t_2)e^{-i\phi_2} + a(t_2)e^{i\phi_2}):\\
=& :\highlightr{a^{\dag}(t_1)a^{\dag}(t_2)}e^{-i\phi_1}e^{-i\phi_2} + a^{\dag}(t_1)a(t_2)e^{-i\phi_1}e^{i\phi_2}\\
&+ \highlightr{a(t_1)a^{\dag}(t_2)}e^{i\phi_1}e^{-i\phi_2} + a(t_1)a(t_2)e^{i\phi_1}e^{i\phi_2}:\\
=& \highlightb{a^{\dag}(t_2)a^{\dag}(t_1)}e^{-i\phi_1}e^{-i\phi_2} + a^{\dag}(t_1)a(t_2)e^{-i\phi_1}e^{i\phi_2}\\
&+ \highlightb{a^{\dag}(t_2)a(t_1)}e^{i\phi_1}e^{-i\phi_2} + a(t_1)a(t_2)e^{i\phi_1}e^{i\phi_2}\\
=& \highlightr{a^{\dag}(t_1)a^{\dag}(t_2)}e^{-i\phi_1}e^{-i\phi_2} + a^{\dag}(t_1)a(t_2)e^{-i\phi_1}e^{i\phi_2}\\
&+ \highlightr{a(t_1)a^{\dag}(t_2)}e^{i\phi_1}e^{-i\phi_2} + a(t_1)a(t_2)e^{i\phi_1}e^{i\phi_2}\\
&+ [a^{\dag}(t_2), a^{\dag}(t_1)]e^{-i\phi_1}e^{-i\phi_2} + [a^{\dag}(t_2), a(t_1)]e^{i\phi_1}e^{-i\phi_2}\\
=& (a^{\dag}(t_1)e^{-i\phi_1} + a(t_1)e^{i\phi_1})(a^{\dag}(t_2)e^{-i\phi_2} + a(t_2)e^{i\phi_2})\\
&+ [a^{\dag}(t_2)e^{-i\phi_2}, a^{\dag}(t_1)e^{-i\phi_1} + a(t_1)e^{i\phi_1}]\\
=& X^{\phi_1}(t_1)X^{\phi_2}(t_2) + \frac{1}{2}\left[X^{\phi_2}(t_2) - iP^{\phi_2}(t_2), X^{\phi_1}(t_1)\right]\\
=& \frac{1}{2}\left\{X^{\phi_1}(t_1), X^{\phi_2}(t_2)\right\} + \frac{1}{2}\left[X^{\phi_1}(t_1), X^{\phi_2}(t_2)\right]\\
&+\frac{1}{2}\left[X^{\phi_2}(t_2), X^{\phi_1}(t_1)\right] - \frac{i}{2}\left[P^{\phi_2}(t_2), X^{\phi_1}(t_1)\right]\\
&= \frac{1}{2}\left\{X^{\phi_1}(t_1), X^{\phi_2}(t_2)\right\} - \frac{i}{2}\left[P^{\phi_2}(t_2), X^{\phi_1}(t_1)\right]\\
&= \frac{1}{2}\left\{X^{\phi_1}(t_1), X^{\phi_2}(t_2)\right\} - \frac{i}{2}\left[P^{\phi_{min}}(t_{min}), X^{\phi_{max}}(t_{max})\right]\\
\end{align}

\subsubsection{$t_2>t_1$}

\begin{align}
:X^{\phi_1}(t_1)X^{\phi_2}(t_2): =&\\
=& :(a^{\dag}(t_1)e^{-i\phi_1} + a(t_1)e^{i\phi_1})(a^{\dag}(t_2)e^{-i\phi_2} + a(t_2)e^{i\phi_2}):\\
=& :a^{\dag}(t_1)a^{\dag}(t_2)e^{-i\phi_1}e^{-i\phi_2} + a^{\dag}(t_1)a(t_2)e^{-i\phi_1}e^{i\phi_2}\\
&+ \highlightr{a(t_1)a^{\dag}(t_2)}e^{i\phi_1}e^{-i\phi_2} + \highlightr{a(t_1)a(t_2)}e^{i\phi_1}e^{i\phi_2}:\\
=& a^{\dag}(t_1)a^{\dag}(t_2)e^{-i\phi_1}e^{-i\phi_2} + a^{\dag}(t_1)a(t_2)e^{-i\phi_1}e^{i\phi_2}\\
&+ \highlightb{a^{\dag}(t_2)a(t_1)}e^{i\phi_1}e^{-i\phi_2} + \highlightb{a(t_2)a(t_1)}e^{i\phi_1}e^{i\phi_2}\\
=& a^{\dag}(t_1)a^{\dag}(t_2)e^{-i\phi_1}e^{-i\phi_2} + a^{\dag}(t_1)a(t_2)e^{-i\phi_1}e^{i\phi_2}\\
&+ \highlightr{a(t_1)a^{\dag}(t_2)}e^{i\phi_1}e^{-i\phi_2} + \highlightr{a(t_1)a(t_2)}e^{i\phi_1}e^{i\phi_2}\\
&+ [a^{\dag}(t_2), a(t_1)]e^{i\phi_1}e^{-i\phi_2} + [a(t_2), a(t_1)]e^{i\phi_1}e^{i\phi_2}\\
=& (a^{\dag}(t_1)e^{-i\phi_1} + a(t_1)e^{i\phi_1})(a^{\dag}(t_2)e^{-i\phi_2} + a(t_2)e^{i\phi_2})\\
&+ [a^{\dag}(t_2)e^{-i\phi_2} + a(t_2)e^{i\phi_2}, a(t_1)e^{i\phi_1}]\\
=& X^{\phi_1}(t_1)X^{\phi_2}(t_2) + \frac{1}{2}\left[X^{\phi_2}(t_2), X^{\phi_1}(t_1) + i P^{\phi_1}(t_1)\right]\\
=& \frac{1}{2}\left\{X^{\phi_1}(t_1), X^{\phi_2}(t_2)\right\} + \frac{1}{2}\left[X^{\phi_1}(t_1), X^{\phi_2}(t_2)\right]\\
&+ \frac{1}{2}\left[X^{\phi_2}(t_2), X^{\phi_1}(t_1)\right] + \frac{i}{2}\left[X^{\phi_2}(t_2), P^{\phi_1}(t_1)\right]\\
=& \frac{1}{2}\left\{X^{\phi_1}(t_1), X^{\phi_2}(t_2)\right\} - \frac{i}{2}\left[P^{\phi_1}(t_1), X^{\phi_2}(t_2)\right]\\
=& \frac{1}{2}\left\{X^{\phi_1}(t_1), X^{\phi_2}(t_2)\right\} - \frac{i}{2}\left[P^{\phi_{min}}(t_{min}), X^{\phi_{max}}(t_{max})\right]\\
\end{align}

\subsubsection{Summary}

To summarize:
\begin{align}
:X^{\phi_1}(t_1)X^{\phi_2}(t_2): &= \frac{1}{2}\left\{X^{\phi_1}(t_1), X^{\phi_2}(t_2)\right\} - \frac{i}{2}\left[P^{\phi_{min}}(t_{min}), X^{\phi_{max}}(t_{max})\right]\\
\end{align}

\subsection{Evaluate commutator expressions}

We now plug in the known commutation relations

\begin{align}
[a(t_1), a(t_2)] &= [a^{\dag}(t_1), a^{\dag}(t_2)] = 0\\
[a(t_1), a^{\dag}(t_2)] &= \delta(t_2-t_1)
\end{align}

From this we can derive

\begin{align}
[X^{\phi_1}(t_1), X^{\phi_2}(t_2)] &= [a^{\dag}(t_1)e^{-i\phi_1} + a(t_1)e^{i\phi_1}, a^{\dag}(t_2)e^{-i\phi_2} + a(t_2)e^{i\phi_2}]\\
&= (-e^{-i\phi_1}e^{i\phi_2} + e^{i\phi_1}e^{-i\phi_2})\delta(t_2-t_1)\\
&= 2i\sin(\phi_1-\phi_2) \delta(t_2-t_1)
\end{align}

\begin{align}
[X^{\phi_1}(t_1), P^{\phi_2}(t_2)] &= [a^{\dag}(t_1)e^{-i\phi_1} + a(t_1)e^{i\phi_1}, i(a^{\dag}(t_2)e^{-i\phi_2} - a(t_2)e^{i\phi_2})]\\
&=i(e^{-i\phi_1}e^{i\phi_2} + e^{i\phi_1}e^{-i\phi_2})\delta(t_2-t_1)\\
&=2i\cos(\phi_1-\phi_2)\delta(t_2-t_1) 
\end{align}

\subsection{Re-expression}

Reconsider 

\begin{align}
:X^{\phi_1}(t_1)X^{\phi_2}(t_2): &= \frac{1}{2}\left\{X^{\phi_1}(t_1), X^{\phi_2}(t_2)\right\} - \frac{i}{2}\left[P^{\phi_{min}}(t_{min}), X^{\phi_{max}}(t_{max})\right]\\
\end{align}

\subsubsection{First re-expression}
We rewrite the above as

\begin{align}
:X^{\phi_1}(t_1)X^{\phi_2}(t_2): &= \frac{1}{2}\left\{X^{\phi_1}(t_1), X^{\phi_2}(t_2)\right\} + \frac{i}{2}\left[X^{\phi_{max}}(t_{max}), P^{\phi_{min}}(t_{min})\right]\\
\end{align}

and quickly notice this can be written as

\begin{align}
:X^{\phi_1}(t_1)X^{\phi_2}(t_2): &= \frac{1}{2}\left\{X^{\phi_1}(t_1), X^{\phi_2}(t_2)\right\} - \cos(\phi_{max} - \phi_{min})\delta(t_2-t_1)\\
&= \frac{1}{2}\left\{X^{\phi_1}(t_1), X^{\phi_2}(t_2)\right\} - \cos(\phi_1 - \phi_2)\delta(t_2-t_1)\\
\end{align}

I used the evenness of the cosine function here.

\subsubsection{Second re-expression}
Note that we have

\begin{align}
\frac{1}{2}\{X, Y\} = XY - \frac{1}{2}[X,Y]
\end{align}

So this can be written as

\begin{align}
&:X^{\phi_1}(t_1)X^{\phi_2}(t_2): =\\
&= X^{\phi_1}(t_1)X^{\phi_2}(t_2)
-\frac{1}{2}\left[X^{\phi_1}(t_1), X^{\phi_2}(t_2)\right] - \cos(\phi_{1} - \phi_{2})\delta(t_2-t_1)\\
&= X^{\phi_1}(t_1)X^{\phi_2}(t_2) - i\sin(\phi_1 - \phi_2)\delta(t_2-t_1) - \cos(\phi_{1} - \phi_{2})\delta(t_2-t_1)\\
&= X^{\phi_1}(t_1)X^{\phi_2}(t_2) - e^{i\phi_1}e^{-i\phi_2}\delta(t_2-t_1)
\end{align}

\subsection{Summary}

We have found

\begin{align}
:X^{\phi_1}(t_1)X^{\phi_2}(t_2): &=\\ 
&= \frac{1}{2}\left\{X^{\phi_1}(t_1), X^{\phi_2}(t_2)\right\} - \frac{i}{2}\left[P^{\phi_{min}}(t_{min}), X^{\phi_{max}}(t_{max})\right]\\
&= \frac{1}{2}\left\{X^{\phi_1}(t_1), X^{\phi_2}(t_2)\right\} - \cos(\phi_{max} - \phi_{min})\delta(t_2-t_1)\\
&= X^{\phi_1}(t_1)X^{\phi_2}(t_2) - e^{i\phi_1}e^{-i\phi_2}\delta(t_2-t_1)
\end{align}

\subsection{Specialization}

We specialize to the case that $\phi_1=\phi_2=-\frac{\pi}{2}$ so we get

\begin{align}
:P(t_1)P(t_2): &= \\
&= \frac{1}{2}\left\{P(t_1), P(t_2)\right\} + \frac{i}{2}\left[X(t_{min}), P(t_{max})\right]\\
&= \frac{1}{2}\left\{P(t_1), P(t_2)\right\} - \delta(t_2-t_1)\\
&= P(t_1)P(t_2) - \delta(t_2-t_1)
\end{align}

\end{document}