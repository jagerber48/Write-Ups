\documentclass[12pt]{article}
\usepackage{amssymb, amsmath, amsfonts}

\usepackage{biblatex}
\usepackage{bbm}
\usepackage[hidelinks]{hyperref}
\usepackage{braket}
\usepackage[colorinlistoftodos,prependcaption,textsize=tiny]{todonotes}

\newcommand{\ddt}[1]{\frac{d #1}{dt}}
\newcommand{\ppt}[1]{\frac{\partial #1}{\partial t}}
\newcommand{\ep}{\epsilon}
\newcommand{\sinc}{\text{sinc}}
\newcommand{\bv}[1]{\boldsymbol{#1}}
\newcommand{\ahat}{\hat{a}}
\newcommand{\adag}{\ahat^{\dag}}
\newcommand{\braketacomm}[1]{\left\langle\left\{#1\right\} \right\rangle}
\newcommand{\braketcomm}[1]{\left\langle\left[#1\right] \right\rangle}
\newcommand{\ketbra}[2]{\Ket{#1}\!\Bra{#2}}

\bibliography{refs}

\begin{document}
\title{Gaussian Beam}
\author{Justin Gerber}
\date{\today}
\maketitle

\section{Introduction}
In this document I write down the formulas for higher order Gaussian beams. I follow and trust Wikipedia.

\section{Derivation of the Gaussian Beam}

\subsection{Maxwell's Equations}
Maxwell's equations in vacuum are written as

\begin{align}
\nabla \cdot \bv{E} =& 0\\
\nabla \cdot \bv{B} =& 0\\
\nabla \times \bv{E} =& - \ppt{\bv{B}}\\
\nabla \times \bv{B} =& \frac{1}{c^2} \ppt{\bv{E}}
\end{align}

\subsection{Deriving the Helmholtz Equation}
By taking the curl of the third equation (and interchanging the temporal and spatial derivatives) we get

\begin{align}
\nabla \times \left( \nabla \times \bv{E}\right) = -\ppt{} \left(\nabla \times \bv{B}\right) = -\frac{1}{c^2} \frac{\partial^2}{\partial t^2} \bv{E}
\end{align}

Using a vector calculus identity:

\begin{align}
\nabla\left(\nabla \cdot \bv{E}\right) - \nabla^2 \bv{E} = -\nabla^2\bv{E} = -\frac{1}{c^2} \frac{\partial^2}{\partial t^2} \bv{E}
\end{align}

We have applied the first of Maxwell's equations.
Rearranging:

\begin{align}
\left(\nabla^2 - \frac{1}{c^2}\frac{\partial^2}{\partial t^2}\right)\bv{E} = 0
\end{align}

This is the vector wave equation.
We implement the ansatz $\bv{E} = E_0\bv{u}e^{+i\omega t}$ with $\bv{u}$ time-independent so that

\begin{align}
\left(\nabla^2 - \frac{(+i\omega)^2}{c^2}\right) = \left(\nabla^2 + k^2\right)\bv{u} = 0
\end{align}

This is the vector Helmholtz equation.
Note that $\bv{u}$ is now a dimensionless mode shape function.
It is more difficult to solve the vector Helmholtz equation than the scalar Helmholtz equation, however, it is possible for one to bootstrap from a solution of the scalar Helmholtz equation to a solution to the vector Helmholtz equation.

If $u$ satisfies the scalar Helmholtz equation:

\begin{align}
\left(\nabla^2 + k^2\right)u = 0
\end{align}

Then

\begin{align}
&\bv{r}\times \nabla u\\
&\nabla \times (\bv{r}\times \nabla u)
\end{align}

Are two oppositely polarized solutions to the vector Helmholtz equation. See \cite{Reitz1960} for more on this.

For this reason we will devote our attention to finding solutions to the scalar Helmholtz equation.

\subsection{The Paraxial Helmholtz Equation}

In fact, we will restrict even more to the paraxial Helmholtz equation.
In the paraxial Helmholtz equation we are considering `beams' of light.
These are patterns of the electromagnetic field which are extended in one dimension, $z$ and confined in the transverse direction.
We will see that as long as the confinement in the transverse direction is not too large (the relevant length scale is $\lambda \sim \frac{1}{k}$) then the variation in the axial direction is moderate.
This will allow us to approximate the Helmholtz equation with the paraxial Helmholtz equation as follows.

\begin{align}
\left(\nabla^2 + k^2\right)u = \left(\partial_x^2 +  \partial_y^2 + \partial_z^2 + k^2\right)u = 0
\end{align}

We now make the ansatz

\begin{align}
u = a e^{-ikz}
\end{align}

The idea is that $a$ is a slowly spatially varying function. 
We then have

\begin{align}
\partial_z^2 \left(a e^{-ikz}\right) =& \partial_z\left((\partial_z a) e^{-ikz} - ik a e^{-ikz}\right)\\
=& (\partial_z^2 a) e^{-ikz} - ik (\partial_z a) e^{-ikz} - ik(\partial_z a)e^{ikz} - k^2 a e^{-ikz}\\
=& \left(\partial_z^2 a - 2ik \partial_z a - k^2 a\right)e^{-ikz}
\end{align}

The paraxial approximation is to suppose $\partial_z^2 A \ll k^2 A$ so that we end up with

\begin{align}
\partial_z^2\left(ae^{-ikz}\right) = \left(-2ik\partial_z a - k^2 a\right)e^{-ikz}
\end{align}

Plugging this into the full scalar wave equation we get

\begin{align}
\left(\nabla^2 + k^2\right)a e^{-ikz} \approx \left((\partial_x^2 + \partial_y^2)a - 2ik \partial_z a\right)e^{-ikz} = 0
\end{align}

We can write this as

\begin{align}
\label{eq:Parax}
\left(\nabla_T^2 - 2ik \partial_z\right)a = 0
\end{align}

This is the paraxial Helmholtz equation where $\nabla^2_T = \partial_x^2 + \partial_y^2$.

\subsection{Solving the Paraxial Helmholtz Equation in Fourier Space}

We can find a solution to this equation by performing a 2D Fourier transform.
We write $a(x,y,z) = a_z(x,y)$, so that $a_x$ represents the complex field in a plane with fixed $z$.
We perform a 2D Fourier transform in this plane.
Here we use the Fourier convention:

\begin{align}
\mathcal{FT}[f(x)](k_x) = \tilde{f}(k_x) = \int e^{-ik_x x} f(x) dx
\end{align}

Under this convention the 2D-Fourier transform of the paraxial Helmholtz equation yields

\begin{align}
\left((ik_x)^2 + (ik_y)^2 - 2ik \partial_z\right)\tilde{a}_z(k_x,k_y) =& 0\\
= (-k_x^2 - k_y^2 - 2ik\partial_z)\tilde{a}_z(k_x,k_y) =& 0
\end{align}

Here

\begin{align}
\tilde{a}_z(x,y) = \int e^{-ik_x x}e^{-ik_y y} a_z(x,y) dx dy
\end{align}

This equation can be rewritten as

\begin{align}
\label{eq:ParaxHelmFourier}
\partial_z \tilde{a}_z(k_x, k_y) = i \frac{1}{2k}(k_x^2 + k_y^2) \tilde{a}_z(k_x, k_y)
\end{align}

This differential equation is easily solved by

\begin{align}
\label{eq:ParabFourier}
\tilde{a}_z(k_x, k_y) = e^{i\frac{z}{2k}(k_x^2 + k_y^2)} = e^{i\frac{\lambda z}{4\pi} (k_x^2 + k_y^2)}
\end{align}

The inverse Fourier transform of this is given by

\begin{align}
a_z(x, y) = \frac{i}{\lambda z} e^{-i\frac{\pi}{\lambda z}(x^2 + y^2)}
\end{align}

This function is the so called parabaloidal wave and it is a classic solution to the paraxial Helmholtz equation.
It is tied closely with Huygens' principle but we will not go into that here.

\subsection{From Parabaloidal to Gaussian Wave - The Complex Beam Parameter}

We can anticipate a problem with the parabaloidal wave. In particular, if we look in Fourier space at Eq. (\ref{eq:ParabFourier}) we can see that the parabaloidal wave has support at all spatial frequencies.
This means that it will have infinite power as it is an integral over an infinite number of plane waves with constant amplitude.

We can solve this problem of infinite power by returning to the Fourier representation:

consider

\begin{align}
\tilde{a}_z(k_x, k_y) = e^{i \frac{\lambda z}{4\pi}(k_x^2 + k_y^2)}
\end{align}

If we let $z\rightarrow q(z) = (z-z_0) + i z_R = \Delta z + i z_R$ we get

\begin{align}
\tilde{a}_z(k_x, k_y) = e^{i \frac{\lambda q(z)}{4\pi}(k_x^2 + k_y^2)} = e^{i \frac{\lambda (-z_0 + i z_R)}{4\pi}(k_x^2 + k_y^2)} e^{i\frac{\lambda z}{4\pi}(k_x^2 + k_y^2)}
\end{align}

We see that this new function is a multiplication of the previous Fourier space solution by a function which does not depend on $z$.
This means that this new function will also be a solution of the Fourier space version of the paraxial Helmholtz equation, Eq. (\ref{eq:ParaxHelmFourier}).

We will call $q(z)$ the complex beam parameter.

The advantage of the introduction of $q(z)$ is two-fold.
First, the introduction of $z_0$ will cause the resultant wave to be centered about $z=z_0$ rather than $z=0$. 
Second, the main advantage, is that the $iz_R$ term acts to introduce a decaying Gaussian envelope in Fourier space to the Fourier representation of the wave.
This  limits the total power or energy in the beam to a finite value.
We will give a more physical intuition for $z_R$ shortly.

We take the inverse Fourier transform again to find

\begin{align}
a_z(x,y) = \frac{i}{\lambda q(z)} e^{-i\frac{\pi}{\lambda q(z)}(x^2 + y^2)} = \frac{i}{\lambda q(z)} e^{-i\frac{\pi}{\lambda q(z)} \rho^2 }
\end{align}

This it turns out will be the basic Gaussian beam common in optics.
I've defined $\rho^2 = x^2 + y^2$.

\subsection{Deriving the Familiar Gaussian Beam}
We are now going to perform a series of manipulations and defintions to bring this into the familiar form for a Gaussian beam.
We will first go through with little motivation or intuition and then, once the final result is established, we will loop back and explore the definitions more carefully.

First we find

\begin{align}
\frac{1}{q(z)} = \frac{1}{\Delta z + i z_R} =& \frac{\Delta z}{\Delta z^2 + z_R^2} - i \frac{z_R}{\Delta z^2 + z_R^2}\\
=& \frac{1}{\Delta z\left(1 + \left(\frac{z_R}{\Delta z}\right)^2\right)} - i \frac{1}{z_R\left(1 + \left(\frac{\Delta z}{z_R}\right)^2\right)}
\end{align}

We plug this in to the exponent above (we will manipulate the coefficient in front shortly):

\begin{align}
a_{q(z)}(x,y) = \frac{i}{\lambda q(z)} e^{-i\frac{\pi}{\lambda} \frac{\rho^2}{\Delta z\left(1+\left(\frac{z_R}{\Delta z}\right)^2\right)}} e^{-\frac{\pi}{\lambda}\frac{\rho^2}{z_R\left(1+\left(\frac{\Delta z}{z_R}\right)^2\right)}}
\end{align}

In this form we then define the beam waist as a function of $z$

\begin{align}
w(z)^2 =& \frac{\lambda z_R}{\pi}\left(1+\left(\frac{\Delta z}{z_R}\right)^2\right)\\
=& w_0^2 \left(1+\left(\frac{\Delta z}{z_R}\right)^2\right)^2
\end{align}

and the wavefront radius of curvature:

\begin{align}
R(z) = \Delta z\left(1+\left(\frac{z_R}{\Delta z}\right)^2\right)
\end{align}

We then rewrite

\begin{align}
a_{q(z)}(x,y) = \frac{i}{\lambda q(z)} e^{-i k \frac{\rho^2}{2R(z)}} e^{- \frac{\rho^2}{w(z)^2}}
\end{align}

We now re-express

\begin{align}
\frac{i}{q(z)} = \frac{i}{\Delta z + iz_R}
\end{align}

We consider the reciprocal:

\begin{align}
\frac{q(z)}{i} =& z_R - i\Delta z= \sqrt{z_R^2 + \Delta z^2} e^{i\arctan\left( - \frac{\Delta z}{z_R}\right)}\\
=& z_R \sqrt{1 + \left(\frac{\Delta z}{z_R}\right)^2} e^{-i\arctan\left(\frac{\Delta z}{z_R}\right)}
\end{align}

We define the Gouy phase

\begin{align}
\psi(z) = \arctan\left(\frac{\Delta z}{z_R}\right)
\end{align}

We can then write

\begin{align}
\frac{i}{\lambda q(z)} =& \frac{1}{\lambda z_R \sqrt{1+\left(\frac{\Delta z}{z_R}\right)^2}} e^{i\psi(z)}\\
=& \frac{1}{\pi} \frac{1}{w(z)} e^{i\psi(z)}
\end{align}

We plug this in to find a more familiar Gaussian beam expression:

\begin{align}
a_{q(z)} = \frac{1}{\pi} \frac{1}{w(z)}e^{-\frac{\rho^2}{w(z)^2}}e^{-ik\frac{\rho^2}{2R(z)}}e^{i\psi(z)}
\end{align}

We recall that $a_{q(z)}$ was given by

\begin{align}
E = E_0 u e^{+i\omega t} = E_0 a e^{-ikz}e^{+i\omega t}
\end{align}

So that the full (scalar) electric field for a Gaussian beam is given by

\begin{align}
E(z, \rho, t) = E_0 \frac{1}{\pi} \frac{1}{w(z)} e^{-\frac{\rho^2}{w(z)^2}}e^{-ik\frac{\rho^2}{2R(z)}}e^{i\psi(z)} e^{-ikz} e^{+i\omega t}
\end{align}

Note that an arbitrary rescaling doesn't matter so we can multiply by $\pi \omega_0$ to get

\begin{align}
E(z, \rho, t) = E_0 \frac{w_0}{w(z)} e^{-\frac{\rho^2}{w(z)^2}}e^{-ik\frac{\rho^2}{2R(z)}}e^{i\psi(z)} e^{-ikz} e^{+i\omega t}
\end{align}

This scaling has the advantage that $E(0, 0, 0) = E_0$.

We also have that

\begin{align}
|E(z, \rho, t)|^2 = |E_0|^2 \left(\frac{w_0}{w(z)}\right)^2 e^{-2 \frac{\rho^2}{w(z)^2}}
\end{align}

\subsection{Exploring the Terms in the Gaussian Beam}

\subsubsection{Radial Dependence}\todo{add plot}
We first explore the waist of the Gaussian beam.

\begin{align}
w(z) = w_0 \sqrt{1+\left(\frac{\Delta z}{z}\right)^2}
\end{align}

First we see that is called the waist because in a plane of constant $z$ the field intensity goes like

\begin{align}
e^{-2 \frac{\rho^2}{w(z)^2}}
\end{align}

Thus we see that points in space with the same $\rho$ have the same intensity, the pattern is radially symmetric.
Next we see that when $\rho = w(z)$ that the intensity falls to $e^{-2} = \frac{1}{e^2}$ of its original value.
$w(z)$ is called the beam waist at position $z$.
$w_0$ is called \textit{the} beam waist and is a clear radial length scale for the beam.

A word of caution. 
In mathematical and statistics literature on encounters the Gaussian function:

\begin{align}
G(x) = \frac{1}{\sigma \sqrt{2\pi}} e^{-\frac{1}{2} \left(\frac{x}{\sigma}\right)^2}
\end{align}

Here $\sigma$ represents the standard deviation of the Gaussian and $\sigma^2$ represents the variance. 
We can see that the Gaussian beam above is in fact a Gaussian function, but we must be wary and note that $w(z)$ is NOT the variance of the Gaussian. In particular one would have

\begin{align}
\frac{2}{w^2} =& \frac{1}{2\sigma^2}\\
w =& 2\sigma\\
\sigma =& \frac{w}{2}
\end{align}

We now explore the profile for $\Delta z \gg z_R$ and the relationship between $w_0$ and $z_R$.

\begin{align}
w(z) \approx \frac{w_0}{z_R} \Delta z
\end{align}

That is the waist grows linearly as a function of propagation distance.
If we imagine a 2D profile in space of $\rho = w(z)$ this profile would be conical with a half-angle $\theta$ with

\begin{align}
\tan(\theta) = \frac{\omega_0}{z_R}
\end{align}

If $\theta \ll 1$ then

\begin{align}
\theta \approx \frac{\omega_0}{z_R}
\end{align}

We write a few relationships between $w_0$, $z_R$, and $\theta$ involving $\lambda$ and $k$.

\begin{align}
\omega_0 =& \sqrt{\frac{\lambda z_R}{\pi}} = \sqrt{2 \frac{z_R}{k}}\\
z_R =& \frac{\pi w_0^2}{\lambda} = \frac{1}{2} k w_0^2\\
\tan(\theta) =& \frac{\omega_0}{z_R} = \sqrt{\frac{\lambda}{\pi z_R}} = \sqrt{\frac{2}{k z_R}} = \frac{\lambda}{\pi w_0} = \frac{2}{k\omega_0}
\end{align}

We see that $\theta \ll 1$ if $\omega_0 \gg \lambda$ or $z_R \gg \lambda$, one of which holds if the other does.

\subsubsection{Axial Dependence}\todo{add plot}

We next explore the axial dependence of the beam.
Along the line of $\rho=0$ the beam intensity goes like

\begin{align}
\left(\frac{w_0}{w(z)}\right)^2 = \frac{1}{1 + \left(\frac{\Delta z}{z_R}\right)^2}
\end{align}

We see that the intensity is highest when $\Delta z$ and that when $\Delta z=z_R$ the intensity falls off to $\frac{1}{2}$ it's value at the waist.
$z_R$ is thus an axial length scale for the beam.

\subsubsection{Wavefront Radii of Curvature}

Now we will explore why $R(z)$ is called the wavefront radius of curvature.
First a reminder about radius of curvature.

Recall from calculus a ciruclar curve

\begin{align}
y^2 + x^2 =& R^2\\
y =& \sqrt{R^2 - x^2}
\end{align}

We calculate the 2nd derivative

\begin{align}
\frac{dy}{dx} =& -\frac{x}{\sqrt{R^2 - x^2}}\\
\frac{d^2y}{dx^2} =& \frac{1}{\sqrt{R^2 - x^2}} + \frac{x^2}{(R^2 - x^2)^{\frac{3}{2}}}
\end{align}

If we evaluate this at $x=0$ we get

\begin{align}
y'' = \frac{1}{R}
\end{align}

In particular, what I would like to draw from this is that for a curve which has a vanishing derivative but non-vanishing second derivative can be approximated by a circle with a radius of curvature $R = \frac{1}{y''}$.

We are now going to apply this to the 2D surfaces representing the surfaces of constant phase (the wavefronts) in the Gaussian beam.
The relevant terms are

\begin{align}
e^{-ik \frac{\rho^2}{2R(z)}} e^{-ikz} e^{i\psi(z)}
\end{align}

\begin{align}
R(z) = \Delta z\left(1+\left(\frac{z_R}{\Delta z}\right)^2\right)
\end{align}

\begin{align}
\psi(z) = \arctan\left(\frac{\Delta z}{z_R}\right)
\end{align}

We note that $R(z)$ and $\psi(z)$ vary slowly as a functions of $z$ so for a wavefront centered at some value of $z$ we will approximate $R(z) = R$ and $\psi(z) = \psi$. We then see that we have

\begin{align}
e^{-ik\frac{\rho^2}{2R} - ikz + i\psi}
\end{align}

Surfaces of constant phase are then defined by

\begin{align}
\frac{\rho^2}{2R} + z = \phi
\end{align}

with $\phi$ constant

\begin{align}
z =& \phi - \frac{\rho^2}{2R}\\
z' =& \frac{\rho}{R}\\
z'' = & \frac{1}{R}
\end{align}

So we see that the curvature of this surface is given exactly by $R$.
This justifies us calling $R(z)$ the radius of curvature.

Note that for $z = 0$ the radius of curvature is infinite, that is the phase fronts are flat and for $z\rightarrow \infty$ that $R(z)\rightarrow z$.
The maximum radius of curvature occurs when $z=z_R$ and $R(z) = 2 z_R$.

\subsubsection{Guoy Phase}

Much or little could be said about the Guoy phase.
Here we simply state the Guoy phase and indicate that it is an additional phase which is collected by a Gaussian beam compared to a plane wave of the same wavelength.
It will have important implication for the resonant frequencies of higher order spatial modes in optical cavities.

\begin{align}
\psi(z) = \arctan\left(\frac{\Delta z}{z_R}\right)
\end{align}

\subsection{Summary of the Gaussian Beam}

The field of Gaussian beam is given by

\begin{align}
E(z, \rho, t) =& E_0 \frac{w_0}{w(z)} e^{-\frac{\rho^2}{w(z)^2}} e^{-ik\frac{\rho^2}{2R(z)}} e^{i\psi(z)} e^{-ikz} e^{+i\omega t}\\
w(z) =& w_0\sqrt{1+\left(\frac{\Delta z}{z_R}\right)^2}\\
R(z) =& z\left(1+\left(\frac{z_R}{z}\right)^2\right)\\
\psi(z) =& \arctan\left(\frac{\Delta z}{z}\right)
\end{align}

The intensity is given by

\begin{align}
|E(z, \rho, t)|^2 = |E_0|^2 \left(\frac{w_0}{w(z)}\right)^2 e^{-2\frac{\rho^2}{w(z)^2}}
\end{align}

Some relationships between $w_0$ and $z_R$ and $\theta$:

\begin{align}
\omega_0 =& \sqrt{\frac{\lambda z_R}{\pi}} = \sqrt{2 \frac{z_R}{k}}\\
z_R =& \frac{\pi w_0^2}{\lambda} = \frac{1}{2} k w_0^2\\
\tan(\theta) =& \frac{\omega_0}{z_R} = \sqrt{\frac{\lambda}{\pi z_R}} = \sqrt{\frac{2}{k z_R}} = \frac{\lambda}{\pi w_0} = \frac{2}{k\omega_0}
\end{align}

The Gaussian beam can alternatively be expressed as 

\begin{align}
E(z, \rho, t) = E_0 \frac{q_0}{q(z)} e^{-i k \frac{\rho^2}{2q(z)}} e^{-ikz} e^{+i\omega t}
\end{align}

With

\begin{align}
q(z) = (z-z_0) + i z_R = \Delta + i z_R
\end{align}

with the relations that

\begin{align}
\frac{1}{q(z)} =& \frac{1}{R(z)} - i \frac{\lambda}{\pi w(z)^2}\\
\frac{i}{\lambda q(z)} =& \frac{1}{\pi} \frac{1}{w(z)} e^{i\psi(z)}
\end{align}

\section{Higher Order Gaussian Beams}

Above we have calculated one solution to the paraxial wave equation.
However, it is general possible, through separation of variables, to solve for entire families of solutions such that any solution can be decomposed into modes within these families.

\cite{} and \cite{}\todo{add references} give some detail on how to solve the paraxial Helmholtz equation in Cartesian coordinates to derive the Hermite-Gaussian family of modes.
The separation of variables is, however, a bit complicated.
Unfortunately I have not found a thoroughly simple and pedagogical walkthrough of the derivation and have not been able to create my own derivation.
For that reason I will simply state the final results here.
The forms I give are close to those given in all references including Wikipedia, though may differ slightly from all of them.

When the separation of variables is performed in a cylindrical rather than Cartesian coordinate system the resultant family of modes are the Laguerre-Gaussian modes.
There is even less material in the literature on the derivation of these modes.

\cite{}\todo{add references} provides a unique approach to deriving both the standard Hermite-Gaussian and Laguerre-Gaussian modes as well as the so-called `elegant' Hermite-Gaussian and Laguerre-Gaussian modes.


\subsection{Hermite-Gaussian Modes}

The Hermite Gaussian modes are given by

\begin{align}
u_{nm}(x,y,z) = &\frac{w_0}{w(z)} e^{-\frac{x^2+y^2}{w(z)^2}} e^{-i\frac{x^2+y^2}{2R(z)}} e^{-ikz} e^{i\psi(z)}\\
\times& C_{nm}H_n\left(\frac{\sqrt{2}x}{w(z)}\right)H_m\left(\frac{\sqrt{2}y}{w(z)}\right)e^{i(n+m)\psi(z)}
\end{align}

The first line above is the exact same as the standard Gaussian beam.
The second line is a modulation by the Hermite polynomials $H_n$ and $H_m$.
$C_{nm}$ is a normalization constant which we will calculate imminently.
We see also that these higher order modes come with extra Guoy  phase factors which scale with the mode numbers.

We calculate the normalization factor by demanding that the modes or orthogonal in the $z=0$ plane.

\begin{align}
\int\int u_{nm}(x,y,0)u_{n'm'}^*(x,y,0) dx dy = \delta_{nn'}\delta_{mm'}
\end{align}

We work this out by first writing down

\begin{align}
u_{nm}(x, y, 0) = C_{nm}e^{-\frac{x^2}{w_0^2}} H_n\left(\frac{\sqrt{2}x}{w_0}\right) e^{-\frac{y^2}{w_0^2}}H_m\left(\frac{\sqrt{2}y}{w_0}\right)
\end{align}

The double integration then separates nicely:

\begin{align}
&\int\int u_{nm}(x,y,0)u_{n'm'}^*(x,y,0) dx dy =\\
= C_{nm}C_{n'm'} &\left[\int e^{-2\frac{x^2}{w_0^2}}H_n\left(\frac{\sqrt{2}x}{w_0}\right) H_{n'}\left(\frac{\sqrt{2}x}{w_0}\right) dx \right]\\
\times &\left[\int e^{-2\frac{y^2}{w_0^2}}H_m\left(\frac{\sqrt{2}y}{w_0}\right) H_{m'}\left(\frac{\sqrt{2}y}{w_0}\right) dy \right]
\end{align}

We implement a change of variables $\frac{\sqrt{2}x}{w_0} = \nu$ and similarly for $y$ to get

\begin{align}
=& C_{nm}C_{n'm'}  \frac{w_0^2}{2} \int e^{-\nu^2}H_n(\nu)H_{n'}(\nu)d\nu \int e^{-\mu^2}H_m(\mu)H_{m'}(\mu)d\mu=1
\end{align}

The normalization for Hermite polynomials is given by

\begin{align}
\int e^{-\nu^2}H_n(\nu)H_{n'}(\nu)d\nu = \sqrt{\pi}2^n n! \delta_{nn'}
\end{align}

so we see

\begin{align}
C_{nm}C_{n'm'} \frac{w_0^2}{2} (\sqrt{\pi}2^n n!)(\sqrt{\pi}2^m m!) \delta_{nn'}\delta_{mm'} = 1
\end{align}

So we see that we get the orthogonality as desired and for $n=n'$, $m=m'$ we have

\begin{align}
C_{nm} = \left(\frac{\sqrt{\frac{2}{\pi}}}{2^n n! w_0}\right)^{\frac{1}{2}}
\left(\frac{\sqrt{\frac{2}{\pi}}}{2^m m! w_0}\right)^{\frac{1}{2}}
\end{align}

The total orthonormal solution is then given by

\begin{align}
u_{nm}(x,y,z) = &\frac{w_0}{w(z)} e^{-\frac{x^2+y^2}{w(z)^2}} e^{-i\frac{x^2+y^2}{2R(z)}}e^{-ikz}e^{i\psi(z)}\\
\times & \left(\frac{\sqrt{\frac{2}{\pi}}}{2^n n! w_0}\right)^{\frac{1}{2}} H_n\left(\frac{\sqrt{2}x}{w(z)}\right) e^{in\psi(z)}\\
\times & \left(\frac{\sqrt{\frac{2}{\pi}}}{2^m m! w_0}\right)^{\frac{1}{2}} H_m\left(\frac{\sqrt{2}y}{w(z)}\right) e^{im\psi(z)}\\
\end{align}

\section{Laguerre-Gaussian Modes}

The Laguerre-Gaussian mode function are given by

\begin{align}
u_{lm}(\rho, \phi, z) =& \frac{w_0}{w(z)} e^{-\frac{\rho^2}{w(z)^2}} e^{-ik\frac{\rho^2}{R(z)}} e^{-ikz} e^{i\psi(z)}\\
& \times C_{lm} \left(\frac{\sqrt{2}\rho}{w(z)}\right)^{|m|} L_l^{|m|}\left(\frac{2\rho^2}{w(z)^2}\right) e^{-im\phi} e^{i(|m|+2l)\psi(z)}
\end{align}

Again the first line is the standard Gaussian solution and the second is the modulation representing the higher order modes.
$C_{lm}$ is again an orthonormalization constant which we will solve by by ensuring the mode function ar orthonormal at $z=0$.
$L_l^{|m|}$ are the generalized Laguerre polynomials.

Solving at $z=0$:

\begin{align}
u_{lm}(\rho, \phi, 0) = C_{lm} e^{-\frac{\rho^2}{w_0^2}}\left(\frac{\sqrt{2}\rho}{w_0}\right)^{|m|}L_l^{|m|}\left(\frac{2\rho^2}{w_0^2}\right)e^{-im\phi}
\end{align}

We now must calculate

\begin{align}
\int_{\phi=0}^{2\pi} \int_{\rho=0}^{\infty} u_{lm}(\rho, \phi, 0) u_{l'm'}^*(\rho,\phi,0) \rho d\rho d\phi
\end{align}

Working this out we get

\begin{align}
C_{lm}C_{l'm'}\int_{\phi=0}^{2\pi} e^{-i(m-m')\phi} d\phi \int_{\rho=0}^{\infty} e^{-2\frac{\rho^2}{w_0^2}} \left(\frac{\sqrt{2}\rho}{w_0}\right)^{|m|+|m'|} L_l^{|m|}\left(\frac{2\rho^2}{w_0^2}\right)L_{l'}^{|m'|}\left(\frac{2\rho^2}{w_0^2}\right) \rho d\rho
\end{align}

The $\phi$ integral is equal to $2\pi$ if $m=m'$ and zero otherwise so this reduces to

\begin{align}
C_{lm}C_{l'm}2\pi \int_{\rho=0}^{\infty} e^{-2\frac{\rho^2}{w_0^2}}\left(\frac{\sqrt{2}\rho}{w_0}\right)^{2|m|} L_l^{|m|}\left(\frac{2\rho^2}{w_0^2}\right)L_{l'}^{|m|}\left(\frac{2\rho^2}{w_0^2}\right) \rho d\rho
\end{align}

We perform a change of variables $t = \frac{2\rho^2}{w_0^2}$ with $dt = \frac{4}{w_0^2} \rho d\rho$.
This leads to

\begin{align}
= C_{lm}C_{l'm} 2\pi \frac{w_0^2}{4} \int_{t=0}^{\infty} e^{-t} t^{|m|} L_l^{|m|}(t) L_{l'}^{|m|}(t) dt
\end{align}

This integral works out to

\begin{align}
\int_{t=0}^{\infty} e^{-t} t^{|m|} L_l^{|m|}(t) L_{l'}^{|m|}(t) dt = \frac{(l+|m|)!}{l!} \delta_{ll'}
\end{align}

So we get

\begin{align}
C_{lm}^2 \frac{w_0^2}{2} \frac{(l+|m|)!}{l!} =& 1\\
C_{lm} = \sqrt{\frac{2l!}{(l+|m|)! w_0^2}}
\end{align}

Putting this all together we get

\begin{align}
u_{lm}(\rho, \phi, z) =& \frac{w_0}{w(z)} e^{-\frac{\rho^2}{w(z)^2}} e^{-ik\frac{\rho^2}{R(z)}} e^{-ikz} e^{i\psi(z)}\\
& \times \sqrt{\frac{2l!}{\pi(l+|m|)!}}\frac{1}{w_0} \left(\frac{\sqrt{2}\rho}{w(z)}\right)^{|m|} L_l^{|m|}\left(\frac{2\rho^2}{w(z)^2}\right) e^{-im\phi} e^{i(|m|+2l)\psi(z)}
\end{align}

\section{Appendix: Curl of Curl identity}

consider the curl of the curl of vector field $\bv{A}$.

\begin{align}
\nabla \times \nabla \times \bv{A} = \nabla \times \left(\nabla \times \bv{A}\right)
\end{align}

We write the $i^{\text{th}}$ component of this vector field in index notation:

\begin{align}
\left(\nabla \times \nabla \times \bv{A}\right)_i =& \ep_{ijk} \partial_j \ep_{klm} \partial_l A_m\\
=& \ep_{ijk}\ep_{klm} \partial_j \partial_l A_m\\
=& \ep_{kij} \ep_{klm} \partial_j \partial_l A_m\\
=& \left( \delta_{il}\delta_{jm} - \delta_{im}\delta_{jl}\right)\partial_j \partial_l A_m\\
=& \partial_m \partial_i A_m - \partial_l \partial_l A_i\\
=& \partial_i \partial_l A_l - \partial_l \partial_l A_i\\
=& \left(\nabla \left(\nabla \cdot \bv{A}\right) - (\nabla\cdot \nabla) \bv{A}\right)_i\\
=&  \left(\nabla \left(\nabla \cdot \bv{A}\right) - \nabla^2 \bv{A}\right)_i\\
\end{align}

So we say that

\begin{align}
\nabla \times \nabla \times \bv{A} = \nabla\left(\nabla \cdot \bv{A}\right) - \nabla^2\bv{A}
\end{align}

\end{document}