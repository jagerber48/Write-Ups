\documentclass[12pt]{article}
\usepackage{amssymb, amsmath, amsfonts}

\usepackage[utf8]{inputenc}
\bibliographystyle{plain}
\usepackage{subfigure}%ngerman
\usepackage[pdftex]{graphicx}
\usepackage{textcomp} 
\usepackage{color}
\usepackage[hidelinks]{hyperref}
\usepackage{anysize}
\usepackage{siunitx}
\usepackage{verbatim}
\usepackage{float}
\usepackage{braket}
\usepackage{xfrac}
\usepackage{booktabs}

\newcommand{\bv}[1]{\mathbf{#1}}
\newcommand{\ep}{\epsilon}
\newcommand{\sinc}{\text{sinc}}

\begin{document}
\title{Diffraction Limit}
\author{Justin Gerber}
\date{\today}
\maketitle

\section{Introduction}

I've been puzzled by the diffraction limit since I was an undergrad working on near-field microscopy. I think I've finally worked out all of my confusions so I'll report my findings here. It should be noted that this is almost a direct transcription of a few sections from Saleh and Teich ``Fundamentals of Photonics'' but it will be nice for me to work it out and re-write it in my own words.

The key is to treat the propagation of light through free space and optical components as a linear system. That is, consider light of angular frequency $\omega$ with wavevector $k = \frac{\omega}{c} = \frac{2\pi}{\lambda}$. An optical field (neglecting polarization) can be described by its amplitude at all locations in space, $U(x,y,z)$. Now consider the amplitude on a single plane in space, for example $z=0$. We consider the amplitude of the field on this plane, $U(x,y,0) = f(x,y)$ to be the input to the linear system. Suppose the field has a non-zero component in the $+z$ direction, that is $k_z > 0$. We can then ask the question what will the field be at a given distance $d$ from the first plane. That is, what is $U(x,y,d) = g(x,y)$ contingent upon the intervening media and components. $g(x,y)$ will be the output of the linear system. The task then is to find the transfer functions and, equivalently, the impulse responses of the different possible systems.

Note that I'll unapologetically make the paraxial, Fresnel, and Fraunhofer approximations repeatedly. Admittedly I don't think these approximations are valid for high $NA$ lenses (such as $NA > 0.7$ that we are considering for E6) but I figure I should at least understand the linear, paraxial version of the diffraction limit first before trying to tackle high $NA$. I believe the differences that arise in the high $NA$ case can be treated as aberrations relative to the low $NA$ case. The job of a good high $NA$ objective is to then correct for these and other resultant aberrations. 

\section{Dielectric Slabs and Thin Lenses}
\subsection{Flat Slab}

Consider a dielectric slab of thickness $d$ and index of refraction $n$. Consider an incident beam at angle $\theta$ to the optical axis. This incident beam can be described as a plane wave with wavevector $\bv{k}_0 = k_x \hat{\bv{x}} + k_y \hat{\bv{y}} + k_z \hat{\bv{z}}$ with $|\bv{k}_0| = k_0$. Suppose $k_y=0$ so that $k_x = k \sin(\theta)$ and $k_z = k \cos(\theta)$.

Inside the dielectric slab the beam will travel and angle $\theta_1$ according to Snell's law, $\sin(\theta) = n\sin(\theta_1)$ with wavevector magnitude $k_1 = nk_0$. The field at the other end $d$ can be described by

\begin{align}
U(x,y,d) = U(x,y,0) e^{-i \bv{k} \cdot \bv{d}} = U(x,y,0) \exp[-in k_0(d\cos(\theta_1)+x\sin(\theta_1))]
\end{align}

We'll consider $U(x,y,z)$ to be dimensionless. There will be coefficients describing amplitudes of certain waves that use the symbol $U$ with a subscript, these amplitudes will not necessarily be dimensionless depending on the context.
We can write down the transmittance function (note that here I talk about the function which takes a harmonic (plane wave) real-space input into a harmonic real-space output so it wouldn't be correct to use either the term transfer function or impulse response) 

\begin{align}
t(x,y) = \exp[-in k_0(d\cos(\theta_1)+x\sin(\theta_1))]
\end{align}

and Taylor expand in $\theta_1$ making the paraxial approximation. Note in the paraxial approximation $\theta_1 \approx \frac{\theta}{n} \ll 1$.

\begin{align}
(d\cos(\theta_1)+x\sin(\theta_1)) \approx d + \frac{d}{2}\theta_1^2 + x\theta_1
\end{align}

The contribution of both of these terms to the total phase should be very small. That is

\begin{align}
\frac{nk_0d \theta^2}{2n} \ll 2\pi && k_0 x \theta \ll 2\pi
\end{align}

The first constraint constrains the thickness of the mirror relative to the wavelength and numerical aperture of the input beam and the second constrains how far off-axis we can consider beams. If we neglect these terms then we have the transmittance function:

\begin{align}
t(x,y) = \exp[-in k_0d]
\end{align}

\subsection{Slab of Varying Thickness}
If the slab is of varying thickness then we treat consider the slab to have total thickness $d_0$ where $d_0$ is the maximum thickness at any point $(x,y)$. Then at each point the wave travels through a thickness $d(x,y)$ of dielectric and a thickness $d_0 - d(x,y)$ of air. Accordingly the transmittance function has two parts

\begin{align}
t(x,y) &= \exp[-in k_0d(x,y)]\exp[-i k_0 (d_0-d(x,y))]\\
t(x,y) &= \exp[-ik_0d_0] \exp[-i (n-1)k_0 d(x,y)]\\
t(x,y) &= t_0 \exp[-i (n-1)k_0 d(x,y)]
\end{align}

\begin{align}
t_0 = \exp[-ik_0d_0]
\end{align}
\subsection{Plano-Convex Thin Lens}
Consider a single plano-convex lens of total thickness $d_0$ and radius $R$. We work out the thickness at location $(x,y)$ (where $(0,0)$ is on the optical axis. Say the lens is convex on the left at touches $z=0$ and flat on the other side at $z=d_0$. A point $(x,y,z)$ on the surface of the curved surface satisfies

\begin{align}
x^2+y^2+(R-z)^2 = R^2
\end{align}

where the thickness along the line $(x,y)$ is given by

\begin{align}
d(x,y) = d_0 - z
\end{align}

Solving for $z$.

\begin{align}
z= R - \sqrt{R^2 - (x^2+y^2)} = R\left(1-\sqrt{1 - \frac{x^2+y^2}{R^2}}\right)
\end{align}

Taylor Expanding in $x,y \ll R$ amounts to the paraxial approximation.

\begin{align}
z \approx R \left(1 - (1 - \frac{x^2+y^2}{2R^2}\right) = \frac{x^2+y^2}{2R}
\end{align}

So we find

\begin{align}
d(x,y) \approx d_0 - \frac{x^2 + y^2}{2R}
\end{align}

The transmittance is then

\begin{align}
t(x,y) &= \exp[-ik_0d_0] \exp\left[-i(n-1)k_0\left(d_0 - \frac{x^2+y^2}{2R}\right)\right]\\
&= \exp[-i n k_0 d_0] \exp\left[k_0\frac{(n-1)}{2R} (x^2+y^2)\right]
\end{align}

\subsubsection{Aside on Spherical and Paraboloidal Waves}

Electromagnetic waves satisfy the Helmholtz equation

\begin{align}
(\nabla^2+|\bv{k}|^2)U(\bv{r}) = 0
\end{align}

Plane waves of the form $U_{\text{plane}}(\bv{r}) = \exp[-i \bv{k}\cdot\bv{r}]$ solve the Helmholtz equation (as can be verified using index notation, for example.) It is also true that spherical waves of the form

\begin{align}
U_{\text{sph}}(\bv{r}) = U_{0,\text{sph}}\frac{\exp[-i k r]}{r}
\end{align}

also satisfy the Helmholtz equation. Here $U_{0,\text{sph}}$ has dimensions of $[L]$. Here $k = |\bv{k}|$ and $r = |\bv{r}|$. We consider a spherical wave far away from it's source in $z$ but close to the $z$ axis. That is, $x,y \ll z$. note that $r = \sqrt{x^2+y^2+z^2} \approx z+\frac{x^2+y^2}{2z}$. This is a small angle paraxial approximation. We then approximate

\begin{align}
U_{\text{sph}}(\bv{r}) \approx U_{\text{para}}(\bv{r}) = \frac{U_{0,\text{sph}}}{z} \exp[-ikz] \exp\left[-ik\frac{x^2+y^2}{2z}\right]
\end{align}

We expand the phase to quadratic order but the amplitude to zeroth order since the waveshape is more sensitive to the phase. The equal phase surfaces of this function are parabolas so we consider a wave with this phase pattern to be a parabaloidal wave centered at the origin.

Noting all of this we return to transmittance function for a thin plano-convex lens.

\begin{align}
t(x,y) &= \exp[-i n k_0 d_0] \exp\left[k_0\frac{(n-1)}{2R} (x^2+y^2)\right]
\end{align}

We recognize this as the phase pattern for a paraboloidal wave centered on the origin with $z = -\frac{R}{n-1}$. Since $z$ is negative it means that for a wave with $k_z>0$ (i.e. a right traveling wave) the wave fronts are converging down onto the origin. That is, it is a wave focusing down on the origin with the origin being distance $\frac{R}{n-1}$ to the right of the current point. In simple terms: the lens focuses an input plane wave to the point at $\frac{R}{n-1}$. We thus call this distances the focal length of the lens,

\begin{align}
f = \frac{R}{n-1}
\end{align}

so

\begin{align}
t(x,y) &= \exp[-i n k_0 d_0] \exp\left[ik_0\frac{x^2+y^2}{2f}\right] = t_n \exp\left[ik_0\frac{x^2+y^2}{2f}\right]\\
&= t_n \exp\left[i\pi\frac{x^2+y^2}{\lambda f}\right]
\end{align}

\begin{align}
t_n = \exp[-i n k_0 d_0]
\end{align}

\subsection{Double Convex Lens}
Now we consider a double convex lens with first radii $R_1$ and second $R_2$. Note that the typical sign convention for the radii of curvature in such an optical system is that $R_1$ would be positive for a convex first surface and $R_2$ would be negative for a convex second surface. We need to determine the thickness of the lens. We can think of it as two plano-convex lenses with total thickness $d_0' = \frac{d_0}{2}$ and different radii of curvature. The thickness will then be 

\begin{align}
d(x,y) = d_0' - \frac{x^2+y^2}{2|R_1|} + d_0' - \frac{x^2+y^2}{2|R_2|}
\end{align}

Plugging this in as before for the transmittance we find

\begin{align}
t(x,y) &= \exp[-i n k_0 d_0] \exp\left[k_0 \frac{(n-1)}{2}\left(\frac{1}{|R_1|}+\frac{1}{|R_2|}\right) (x^2+y^2)\right]
\end{align}

This time we identify 

\begin{align}
\frac{1}{f} = (n-1) \left(\frac{1}{|R_1|} + \frac{1}{|R_2|}\right)
\end{align}

I've used absolute values so I don't have to worry about the sign convention. The correct convention is that $R_1$ is positive and $R_2$ is negative so

\begin{align}
\frac{1}{f} = (n-1) \left(\frac{1}{R_1} - \frac{1}{R_2}\right)
\end{align}

We again find the lens turns a plane wave into a paraboloidal wave focused on the focal point.

\begin{align}
t(x,y) &= \exp[-i n k_0 d_0] \exp\left[ik_0\frac{x^2+y^2}{2f}\right] = t_n \exp\left[ik_0\frac{x^2+y^2}{2f}\right]\\
&= t_n \exp\left[i \pi \frac{x^2+y^2}{\lambda f}\right]
\end{align}

\section{Fourier Optics}
Now we shift into Fourier Optics. In Fourier optics we consider $f(x,y) = U(x,y,0)$, the field amplitude on a given plane in space, to be the input to a linear system. The wave then propagates through free space and intervening optics to a new plane at $z=d$ and we can calculate the amplitude on this new plane, $g(x,y) = U(x,y,d)$, the output of the linear system. So the input in the field amplitude on one plane and the output is the field amplitude on the other plane. The linear system is whatever free space or optical elements intervene the space between the two planes. The goal is then to find the transfer function, and equivalently, the point spread functions for various optical elements. That is we are interested in what happens if the input is some pure spectral function, or if it is an impulse (delta) function.

First we consider how $f(x,y)$ can be Fourier decomposed. We know, for a plane wave with wavevector $\bv{k}$, that

\begin{align}
U(x,y,z) = U_0 \exp[-i\bv{k}\cdot\bv{r}]
\end{align}

so that

\begin{align}
U(x,y,0) = U_0 \exp[-ik_z z]\exp[-i (k_x x + k_y y)]
\end{align}

$U_0$ is dimensionless here. We see that $k_x$ and $k_y$ indicate the spatial periodicity in $x$ and $y$. $k_x$ and $k_y$ are the equivalent of angular frequencies. We can define cyclic spatial frequencies $\nu_x = \frac{k_x}{2\pi}$ and $\nu_y = \frac{k_y}{2\pi}$ so that

\begin{align}
f(x,y) = U(x,y,0) = U_0 \exp[-ik_z z]\exp[-i 2\pi(\nu_x x + \nu_y y)]
\end{align}

It is clear then that for a plane wave with wavevector $\bv{k}$ that $f(x,y)$ is a function on the plane with spatial frequencies $\nu_x$ and $\nu_y$. Note that the plane wave travels in direction $\bv{k}$ which makes angles 
\begin{align}
\theta_x &= \arcsin\left(\frac{k_x}{k}\right) =  \arcsin(\nu_x \lambda) \approx \nu_x \lambda\\
\theta_y &= \arcsin\left(\frac{k_y}{k}\right) =  \arcsin(\nu_y \lambda) \approx \nu_y \lambda\\
\end{align}

Where the approximation is again a small angle paraxial approximation.

Let's have a comment on evanescent waves here since it is a bit important to the diffraction limit discussion. We have, for solutions to the Helmholtz equation, that

\begin{align}
k_x^2+k_y^2+k_z^2 = |\bv{k}|^2 = \frac{\omega}{c}
\end{align}

That is the components of the wavevector are constrained. For a plane wave all of the components of the wavevector are real which constrains $|k_x|,|k_y|,|k_z| \le |\bv{k}| = \frac{2\pi}{\lambda}$. Equivalently it constrains $|\nu_x|,|\nu_y| \le \frac{1}{\lambda}$. That is, in the case we are strictly dealing with plane waves it is impossible to have an input function $f(x,y)$ with spatial periodicity shorter than $\lambda$. Consider the following limits. In the case that the plane wave is parallel to the $z$ axis there is no periodicity in $x$ or $y$, the function is constant. However, as $\bv{k}$ is tilted upwards in say the $x$ direction we being to see phase lines on the $f(x,y)$ plane with low periodicity. As the angle becomes more extreme the lines get closer and closer together as $k_x\propto \nu_x$ increases. However, even in the limit that $\bv{k}$ is pointing very close or even in the $x$ direction (no more $z$ component), the spacing between the phase planes is never less than $\lambda$.

However, in my mind, two related things are apparent. 1) If we allow the components of $\bv{k}$ to be imaginary we still get something which is a solution to the Helmholtz equation and 2) it is possible to conceive of functions $f(x,y)$ which vary faster than $\lambda$ so what happens if one of those fields is the input field? For example what if we have a point radiator which is much smaller than $\lambda$. This answer is of course that these things must happen at the same time. We suppose $k_x$ and $k_y$ are real since they represent frequencies of the Fourier transform of $f(x,y)$ so if one or both of them is greater than $|\bv{k}|$ (high spatial frequency) it indicates that $k_z$ must be imaginary. Furthermore, if $k_z$ is imaginary it means that as we move along in the $z$ direction away from $z=0$ the field amplitude is exponentially decaying. That is 1) we don't have a normal plane wave anymore and 2) the local high resolution information rapidly damps away spatially.

The summary is that this formalism can handle input functions $f(x,y)$ with all spatial frequencies, but if those spatial frequencies are greater than $\frac{1}{\lambda}$ the fields involved will not be traveling waves but rather they will be decaying evanescent waves.

We formalize this idea in the next section.

\subsection{Transfer function of free space}

We consider an input field $f(x,y)$ with spatial frequencies $\nu_x$ and $\nu_y$ at plane $z=0$. We want to calculate the field amplitude distance $d$ away, $g(x,y)$. The ratio of the output to the input as a function of $\nu_x$ and $\nu_y$ will be transfer function of free space It is not too difficult

\begin{align}
f(x,y) = U(x,y,0) = U_0 \exp[-i 2\pi (\nu_x x + \nu_y y)]
\end{align}

and

\begin{align}
g(x,y) = U(x,y,d) = U_0 \exp[-i k_z d]\exp[-i 2\pi (\nu_x x + \nu_y y)]
\end{align}

We then see that

\begin{align}
H(\nu_x,\nu_y) = \frac{g(x,y)}{f(x,y)} = \exp[-i k_z d]
\end{align}

We recall that $k_z = \sqrt{k^2 - k_x^2 - k_z^2} = 2\pi \sqrt{\frac{1}{\lambda^2} - \nu_x^2 - \nu_y^2}$ so that 

\begin{align}
H(\nu_x,\nu_y) = \exp\left[-i 2\pi \left(\frac{1}{\lambda^2} - \nu_x^2 - \nu_y^2\right)^{\frac{1}{2}} d\right]
\end{align}

We see that for $\nu_x^2+\nu_y^2<\frac{1}{\lambda^2}$ the output of the square root is a real number so the magnitude of the transfer function is unity. That is, no energy is lost to free space. However, when $|\nu_x|$ or $|\nu_y|$ becomes larger than $\frac{1}{\lambda}$ the argument of the square root goes negative which means the result is imaginary and thus the exponent is negative and real. This means the transfer function is a decaying exponential with distance. In addition to the amplitude we see that the transfer function also gives some phase shift depending on the relative angle of the wavevector, however, in the evanescent case the phase of the transfer function is zero.

\subsection{Paraxial Approximation for transfer function}

We're going to approximate the transfer function in the regime $\nu_x, \nu_y \ll \frac{1}{\lambda}$ by Taylor expanding the square root term. The total angle between the wavevector and optical axis can be written and approximated at $\theta^2 = \theta_x^2 + \theta_y^2 = \lambda^2(\nu_x^2+\nu_y^2) \ll 1$ so that

\begin{align}
\left(\frac{1}{\lambda^2} - \nu_x^2 - \nu_y^2\right)^{\frac{1}{2}} = \frac{1}{\lambda} \left(1 - \theta^2\right)^{\frac{1}{2}} \approx \frac{1}{\lambda}\left(1-\frac{\theta^2}{2}\right)
\end{align}

We plug this in to get

\begin{align}
H(\nu_x,\nu_y) &= \exp\left[-i 2\pi \frac{d}{\lambda}\right] \exp\left[i 2 \pi \frac{d}{\lambda} \frac{\theta^2}{2}\right]\\
H(\nu_x,\nu_y) &= \exp\left[-ik d\right] \exp\left[ i \pi \lambda d (\nu_x^2+\nu_y^2)\right] = H_0 \exp\left[i \pi \lambda d (\nu_x^2+\nu_y^2)\right]
\end{align}

\begin{align}
H_0 = \exp[-ikd]
\end{align}

The term we have neglected in the exponent is

\begin{align}
2\pi \frac{d}{\lambda}\frac{\theta^4}{8} = \pi \frac{\theta^4 d}{4 \lambda}
\end{align}

For the Taylor expansions to be valid it is necessary that $\theta \ll 1$ so that $\theta^4 \ll \theta^2$. However, that condition is not strong enough for the validity of this approximation. For example, even if $\theta \ll 1$ it could be possible that $\frac{d}{\lambda} \gg 1$ in such a way that $\pi \frac{\theta^4 d}{\lambda} > \pi$ in which the phase contribution from this term would not be negligible. Therefore, the necessary condition for this approximation is 

\begin{align}
\frac{\theta^4 d}{4 \lambda} \ll 1
\end{align}

Suppose we constrain analysis to a circle of radius $a$ in the output plane so that there is a maximum deflection from axial, $\theta_m \approx \frac{a}{d}$. then we have

\begin{align}
\frac{\theta^4 d}{4 \lambda} < \frac{\theta_m^4 d}{4 \lambda} = \frac{a^2 d \theta_m^2}{4 d^2 \lambda} = \frac{a^2}{d\lambda} \frac{\theta_m^2}{4} \ll 1
\end{align}

The first combination of constants is called the Fresnel number for the system. $N_F = \frac{a^2}{d\lambda}$. The Fresnel condition, necessary for this approximation to the free space transfer function, is that

\begin{align}
\frac{N_F \theta_m^2}{4} \ll 1
\end{align}


We can of course apply the superposition principle with the transfer function here. Consider

\begin{align}
f(x,y) = \int \int F(\nu_x,\nu_y) \exp[-i 2\pi (\nu_x x +\nu_y y)] d\nu_x d\nu_y
\end{align}

Where $F(\nu_x,\nu_y)$ is the Fourier transform of $f(x,y)$.

\begin{align}
F(\nu_x,\nu_y) = \int \int f(x,y) \exp[+i 2\pi (\nu_x x +\nu_y y)] dx dy
\end{align}

We then have that 

\begin{align}
g(x,y) = \int \int H(\nu_x, \nu_y)F(\nu_x,\nu_y) \exp[-i 2\pi (\nu_x x +\nu_y y)] d\nu_x d\nu_y
\end{align}

We multiply each harmonic component by its corresponding transfer function.

\subsection{Impulse Response}
The impulse response is not difficult to work out now that we know the transfer function. The impulse response is $g(x,y) = h(x,y)$ evaluated when $f(x,y) = \delta(x)\delta(y)$. Note that $f(x,y)$ has dimensions of $[L^{-2}]$ here. This is indicative of the fact that we must always integrate over the impulse response function to get a meaningful physical quantity. In that case we have that $F(\nu_x,\nu_y) = 1$. The Fourier transform is unity everywhere. We then see that 

\begin{align}
g(x,y) = \int \int H(\nu_x, \nu_y) \exp[-i 2\pi (\nu_x x +\nu_y y)] d\nu_x d\nu_y
\end{align}

This is the expression for the inverse Fourier transform of the transfer function. In the paraxial case we have 

\begin{align}
H(\nu_x,\nu_y) &=  H_0 \exp\left[ i \pi \lambda d (\nu_x^2+\nu_y^2)\right]
\end{align}

We need to take the Fourier transform of a Gaussian here. This can be done by hand or in Mathematica

\begin{align}
h(x,y) = \frac{i}{d\lambda} \exp[-ikd] \exp\left[-ik \frac{x^2+y^2}{2d}\right] = h_0 \exp\left[-ik \frac{x^2+y^2}{2d}\right]
\end{align}

\begin{align}
h_0 = \frac{i}{d\lambda} \exp[-ikd]
\end{align}

Interestingly we see that this is the expression for a paraboloidal wave originating from the origin. That is to say: the delta impulse at the origin emits a paraboloidal wave. If multiple (or infinite) number of delta functions are present then the output will be the superposition of paraboloidal waves emitted from each point. This is a mathematical statement of Huygen's principle that each point in space acts as a source for a spherical wave.

I'll point out that

\begin{align}
\mathcal{FT}^{-1}[H(\nu_x,\nu_y)](x,y) = h(x,y)
\end{align}

The impulse response function is the inverse Fourier transform of the transfer function. I note also that the impulse response function has dimensions of $[L^{-2}]$.

\section{Free Space as Spatial Fourier Filter}

The idea here is simpler than it sounds. Early in the Fourier Optics section we realized that different Fourier components, $\nu_x$, and $\nu_y$, in $f(x,y)$ arise because $U(x,y,z)$ is composed of a variety of plane waves traveling at different angles to the optical axis. The point then is that if you let these plane waves propagate for some distance $d$ the plane waves traveling at larger angles will end up further away from the optical axis than ones traveling at a smaller angle. So if you want to know the value of some Fourier component of $f(x,y)$ such as $F(\nu_x,\nu_y)$ at $z=0$ then you can look at the appropriate point in space at plane $z=d$ and readout $g(x,y)$ (for the right values of $(x,y)$) to determine the answer.

In words: earlier we found $g(x,y)$ by taking the inverse fourier transform of the product of the transfer function with the Fourier transform of $f(x,y)$. According to the convolution theorem this is equivalent to taking the convolution of the impulse response function (inverse Fourier transform of transfer function) with the input function. That is

\begin{align}
g(x,y) &= \int \int f(x',y') h(x-x',y-y') dx'dy'\\
&= h_0 \int \int f(x',y') \exp\left[-ik \frac{(x-x')^2+(y-y')^2}{2d}\right] dx' dy'\\
&= h_0 \exp\left[-i\pi \frac{x^2+y^2}{\lambda d}\right]\\
\times& \int \int f(x',y') \exp\left[-i2\pi \left(\frac{x}{\lambda d} x' + \frac{y}{\lambda d} y'\right)\right] \exp\left[-i \pi \frac{x'^2+y'^2}{\lambda d} \right] dx' dy'
\end{align}

I will rapidly make two assumptions. First I'll neglect terms like $\frac{x'^2}{\lambda d}$. This amounts to saying the input field is confined to a small region of radius $b$ and that $N'_F = \frac{b^2}{\lambda d} \ll1$, the input plane Fresnel number is very small.. I will also neglect terms like $\frac{x^2}{\lambda d}$ which says that we are only looking at points in a region of radius $a$ close to the optical axis in the output plane satisfying $N_F = \frac{a^2}{\lambda d} \ll1$, that the Fresnel number in the output plane is also very small. These two approximations constitute the Fraunhofer approximation which is more restrictive than the Fresnel approximation. Recall that the Fresnel approximation can be valid even if $N_F>1$ if is guaranteed that $\theta_m$ is very small. We then find, in the Fraunhofer approximation:

\begin{align}
g(x,y) = h_0\int \int f(x', y') \exp\left[ +i2\pi\left(\frac{x}{\lambda d}x' + \frac{y}{\lambda d} y'\right)\right] dx' dy'
\end{align}

We see that this is exactly the expression for

\begin{align}
g(x,y) =h_0 F\left(\frac{x}{\lambda d}, \frac{y}{\lambda d}\right)
\end{align}

That is, if we want to know $F(\nu_x,\nu_y)$, the Fourier component for spatial frequencies $\nu_x, \nu_y$ in the input plane, we can just look at the local field amplitude at the output plane $g(x,y)$ with $x = \nu_x \lambda d$ and $y = \nu_y \lambda d$. Note this makes a bit of sense because, for example, $\nu_x \lambda = \theta_x$ so, in a small angle approximation, we expect a ray starting at the origin to be at location $\theta_x d$ after the wave propagates distance $d$.


\section{Diffraction limit}

The diffraction limit states that, in the far-field, the best resolution obtainable by an optical system is of order $\lambda$. There are one or two root causes for this fundamental limit, each of which has been referenced earlier in this write up. 

1) The finite size of the imaging optic. The argument here is, roughly, that small features imply the presence of large transverse spatial frequencies, $\nu_x$ and $\nu_y$. 
These correspond to waves traveling at large angles to the optical axis. 
If the lens has finite aperture, then due to the distance between the lens and the objective, the finite aperture of the lens will act as a spatial filter which cuts out the high frequency components - the large angle waves `miss' the lens since they travel at too large of an angle to be collected. Thus, when we look in the Fourier plane we will not see the high frequency components, and if we add another lens to take the inverse Fourier transform of the Fourier plane, we will see a spatially low pass filtered version of the original image resulting in lower resolution.

This argument is certainly the reason why the resolution of optical components in the paraxial or low-NA regime is limited. However, I have always been puzzled by two questions which I'll raise below. I think the above answer 1) does partially address the two questions a) and b) below but there is a little bit more to the diffraction limit story. 

a) Why couldn't you make a bigger lens to capture higher spectral components and get better resolution? In fact, in this way, naively, you could imagine getting arbitrarily better spatial resolution with a larger lens.

b) It is possible to write down the formula for a gaussian beam with a waist smaller than $\lambda$ and determine its shape. Isn't this a solution to the Helmholtz equation and shouldn't you be able to shape it with a lens? For example, at some distance (larger than the Rayleigh range) away from the focus of such a tightly focussed Gaussian beam we could calculate the curvature of the beam and the beam waist. It would not be difficult to then construct a lens that takes an input beam and creates a beam with the desired curvature and waist at the output. When this beam then propagates would it not then be focused to the desired spot size smaller than $\lambda$?

I will actually answer b) first because a bit of digging revealed to me that the answer to this part is actually not so difficult or deep. The answer is that Gaussian beams are not necessarily solutions to the full Helmholtz equation. They are, rather solutions to the paraxial Helmholtz equation. Under certain approximation solutions to the paraxial Helmholtz equation will approximately solve the full Helmholtz equation and thus reflect possible real electromagnetic fields which satisfy Maxwell's equations. However, if those approximations break down then the solution to the paraxial approximation may not satisfy the Helmholtz equation at all. It turns out that for Gaussian beams these assumptions begin to break down precisely when $w_0 \lesssim \lambda$. Thus, if we were to perform the manipulations described above with a lens it is true the field would maybe look like a Gaussian beam initially, but, for reasons having to do with the answer to a) below, we will see that it will not focus down to a spot smaller than $\lambda$.

The answer to question a) brings me to the second, and more fundamental - in my opinion, reason for the diffraction limit:

2) The answer to a) for me is related to the transfer function of free space. It is true that increasing the size of the lens will allow you to capture high Fourier frequencies of the original signal and thus get better resolution. However, though you will always increase your resolution with a bigger lens, you will not be able to get arbitrarily good resolution, rather, you will asymptotically approach a maximal resolution. this limit is due to the cutoff of the transfer function of free space. That is, even waves traveling almost perpendicular to the optical axis only carry information of resolution $~\lambda$. That is, $\nu_x \approx \frac{1}{\lambda}$ but no larger. Higher resolution information is never carried into the far-field anyways so it doesn't matter how good or large of a lens you have, there is no way to capture that information. This is a fundamental realization for me because I was always confused as to if the diffraction limit was something technical about how we use lenses or something fundamental about light. It is clear to me now, that due to the transfer function of free space, it is certainly a fundamental property of light itself. Of course, if you can probe the large wavevector evanescent (near) field you can get higher resolution information.

Let me quickly loop back to question b). There was a question as to what would happen if you used a lens to sharply bend an incident field in such a way as to create a Gaussian beam focused to a spot size smaller than $\lambda$. Let me flip this on its head for a second. Imagine you begin with a spot of your desired size. The question is then what would happen as you look at planes further and further away from this point. The answer is that you would need spatial frequencies larger than $\frac{1}{\lambda}$ to describe this initial state. Then, as you look at planes further from the starting point you would see the large wavevector information would evanescently decay away. My guess is that you would after some time see the same thing you would have seen if your spot had been the same size as $\lambda$ anyways. Back to the lens question. That is to say, when you focus your beam in the far field to try to get the really tight focal spot you won't succeed. You will be able to get transverse spatial frequencies very close to $\frac{1}{\lambda}$, but the propagating field will never have the very large wavevectors necessary to construct a very small spot. It can't due to the transfer function of free space.

Ok, so given all of this I am now going to do some math to justify the statements made in 1). That is, I'll return to the low-NA, paraxial limit and derive the diffraction limit for those sorts of optics.

\subsection{Diffraction Limit Math}

I will consider two different situations. In the first situation I will consider a lens being used to image a point emitter. Here we will see that the spot size of the image is again limited by the size of the imaging lens.In the second situation I will consider using a lens to focus a plane wave to a small spot. In this case the size of the lens will determine how small of a spot can be created. 


\section{Single lens cascaded system}

Consider a thin lens with thickness $\Delta d$ with focal length $f$ place at location $z=d_1$. The input plane is at $z=0$ and the output plane will be at $z = d_1+\Delta d + d_2$. Also consider the lens to have a diameter $D$.
We begin with 

\begin{align}
f_0(x,y) = U(x,y,0)
\end{align}

This field then propagates through free space for distance $d_1$ under convolution with the free space impulse function

\begin{align}
f_1(x',y') = \int \int f_0(x'',y'') h_{d_1}(x'-x'',y'-y'') dx''dy''
\end{align}

This field then passes through the lens at which point is modified by the lens transmittance function but is also spatially clipped by the aperture of the lens.

\begin{align}
f_2(x',y') = f_1(x',y')t(x',y')p(x',y')
\end{align}

Finally, this wave propagates for distance $d_2$, again described by convolution with the free space impulse response function.

\begin{align}
&g(x,y) = f_3(x,y) = \int \int f_2(x',y') h_{d_2}(x-x',y-y') dx'dy'\\
&= \int\int\int\int f_0(x'',y'') h_{d_1}(x'-x'',y'-y'') t(x',y')p(x',y') h_{d_2}(x-x',y-y') dx'dy'dx''dy''
\end{align}

We have

\begin{align}
t(x,y) &= t_n \exp\left[ \frac{i\pi}{\lambda f}(x^2+y^2)\right]\\
t_n &= \exp\left[i 2\pi n\frac{\Delta d}{\lambda} \right]
\end{align}

\begin{align}
h_d(x,y) &= h_{0,d} \exp\left[-\frac{i\pi}{\lambda d} (x^2+y^2)\right]\\
h_{0,d} &= \frac{i}{d \lambda} \exp\left[ -i 2\pi \frac{d}{\lambda}\right]
\end{align}

\begin{align}
p(x,y) = 
\begin{cases}
1 & \text{for } \sqrt{x^2+y^2} \le \frac{D}{2}\\
0 & \text{for } \sqrt{x^2+y^2} > \frac{D}{2}\\
\end{cases}
\end{align}

We begin expanding

\begin{align}
g(x,y) =& h_{0,d_1} t_n h_{0,d_2} \int\int\int\int \Bigg\{f_0(x'',y'')p(x',y')\\
&\times \exp\left[-\frac{i\pi}{\lambda d_1} ((x'-x'')^2+(y'-y'')^2)\right]\\
&\times \exp\left[ \frac{i\pi}{\lambda f}(x'^2+y'^2)\right]\\
&\times \exp\left[-\frac{i\pi}{\lambda d_2} ((x-x')^2+(y-y')^2)\right] \Bigg\} dx'dy'dx''dy''
\end{align}

This formula is quite general and could perhaps be used to numerically calculate images resulting from different objects.

\section{Point Spread Function - input delta}
First I'll calculate the response of the system to a point source emitter at $(x_0,y_0)$ in the object plane. For this I take $f_0(x,y) = \delta(x-x_0)\delta(y-y_0)$. We see immediately that

\begin{align}
g(x,y) =& h_{0,d_1} t_n h_{0,d_2} \int\int \Bigg\{p(x',y')\\
&\times \exp\left[-\frac{i\pi}{\lambda d_1} ((x'-x_0)^2+(y'-y_0)^2)\right]\\
&\times \exp\left[ \frac{i\pi}{\lambda f}(x'^2+y'^2)\right]\\
&\times \exp\left[-\frac{i\pi}{\lambda d_2} ((x-x')^2+(y-y')^2)\right] \Bigg\} dx'dy'
\end{align}

I'll look at the $x$'s in the exponent knowing that the $y$'s will be the same. In the exponent we have

\begin{align}
&\frac{i\pi}{\lambda}\left\{-\frac{1}{d_1}(x'^2+x_0^2-2x_0x') + \frac{1}{f} x'^2 - \frac{1}{d_2}(x^2 + x'^2 - 2xx')\right\}\\
&= \frac{i\pi}{\lambda}\left\{\left(\frac{1}{f}-\frac{1}{d_1}-\frac{1}{d_2}\right) x'^2 - \frac{1}{d_1} x_0^2 -\frac{1}{d_2}x^2 +\frac{2}{d_2} \left(x+\frac{d_2}{d_1}x_0\right)x' \right\}
\end{align}

I will define the magnification of the system $M = -\frac{d_2}{d_1}$ and $x_M = Mx_0$. I'l also define the focusing error $\epsilon$

\begin{align}
\epsilon = \frac{1}{d_2}+\frac{1}{d_1} - \frac{1}{f}
\end{align}

Then I re-write

\begin{align}
&= \frac{i\pi}{\lambda}\left\{-\frac{1}{d_1} x_0^2 -\frac{1}{d_2}x^2 - \epsilon x'^2 + \frac{2}{d_2} \left(x-x_M\right)x' \right\}
\end{align}

Plugging this and the analogous expression for $y$ into the integral:

\begin{align}
g(x,y) =& h_{0,d_1} t_n h_{0,d_2} \exp\left[-\frac{i\pi}{\lambda d_1}(x_0^2+y_0^2)\right]\exp\left[-\frac{i\pi}{\lambda d_2}(x^2+y^2)\right]\\
&\times \int\int \Bigg\{p(x',y') \exp\left[-\frac{i\pi}{\lambda}\epsilon (x'^2+y'^2)\right]\\
&\times \exp\left[i2\pi\left(\frac{x-x_M}{\lambda d_2}x' + \frac{y-y_M}{\lambda d_2}y'\right)\right] \Bigg\} dx'dy'
\end{align}

If we define the generalized pupil function

\begin{align}
p_1(x,y) = p(x,y) \exp\left[-\frac{i\pi}{\lambda}\epsilon(x^2+y^2)\right]
\end{align}

We can see that the integral term

\begin{align}
&\int\int \Bigg\{p(x',y') \exp\left[-\frac{i\pi}{\lambda}\epsilon (x'^2+y'^2)\right]\\
&\times \exp\left[i2\pi\left(\frac{x-x_M}{\lambda d_2}x' + \frac{y-y_M}{\lambda d_2}y'\right)\right] \Bigg\} dx'dy'\\
&= P_1\left(\frac{x-x_M}{\lambda d_2}, \frac{y-y_M}{\lambda d_2}\right)
\end{align}

is the Fourier transform of the generalized pupil function evaluated at the spatial frequencies $\left(\frac{x-x_M}{\lambda d_2}, \frac{y-y_M}{\lambda d_2}\right)$.

Summarizing the final answer by collection non-spatially dependent phase terms into a single constant.

\begin{align}
h_1 = h_{0,d_1} t_n h_{0,d_2} \exp\left[-\frac{i\pi}{\lambda d_1}(x_0^2+y_0^2)\right]
\end{align}


\begin{align}
g(x,y) = h_1 \exp\left[-\frac{i\pi}{\lambda d_2}(x^2+y^2)\right] P_1\left(\frac{x-x_M}{\lambda d_2}, \frac{y-y_M}{\lambda d_2}\right)
\end{align}

\subsection{Interpreting the final answer}

A few points for intuition here. First suppose we are in focus so that $\epsilon = 0$ meaning $p_1(x,y) = p(x,y)$. Also let $x_0 = y_0 = 0$. We then have $P_1(\nu_x,\nu_y) = P(\nu_x,\nu_y)$. We can estimate the width of this function. Since $p(x,y)$ has a width of order $~D$ then $P(\nu_x,\nu_y)$ will have a width of order $~\frac{1}{D}$. This means $P_1(\nu_x,\nu_y)$ is small by the time $\nu_x = \frac{x}{\lambda d_2} \approx \frac{1}{D}$ or

\begin{align}
x \approx d_2 \frac{\lambda}{D}
\end{align}

I want to explore the effect of this on the phase term involving $x^2$ outside the integral. Plugging this value in for $x$ we get

\begin{align}
\pi \frac{x^2}{\lambda d_2} &\approx \pi \frac{d_2^2}{\lambda d_2} \frac{\lambda^2}{D^2} = \pi \frac{d_2}{D}\frac{\lambda}{D}\\
&\approx 3 \frac{10^{-1} \text{m}}{2 \times 10^{-2}\text{m}} \frac{10^{-6}\text{m}}{2 \times 10^{-2}\text{m}} \approx 10^{-3} \ll 1
\end{align}

Plugging in a few realistic values for parameters. We see that this phase term is negligible. Neglecting the $x^2$ phase term, and re-writing (still letting $\epsilon = x_0 = 0$)
\begin{align}
g(x,y) = h_1 P\left(\frac{x}{\lambda d_2}, \frac{y}{\lambda d_2}\right)
\end{align}

Succinctly

\textbf{Under ideal conditions we see that the point spread function of a lens is equal to the Fourier Transform of the pupil function of the lens aperture}


Let's plug in a few more of the non-idealities. First let $\epsilon \neq 0$.
We see that the point spread function is equal to the Fourier transform of the generalized pupil function. Let's consider the difference between the pupil function and the generalized pupil function.

\begin{align}
p_1(x,y) = p(x,y) \exp\left[-\frac{i\pi}{\lambda}\epsilon(x^2+y^2)\right]
\end{align}

The larger $\epsilon$ gets the more narrowly peaked becomes the exponential phase term. This means that $p_1(x,y)$ has small spatial features and that it's Fourier transform, $P_1(\nu_x,\nu_y)$ must have support at large spatial frequencies. This means that the point spread function, $g(x,y)$ will cover a larger area. This is an undesirable feature since we seek an accurate representation of the infinitely small point source, so we hope for a point spread function which is as small as possible. This means we seek to focus the lens by satisfying the lens maker equation:

\begin{align}
\frac{1}{f} = \frac{1}{d_1} + \frac{1}{d_2}
\end{align}

Let's add back in the fact that the point source may not be exactly on the optical axis of the optical element: $x_0 \neq 0$. We see now our general formula that

\begin{align}
g(x,y) = h_1 P_1\left(\frac{x-x_M}{\lambda d_2}, \frac{y-y_M}{\lambda d_2}\right)
\end{align}

We see that the image spot from a point at $(x_0,y_0)$ appears as the Fourier transform of the generalized pupil function shifted to be centered at location $(x_M,y_M) = M(x_0,y_0) = -\frac{d_2}{d_1}(x_0,y_0)$. the image is inverted and magnified by $M$. However, it is also `blurred' by the Fourier transform of the pupil function.

\subsection{Fourier Transform of the pupil function}
Let's calculate the Fourier transform of the pupil function.

\begin{align}
P(\nu_x,\nu_y) = \int \int p(x,y) \exp[i 2\pi (\nu_x x + \nu_y y)] dx dy
\end{align}

let

\begin{align}
x &= r_x \cos(\theta_x)\\
y &= r_x \sin(\theta_x)\\
r_x &= \sqrt{x^2+y^2}
\end{align}

and

\begin{align}
\nu_x &= r_{\nu} \cos(\theta_{\nu})\\
\nu_y &= r_{\nu} \sin(\theta_{\nu})\\
r_{\nu} &= \sqrt{\nu_x^2+\nu_y^2}
\end{align}

and performing the change of variables so that

\begin{align}
P(\nu_x,\nu_y) &= \int_{r_x=0}^{\frac{D}{2}} \int_{\theta=-\pi}^{\pi} r_x \exp[i2\pi r_x r_{\nu} (\cos(\theta_x)\cos(\theta_{\nu}) + \sin(\theta_x)\sin(\theta_{\nu}))]  d\theta_x dr_x\\
&= \int_{r_x=0}^{\frac{D}{2}} \int_{\theta=-\pi}^{\pi} r_x \exp[i2\pi r_xr_{\nu} \cos(\theta_x-\theta_{\nu})]d\theta_xdr_x\\
&= \int_{r_x=0}^{\frac{D}{2}} \int_{\theta=-\pi}^{\pi} r_x \exp[i2\pi r_xr_{\nu} \cos(\theta_x)] d\theta_xdr_x
\end{align}

Where I have used the fact that the integral is over a full period of $2\pi$ of $\cos(\theta_x)$ so an arbitrary constant can be added to $\theta_x$ without changing value of the integral. Namely we can add $\theta_{\nu}$ to $\theta_x$ to cancel out the existing factor. Recalling Bessel's integral:

\begin{align}
J_0(z) = \frac{1}{2\pi}\int_{\theta=-\pi}^{\pi} \exp[i z\cos(\theta)] d\theta
\end{align}

We can perform the $\theta_x$ integral to find

\begin{align}
P(\nu_x,\nu_y) = 2\pi \int_{r_x = 0}^{\frac{D}{2}} r_x J_0(2\pi r_{\nu} r_x) dr_x
\end{align}

Changing variables by $z = 2\pi r_{\nu} r_x$

\begin{align}
P(\nu_x,\nu_y) = \frac{1}{2\pi r_{\nu}^2} \int_{z=0}^{\pi r_{\nu} D} z J_0(z) dz
\end{align}

And using the identity

\begin{align}
\int z J_0(z) dz = zJ_1(z)
\end{align}

we get

\begin{align}
P(\nu_x,\nu_y) = \frac{1}{2\pi r_{\nu}^2} zJ_1(z)\bigg|_{z=0}^{\pi r_{\nu} D} = \frac{D}{2r_{\nu}}J_1(\pi r_{\nu}D)
\end{align}

To put it all together:

\begin{align}
g(x,y) &= h_1 P\left(\frac{x-x_M}{\lambda d_2}, \frac{y-y_M}{\lambda d_2}\right)\\
\end{align}

we note 
\begin{align}
r_{\nu} = \sqrt{\left(\frac{x-x_M}{\lambda d_2}\right)^2 + \left(\frac{y-y_M}{\lambda d_2}\right)^2} = \frac{1}{\lambda d_2} \sqrt{(x-x_M)^2 + (y-y_M)^2} = \frac{\Delta r_x}{\lambda d_2}
\end{align}

Summarizing

\begin{align}
g(x,y) = \frac{1}{\lambda d_1} \frac{1}{\lambda d_2} \frac{\lambda d_2 D}{2 \Delta r_x}J_1\left(\pi \frac{\Delta r_x D}{\lambda d_2}\right)
\end{align}

Where I have neglected all constant phase terms. Taking after Saleh and Teich problem 4.4-1 I define

\begin{align}
|h(0,0)| = \frac{D^2}{4\lambda^2 d_1 d_2}
\end{align}

so that

\begin{align}
g(x,y) = h(0,0) 2\frac{J_1\left(\pi \frac{\Delta r_x D}{\lambda d_2}\right)}{\pi \frac{\Delta r_x D}{\lambda d_2}}
\end{align}

\subsection{Rayleigh Criterion}

The Rayleigh criterion states that two point emitters can be resolved if the maximum of the image of the second emitter is beyond the first minimum of the first emitter. The first zero of the Bessel function $J(z)$ occurs when $z = 3.8317$. This means the necessary value for $\Delta r_x$ is found by

\begin{align}
\pi \frac{\Delta r_x D}{\lambda d_2} &= 3.8317\\
\Delta r_x & = 1.22 \lambda \frac{d_2}{D}
\end{align}

for small numerical aperture, ($d_2 \gg D$) we have that $\frac{D}{d_2} \approx 2\theta\approx 2 \sin(\theta) \approx 2\text{NA}$ where $\theta$ is the acceptance angle for the lens and $NA$ is the numerical aperture. In this case wee

\begin{align}
\Delta r_x = 0.61 \frac{\lambda}{\text{NA}} \approx \frac{\lambda}{2\text{NA}}
\end{align}

This is the celebrated Abbe diffraction limit which states that the resolving power of a lens is related to the wavelength of light used for imaging and the numerical aperture of the lens. The maximum possible resolution is approximately $\frac{\lambda}{2}$, consistent with our understanding that spatial frequencies of this order aren't transmitted by free space.

\subsection{Convolution}

Note that the function $g(x,y)$ depends on the location of the unit impulse, $(x_0,y_0)$. It would perhaps rightly be written as $g(x,y,x_0,y_0)$. To calculate the response of the system to a functional input $f_0(x,y)$ which is not just a delta function we would calculate the convolution

\begin{align}
g_{out}(x,y) = \int\int f_0(x',y') g(x-x',y-y',x',y') dx'dy'
\end{align}

We see the convolution is a bit complicated by the fact that the impulse response depends on the location of the input pulse. But, we actually know that the dependence is simply related to the magnification. To recover a convolution type arrangement we can write

\begin{align}
g_{out}(x,y) = \int \int f_0(x',y') g(x-Mx',y-My') dx' dy'
\end{align}

That is we include the magnification in the convolution. Thus a point impulse at $(x_0,y_0)$ will create a response centered at $M(x_0,y_0)$ where $M = -\frac{d_2}{d_1}$.

\section{Focused Beam}

The second important application for a lens is to take an input field and focus it down to a small size to address a small sample. We begin with a plane wave at $z=0$, propagating through distance $z=d_1$ of free space until it impinges upon a lens of thickness $\delta d$ and focal length $f$ and diameter $D$. It then propagates through free space for distance $d_2$ until it is in the image plane at $z=d_1+\Delta d + d_2$. We begin with

\begin{align}
f_0(x,y) = \int \int F(\nu_x, \nu_y) \exp[-i 2\pi (\nu_x x + \nu_y y)] d\nu_x d\nu_y
\end{align}

Propagating through free space:

\begin{align}
f_1(x,y) = \int \int H_{d_1}(\nu_x,\nu_y) F(\nu_x, \nu_y) \exp[-i 2\pi (\nu_x x + \nu_y y)] d\nu_x d\nu_y
\end{align}

Crossing the lens:

\begin{align}
f_2(x,y) = \int \int H_{d_1}(\nu_x,\nu_y) F(\nu_x, \nu_y) p(x,y)t(x,y) \exp[-i 2\pi (\nu_x x + \nu_y y)] d\nu_x d\nu_y
\end{align}

And finally convolution with the free space impulse response function for the second free space propagation.

\begin{align}
g(x,y) = f_3(x,y) = \int \int f_2(x',y') h_{d_2}(x-x',y-y') dx' dy'
\end{align}

Note

\begin{align}
H_{d}(\nu_x,\nu_y) &= H_{0,d} \exp[i\pi \lambda d(\nu_x^2+\nu_y^2)]\\
H_{0,d} &= \exp\left[-i 2\pi \frac{d}{\lambda}\right]
\end{align}

expanding:

\begin{align}
g(x,y) &= H_{0,d_1} t_n h_{0,d_2} \int \int \int \int \Bigg\{F(\nu_x,\nu_y) p(x',y')\\
&\times \exp[-i 2\pi (\nu_x x' + \nu_y y')]\\
&\times \exp[i\pi \lambda d_1(\nu_x^2+\nu_y^2)]\\
&\times \exp\left[ \frac{i \pi}{\lambda f} (x'^2+y'^2)\right]\\
&\times \exp\left[ -\frac{i \pi}{\lambda d_2} ((x-x')^2+(y-y')^2)\right] \Bigg\}dx' dy' d\nu_x d\nu_y
\end{align}

We pick out the phase factors lettings $\lambda d_2 \nu_x = x_i$.

\begin{align}
&i\pi\lambda d_1 \nu_x^2 + \frac{i\pi}{\lambda}\left\{-\frac{2}{d_2}x_ix' + \frac{1}{f}x'^2 - \frac{1}{d_2}(x^2+x'^2-2xx') \right\}\\
&= i\pi\lambda d_1 \nu_x^2 -\frac{i\pi}{\lambda d_2} x^2 + \frac{i\pi}{\lambda}\left\{\left(\frac{1}{f}-\frac{1}{d_2}\right)x'^2 + \frac{2}{d_2}(x-x_i)x' \right\}
\end{align}

We pull the first two exponential factors out. I'll also recognize

\begin{align}
\epsilon_{\infty} = \frac{1}{d_2}-\frac{1}{f}
\end{align}

This is the focusing error assuming the input is focused and infinity. This is what we expect for focusing down a plane wave. We get

\begin{align}
g(x,y) &= H_{0,d_1} t_n h_{0,d_2}\\
&\times \exp\left[-\frac{i\pi}{\lambda d_2}(x^2+y^2)\right]\\
&\times \int \int \int \int \Bigg\{F(\nu_x,\nu_y) p(x',y')\\
&\times \exp\left[i\pi \lambda d_1 (\nu_x^2+\nu_y^2) \right]\\
&\times \exp\left[-\frac{i\pi}{\lambda}\epsilon_{\infty}(x'^2+y'^2)\right]\\
&\times \exp\left[i2\pi \left(\frac{x-x_i}{\lambda d_2}x' + \frac{y-y_i}{\lambda d_2}y'\right)\right]\Bigg\} dx'dy'd\nu_xd\nu_y
\end{align}

We can again define a generalized pupil function

\begin{align}
p_1(x,y) = p(x,y)\times \exp\left[-\frac{i\pi}{\lambda}\epsilon_{\infty}(x'^2+y'^2)\right]
\end{align}

and we can then identify the $x', y'$ integrals as the Fourier transform of $p_1(x,y)$

\begin{align}
P_1\left(\frac{x-x_i}{\lambda d_2},\frac{y-y_i}{\lambda d_2}\right) = \int\int p_1(x',y') \exp\left[i2\pi \left(\frac{x-x_i}{\lambda d_2}x' + \frac{y-y_i}{\lambda d_2}y'\right)\right] dx'dy'
\end{align}

For the final answer we find

\begin{align}
g(x,y) &= H_{0,d_1} t_n h_{0,d_2}\\
&\times \exp\left[-\frac{i\pi}{\lambda d_2}(x^2+y^2)\right]\\
&\times \int\int \Bigg\{F(\nu_x,\nu_y) \exp\left[i\pi \lambda d_1 (\nu_x^2+\nu_y)^2 \right]\\
&\times P_1\left(\frac{x-x_i}{\lambda d_2},\frac{y-y_i}{\lambda d_2}\right) \Bigg\}d\nu_x d\nu_y
\end{align}

Let me re-write this to make something more clear.

\begin{align}
g(x,y) &= H_{0,d_1} t_n h_{0,d_2} \frac{1}{(\lambda d_2)^2}\\
&\times \int\int \Bigg\{F(\frac{x_i}{\lambda d_2},\frac{y_i}{\lambda d_2})\\
&\times \exp\left[\frac{i\pi}{\lambda d_2}\frac{d_1}{d_2} (x_i^2+y_i^2) \right]\\
&\times \exp\left[-\frac{i\pi}{\lambda d_2}(x^2+y^2)\right]\\
&\times P_1\left(\frac{x-x_i}{\lambda d_2},\frac{y-y_i}{\lambda d_2}\right) \Bigg\}dx_i dy_i
\end{align}

\begin{align}
g(x,y) &= H_{0,d_1} t_n h_{0,d_2} \frac{1}{(\lambda d_2)^2}\\
&\times \int\int \Bigg\{F\left(\frac{x_i}{\lambda d_2},\frac{y_i}{\lambda d_2}\right)\\
&\times \exp\left[\frac{i\pi}{\lambda d_2}(\left(\frac{d_1}{d_2}x_i^2-x^2\right)+\left(\frac{d_1}{d_2}y_i^2-y^2\right)) \right]\\
&\times P_1\left(\frac{x-x_i}{\lambda d_2},\frac{y-y_i}{\lambda d_2}\right) \Bigg\}dx_i dy_i
\end{align}

The key here is the the $P_1$ function will be small unless

\begin{align}
x-x_i \lesssim \frac{\lambda d_2}{D}
\end{align}

Under this condition, and supposing $d_1 \approx d_2$ we can write

\begin{align}
\frac{d_1}{d_2}x_i^2 - x^2 \approx 2x_i(x-x_i) \approx 2x_i \frac{\lambda d_2}{D}
\end{align}

The whole phase factor is then

\begin{align}
\frac{1}{\lambda d_2} \frac{\lambda d_2}{D} 2x_i = 2\frac{x_i}{D}
\end{align}

This should be small assuming we are looking near the optical axis. Furthermore, in the case that we are considering a single plane wave then $F(\nu_x,\nu_y)$ is  a delta function and this term just becomes a phase outside the integral. If we do go ahead and ignore that phase then we can see

\begin{align}
h_2' = H_{0,d_1}t_nh_{0,d_2}
h_2 = \frac{1}{(\lambda d_2)^2}
\end{align}

\begin{align}
g(x,y) = h_2'\int \int F\left(\frac{x_i}{\lambda d_2},\frac{y_i}{\lambda d_2}\right) P_1\left(\frac{x-x_i}{\lambda d_2},\frac{y-y_i}{\lambda d_2}\right)dx_idy_i
\end{align}

Consider the case $F\left(\frac{x_i}{\lambda d_2},\frac{y_i}{\lambda d_2}\right) = (\lambda d_2)^2 \delta(x_i-x_0)\delta(y_i-y_0)$ and $\epsilon_{\infty} = 0$. Then

\begin{align}
g(x,y) = h_2 P\left(\frac{x-x_0}{\lambda d_2}, \frac{y-y_0}{\lambda d_2}\right) = h_2 \frac{\lambda d_2 D}{2\Delta r_x}J_1\left(\pi \frac{\Delta r_x D}{\lambda d_2}\right)
\end{align}

where $\Delta(r_x) = \sqrt{(x-x_0)^2+(y-y_0)^2}$.
We again see an Airy disk shaped spot for the focal spot. The first minimum occurs when

\begin{align}
\pi \frac{\Delta r_x D}{\lambda d_2} &= 3.8317\\
\Delta r_x & = 1.22 \lambda \frac{d_2}{D} \approx 1.22 \frac{\lambda}{2\theta} \approx 0.61 \frac{\lambda}{\text{NA}}
\end{align}

We see the Abbe diffraction limit arise again. However the mechanisms are slightly different. In the first case a point emitter emitted waves and the waves were low pass filtered by the lens aperture resulting in a spot with a finite width in the image plane representing the object. In this second case we are illuminating the lens with a plane wave. The lens then adds Fourier components to this wave by focusing it down but the focused beam still ends up with a finite width because it doesn't gain all spatial frequencies. However, even though these different situations are a bit different mathematically, we see the same resolution limit in both cases.

Returning to

\begin{align}
g(x,y) = h_2'\int \int F\left(\frac{x_i}{\lambda d_2},\frac{y_i}{\lambda d_2}\right) P_1\left(\frac{x-x_i}{\lambda d_2},\frac{y-y_i}{\lambda d_2}\right)dx_idy_i
\end{align}

Consider now the case where $P_1\left(\frac{x-x_i}{\lambda d_2},\frac{y-y_i}{\lambda d_2}\right) = (\lambda d_2)^2\delta(x_i)\delta(y_i)$. We can consider this to be the case when the spatial frequencies of the input mode are small compared to the cutoff provided by the lens of $\approx \frac{1}{D}$. In this case we get

\begin{align}
g(x,y) = h_2 F\left(\frac{x}{\lambda d_2},\frac{y}{\lambda d_2}\right)
\end{align}

we see that the field amplitude at point $(x,y)$ is proportional to the value of the Fourier transform of the input field at spatial frequency $\left(\frac{x}{\lambda d_2},\frac{y}{\lambda d_2}\right)$.


\end{document}